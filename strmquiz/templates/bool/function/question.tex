

Etudier la fonction suivante:



Study the following function:

\begin{arab}[utf]
ادرس الدالة الآتية:
\end{arab}
$F(A,B,C,D) ={{  sop_quest }}$



 



${{ function_name }} = {{ minterms }}$

\textbf{Don't Care }

${{ function_name }} = {{ dontcares }}$



\textbf{Truth Table \aRL{جدول الحقيقة}}
{{ truthtable }}


\begin{tabular}{|c|c|c|c||c|}
\hline
{{ variables[0] }} & {{ variables[1] }} & {{ variables[2] }} & {{ variables[3] }} & F \\

    {% if i%4==0 %}
        \hline
    
  
  
  {{ bit_list[0] }} & {{ bit_list[1] }} & {{ bit_list[2] }} & {{ bit_list[3] }} & {{ '1' if i in minterms else 'x' if i in dontcares else 0 }} \\

\hline
\end{tabular}


\textbf{Canonical Forms \aRL{الأشكال القانونية}}
\begin{itemize}
\item ${{ function_name }}(A,B,C,D) =  {{  dnf_dict.formatted }}$
\item ${{ function_name }}(A,B,C,D) = {{  cnf_dict.formatted }}$
 \item ${{ function_name }}(A,B,C,D) =  \sum({{  minterms| join(", ") }})$
 \item ${{ function_name }}(A,B,C,D) =  \prod({{  maxterms| join(", ") }})$
\end{itemize}




 
 \textbf{ NAND forms  \aRL{بوابات نفي الوصل}}
{{ nand_sop }}

{{ nand_dnf }}

 


 \textbf{ NOR forms  \aRL{بوابات نفي الفصل}}
{{ nor_pos }}

{{ nor_cnf }}

 




\textbf{Karnaugh Table \aRL{جدول كارنوف}}

\begin{karnaugh-map}[4][4][1][{{ cd }}][{{ ab }}]
  \minterms{ {{ minterms | join(", ") }} }
  \maxterms{ {{ maxterms | join(", ") }} }
  
    \terms{ {{ dontcares | join(", ") }} }{X}
  

    {{ formatted_simplification |join("\n") }}

 \end{karnaugh-map}

    Simplified Sum of products : ${{ sop_dict.formatted }}$\\
    Simplified product of sums : ${{ pos_dict.formatted }}$\\

\textbf{Logic diagram \aRL{المخطط المنطقي}}

{{ logicdiagram }}






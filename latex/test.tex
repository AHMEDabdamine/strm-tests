
\section{Question}

\paragraph{Q1}


F3(A, B, C, D)= $[3, 4, 5, 6, 7, 8, 9, 10]$

DONT CARE= $[11, 12, 13, 14, 15]$

F2(A, B, C, D)= $[0, 1, 2, 7, 8, 9, 10, 15]$

DONT CARE= $[]$

F1(A, B, C, D)= $[1, 2, 5, 6, 9, 10, 13, 14]$

DONT CARE= $[]$

F0(A, B, C, D)= $[0, 2, 4, 6, 8, 10, 12, 14]$

DONT CARE= $[]$


\hrule width 1\linewidth\pagebreak
\subsection{Correction}

\paragraph{Q1}

\begin{enumerate}
 \item Définition des entrées et des sorties \aRL{تعريف المداخل والمخارج}

         \begin{itemize}

         \item Les entrées \aRL{المداخل}:

         \begin{itemize}
         \item A: \qquad 'on' noté 1 \qquad 'off' noté 0
 \item B: \qquad 'on' noté 1 \qquad 'off' noté 0
 \item C: \qquad 'on' noté 1 \qquad 'off' noté 0
 \item D: \qquad 'on' noté 1 \qquad 'off' noté 0
 \end{itemize}

          \item  Les sorties \aRL{المخارج}

          \begin{itemize}
          \item F3: \qquad 'on' noté 1\qquad 'off' noté  0
\item F2: \qquad 'on' noté 1\qquad 'off' noté  0
\item F1: \qquad 'on' noté 1\qquad 'off' noté  0
\item F0: \qquad 'on' noté 1\qquad 'off' noté  0
\end{itemize}

         \end{itemize}
\item Table de vérité \aRL{جدول الحقيقة}\\
%\begin{table}
        \begin{tabular}{|c|c|c|c|c||c|c|c|c|}
    \toprule
        N° &A & B & C & D & F3 & F2 & F1 & F0\\ \midrule0 & 0 & 0 & 0 & 0 & 0 & 1 & 0 & 1\\1 & 0 & 0 & 0 & 1 & 0 & 1 & 1 & 0\\2 & 0 & 0 & 1 & 0 & 0 & 1 & 1 & 1\\3 & 0 & 0 & 1 & 1 & 1 & 0 & 0 & 0\\\midrule4 & 0 & 1 & 0 & 0 & 1 & 0 & 0 & 1\\5 & 0 & 1 & 0 & 1 & 1 & 0 & 1 & 0\\6 & 0 & 1 & 1 & 0 & 1 & 0 & 1 & 1\\7 & 0 & 1 & 1 & 1 & 1 & 1 & 0 & 0\\\midrule8 & 1 & 0 & 0 & 0 & 1 & 1 & 0 & 1\\9 & 1 & 0 & 0 & 1 & 1 & 1 & 1 & 0\\10 & 1 & 0 & 1 & 0 & 1 & 1 & 1 & 1\\11 & 1 & 0 & 1 & 1 & 0 & 0 & 0 & 0\\\midrule12 & 1 & 1 & 0 & 0 & 0 & 0 & 0 & 1\\13 & 1 & 1 & 0 & 1 & 0 & 0 & 1 & 0\\14 & 1 & 1 & 1 & 0 & 0 & 0 & 1 & 1\\15 & 1 & 1 & 1 & 1 & 0 & 1 & 0 & 0\\\bottomrule
        \end{tabular}
        %%\end{table}
        \item Les formes canoniques \aRL{الأشكال القانونية}\\
\begin{itemize}
\item La fonction F3(A, B, C, D) \aRL{الدالة}\\
%%F3(A, B, C, D) =$[3, 4, 5, 6, 7, 8, 9, 10]$


\begin{itemize}
\item La première forme canonique: \aRL{الشكل القانوني الأول}

F3(A, B, C, D) = $\bar A.\bar B.C.D + \bar A.B.\bar C.\bar D + \bar A.B.\bar C.D + \bar A.B.C.\bar D + \bar A.B.C.D + A.\bar B.\bar C.\bar D + A.\bar B.\bar C.D + A.\bar B.C.\bar D$
\item La deuxième forme canonique: \aRL{الشكل القانوني الثاني}≠

 F3(A, B, C, D) = $(A+B+C+D) . (A+B+C+\bar D) . (A+B+\bar C+D) . (\bar A+B+\bar C+\bar D) . (\bar A+\bar B+C+D) . (\bar A+\bar B+C+\bar D) . (\bar A+\bar B+\bar C+D) . (\bar A+\bar B+\bar C+\bar D)$
\item La première forme canonique; \aRL{الشكل القانوني الرقمي الأول}

F3(A, B, C, D) = $\sum [3, 4, 5, 6, 7, 8, 9, 10]$
\item La deuxième forme canonique;  \aRL{الشكل القانوني الرقمي الثاني}

 F3(A, B, C, D) = $\prod [0, 1, 2, 11, 12, 13, 14, 15]$
\end{itemize}
\item La fonction F2(A, B, C, D) \aRL{الدالة}\\
%%F2(A, B, C, D) =$[0, 1, 2, 7, 8, 9, 10, 15]$


\begin{itemize}
\item La première forme canonique: \aRL{الشكل القانوني الأول}

F2(A, B, C, D) = $\bar A.\bar B.\bar C.\bar D + \bar A.\bar B.\bar C.D + \bar A.\bar B.C.\bar D + \bar A.B.C.D + A.\bar B.\bar C.\bar D + A.\bar B.\bar C.D + A.\bar B.C.\bar D + A.B.C.D$
\item La deuxième forme canonique: \aRL{الشكل القانوني الثاني}≠

 F2(A, B, C, D) = $(A+B+\bar C+\bar D) . (A+\bar B+C+D) . (A+\bar B+C+\bar D) . (A+\bar B+\bar C+D) . (\bar A+B+\bar C+\bar D) . (\bar A+\bar B+C+D) . (\bar A+\bar B+C+\bar D) . (\bar A+\bar B+\bar C+D)$
\item La première forme canonique; \aRL{الشكل القانوني الرقمي الأول}

F2(A, B, C, D) = $\sum [0, 1, 2, 7, 8, 9, 10, 15]$
\item La deuxième forme canonique;  \aRL{الشكل القانوني الرقمي الثاني}

 F2(A, B, C, D) = $\prod [3, 4, 5, 6, 11, 12, 13, 14]$
\end{itemize}
\item La fonction F1(A, B, C, D) \aRL{الدالة}\\
%%F1(A, B, C, D) =$[1, 2, 5, 6, 9, 10, 13, 14]$


\begin{itemize}
\item La première forme canonique: \aRL{الشكل القانوني الأول}

F1(A, B, C, D) = $\bar A.\bar B.\bar C.D + \bar A.\bar B.C.\bar D + \bar A.B.\bar C.D + \bar A.B.C.\bar D + A.\bar B.\bar C.D + A.\bar B.C.\bar D + A.B.\bar C.D + A.B.C.\bar D$
\item La deuxième forme canonique: \aRL{الشكل القانوني الثاني}≠

 F1(A, B, C, D) = $(A+B+C+D) . (A+B+\bar C+\bar D) . (A+\bar B+C+D) . (A+\bar B+\bar C+\bar D) . (\bar A+B+C+D) . (\bar A+B+\bar C+\bar D) . (\bar A+\bar B+C+D) . (\bar A+\bar B+\bar C+\bar D)$
\item La première forme canonique; \aRL{الشكل القانوني الرقمي الأول}

F1(A, B, C, D) = $\sum [1, 2, 5, 6, 9, 10, 13, 14]$
\item La deuxième forme canonique;  \aRL{الشكل القانوني الرقمي الثاني}

 F1(A, B, C, D) = $\prod [0, 3, 4, 7, 8, 11, 12, 15]$
\end{itemize}
\item La fonction F0(A, B, C, D) \aRL{الدالة}\\
%%F0(A, B, C, D) =$[0, 2, 4, 6, 8, 10, 12, 14]$


\begin{itemize}
\item La première forme canonique: \aRL{الشكل القانوني الأول}

F0(A, B, C, D) = $\bar A.\bar B.\bar C.\bar D + \bar A.\bar B.C.\bar D + \bar A.B.\bar C.\bar D + \bar A.B.C.\bar D + A.\bar B.\bar C.\bar D + A.\bar B.C.\bar D + A.B.\bar C.\bar D + A.B.C.\bar D$
\item La deuxième forme canonique: \aRL{الشكل القانوني الثاني}≠

 F0(A, B, C, D) = $(A+B+C+\bar D) . (A+B+\bar C+\bar D) . (A+\bar B+C+\bar D) . (A+\bar B+\bar C+\bar D) . (\bar A+B+C+\bar D) . (\bar A+B+\bar C+\bar D) . (\bar A+\bar B+C+\bar D) . (\bar A+\bar B+\bar C+\bar D)$
\item La première forme canonique; \aRL{الشكل القانوني الرقمي الأول}

F0(A, B, C, D) = $\sum [0, 2, 4, 6, 8, 10, 12, 14]$
\item La deuxième forme canonique;  \aRL{الشكل القانوني الرقمي الثاني}

 F0(A, B, C, D) = $\prod [1, 3, 5, 7, 9, 11, 13, 15]$
\end{itemize}

\end{itemize}
\item Tableaux de Karnough \aRL{مخطط كارنوف}

\begin{itemize}
\item Tableau de Karnough de la fonction  F3 \aRL{مخطط كارنوف للدالة}

\begin{karnaugh-map}[4][4][1][CD][AB]
          \minterms{3, 4, 5, 6, 7, 8, 9, 10}
          \maxterms{0, 1, 2}
        %\autoterms[0]
         \terms{11, 12, 13, 14, 15}{X}
        % simplification
        \implicant{12}{10}
\implicant{4}{14}
\implicant{3}{11}
          %\implicant{5}{15}
          %\implicantedge{8}{8}{10}{10}
          %\implicantedge{8}{8}{10}{10}[8,10]
        \end{karnaugh-map}

\begin{itemize}
\item La forme simplifiée \aRL{الشكل المبسط} 

 F3 = $ a + b + c.d $
\item La deuxième forme simplifiée \aRL{الشكل المبسط الثاني} 

F3 = $(a+b+c).(a+b+d)$

\end{itemize}
\item Tableau de Karnough de la fonction  F2 \aRL{مخطط كارنوف للدالة}

\begin{karnaugh-map}[4][4][1][CD][AB]
          \minterms{0, 1, 2, 7, 8, 9, 10, 15}
          \maxterms{3, 4, 5, 6, 11, 12, 13, 14}
        %\autoterms[0]
         \terms{}{X}
        % simplification
        \implicant{7}{15}
\implicantedge{0}{1}{8}{9}
 \implicantcorner{0}{2}
          %\implicant{5}{15}
          %\implicantedge{8}{8}{10}{10}
          %\implicantedge{8}{8}{10}{10}[8,10]
        \end{karnaugh-map}

\begin{itemize}
\item La forme simplifiée \aRL{الشكل المبسط} 

 F2 = $ b.c.d + \bar b.\bar c + \bar b.\bar d $
\item La deuxième forme simplifiée \aRL{الشكل المبسط الثاني} 

F2 = $(\bar b+c).(\bar b+d).(b+\bar c+\bar d)$

\end{itemize}
\item Tableau de Karnough de la fonction  F1 \aRL{مخطط كارنوف للدالة}

\begin{karnaugh-map}[4][4][1][CD][AB]
          \minterms{1, 2, 5, 6, 9, 10, 13, 14}
          \maxterms{0, 3, 4, 7, 8, 11, 12, 15}
        %\autoterms[0]
         \terms{}{X}
        % simplification
        \implicant{2}{10}
\implicant{1}{9}
          %\implicant{5}{15}
          %\implicantedge{8}{8}{10}{10}
          %\implicantedge{8}{8}{10}{10}[8,10]
        \end{karnaugh-map}

\begin{itemize}
\item La forme simplifiée \aRL{الشكل المبسط} 

 F1 = $ c.\bar d + \bar c.d $
\item La deuxième forme simplifiée \aRL{الشكل المبسط الثاني} 

F1 = $(c+d).(\bar c+\bar d)$

\end{itemize}
\item Tableau de Karnough de la fonction  F0 \aRL{مخطط كارنوف للدالة}

\begin{karnaugh-map}[4][4][1][CD][AB]
          \minterms{0, 2, 4, 6, 8, 10, 12, 14}
          \maxterms{1, 3, 5, 7, 9, 11, 13, 15}
        %\autoterms[0]
         \terms{}{X}
        % simplification
        \implicantedge{0}{8}{2}{10}
          %\implicant{5}{15}
          %\implicantedge{8}{8}{10}{10}
          %\implicantedge{8}{8}{10}{10}[8,10]
        \end{karnaugh-map}

\begin{itemize}
\item La forme simplifiée \aRL{الشكل المبسط} 

 F0 = $ \bar d $
\item La deuxième forme simplifiée \aRL{الشكل المبسط الثاني} 

F0 = $(\bar d)$

\end{itemize}

\end{itemize}
\item Logigrammes \aRL{المخططات المنطقية }\\\section{Schéma globale du circuit}
 \label{logigrammefonctionF3-F2-F1-F0}

 \begin{tikzpicture}

\node (x) at (0, 11.88) {$A$};
\node (y) at (0.5, 11.88) {$B$};
\node (z) at (1, 11.88) {$C$};
\node (w) at (1.5, 11.88) {$D$};

            \node[not gate US, draw, rotate=270] at ($(x) + (0.25, -0.6)$) (notx) {};
            \draw ($(x)+(0,-1ex)$) -| (notx.input); 
            \node[not gate US, draw, rotate=270] at ($(y) + (0.25, -0.6)$) (noty) {};
            \draw ($(y)+(0,-1ex)$) -| (noty.input); 
            \node[not gate US, draw, rotate=270] at ($(z) + (0.25, -0.6)$) (notz) {};
            \draw ($(z)+(0,-1ex)$) -| (notz.input);
            \node[not gate US, draw, rotate=270] at ($(w) + (0.25, -0.6)$) (notw) {};
            \draw ($(w)+(0,-1ex)$) -| (notw.input);
         

      
                \node[and gate US, draw, rotate=0, logic gate inputs=nnnn] at (2.5, 1.20) (xandy1) {};\draw (xandy1.output) -- node[above]{\scriptsize $ \bar D $} ($(xandy1) + (1.8, 0)$);
                % W
\draw [line width=0.25mm,   red] (notw.output)
            -- ([xshift=0cm]notw.output) |- (xandy1.input 4);
\node at (6, 1.20) (xory1) {};


            \draw (xory1) node[above]{\scriptsize $F0$} ($(xory1.east) + (+3ex, 0)$);


            \draw (xandy1.output) -- ([xshift=1.60cm]xandy1.output) |- (xory1);

 

      
                \node[and gate US, draw, rotate=0, logic gate inputs=nnnn] at (2.5, 2.40) (xandy2) {};\draw (xandy2.output) -- node[above]{\scriptsize $ C.\bar D $} ($(xandy2) + (1.8, 0)$);
                % Z
\draw ($(z) + (0, -1ex)$)|- (xandy2.input 3);
% W
\draw [line width=0.25mm,   red] (notw.output)
            -- ([xshift=0cm]notw.output) |- (xandy2.input 4);
 

      
                \node[and gate US, draw, rotate=0, logic gate inputs=nnnn] at (2.5, 3.60) (xandy3) {};\draw (xandy3.output) -- node[above]{\scriptsize $ \bar C.D $} ($(xandy3) + (1.8, 0)$);
                % Z
\draw [line width=0.25mm,   red] (notz.output)
            -- ([xshift=0cm]notz.output) |- (xandy3.input 3);
% W
\draw ($(w) + (0, -1ex)$)|- (xandy3.input 4);
\node[or gate US, draw, rotate=0, logic gate inputs=nnn] at (6, 3.00) (xory2) {};


            \draw (xory2.output) -- node[above]{\scriptsize $F1$} ($(xory2.east) + (+3ex, 0)$);


            \draw (xandy2.output) -- ([xshift=1.60cm]xandy2.output) |- (xory2.input 2);

\draw (xandy3.output) -- ([xshift=1.55cm]xandy3.output) |- (xory2.input 1);

 

      
                \node[and gate US, draw, rotate=0, logic gate inputs=nnnn] at (2.5, 4.80) (xandy4) {};\draw (xandy4.output) -- node[above]{\scriptsize $ B.C.D $} ($(xandy4) + (1.8, 0)$);
                % Y
\draw ($(y) + (0, -1ex)$)|- (xandy4.input 2);
% Z
\draw ($(z) + (0, -1ex)$)|- (xandy4.input 3);
% W
\draw ($(w) + (0, -1ex)$)|- (xandy4.input 4);
 

      
                \node[and gate US, draw, rotate=0, logic gate inputs=nnnn] at (2.5, 6.00) (xandy5) {};\draw (xandy5.output) -- node[above]{\scriptsize $ \bar B.\bar C $} ($(xandy5) + (1.8, 0)$);
                % Y
\draw [line width=0.25mm,   red] (noty.output)
            -- ([xshift=0cm]noty.output) |- (xandy5.input 2);
% Z
\draw [line width=0.25mm,   red] (notz.output)
            -- ([xshift=0cm]notz.output) |- (xandy5.input 3);
 

      
                \node[and gate US, draw, rotate=0, logic gate inputs=nnnn] at (2.5, 7.20) (xandy6) {};\draw (xandy6.output) -- node[above]{\scriptsize $ \bar B.\bar D $} ($(xandy6) + (1.8, 0)$);
                % Y
\draw [line width=0.25mm,   red] (noty.output)
            -- ([xshift=0cm]noty.output) |- (xandy6.input 2);
% W
\draw [line width=0.25mm,   red] (notw.output)
            -- ([xshift=0cm]notw.output) |- (xandy6.input 4);
\node[or gate US, draw, rotate=0, logic gate inputs=nnnn] at (6, 6.00) (xory4) {};


            \draw (xory4.output) -- node[above]{\scriptsize $F2$} ($(xory4.east) + (+3ex, 0)$);


            \draw (xandy4.output) -- ([xshift=1.60cm]xandy4.output) |- (xory4.input 3);

\draw (xandy5.output) -- ([xshift=1.55cm]xandy5.output) |- (xory4.input 2);

\draw (xandy6.output) -- ([xshift=1.60cm]xandy6.output) |- (xory4.input 1);

 

      
                \node[and gate US, draw, rotate=0, logic gate inputs=nnnn] at (2.5, 8.40) (xandy7) {};\draw (xandy7.output) -- node[above]{\scriptsize $ A $} ($(xandy7) + (1.8, 0)$);
                % X
\draw ($(x) + (0, -1ex)$)|- (xandy7.input 1);
 

      
                \node[and gate US, draw, rotate=0, logic gate inputs=nnnn] at (2.5, 9.60) (xandy8) {};\draw (xandy8.output) -- node[above]{\scriptsize $ B $} ($(xandy8) + (1.8, 0)$);
                % Y
\draw ($(y) + (0, -1ex)$)|- (xandy8.input 2);
 

      
                \node[and gate US, draw, rotate=0, logic gate inputs=nnnn] at (2.5, 10.80) (xandy9) {};\draw (xandy9.output) -- node[above]{\scriptsize $ C.D $} ($(xandy9) + (1.8, 0)$);
                % Z
\draw ($(z) + (0, -1ex)$)|- (xandy9.input 3);
% W
\draw ($(w) + (0, -1ex)$)|- (xandy9.input 4);
\node[or gate US, draw, rotate=0, logic gate inputs=nnnn] at (6, 9.60) (xory7) {};


            \draw (xory7.output) -- node[above]{\scriptsize $F3$} ($(xory7.east) + (+3ex, 0)$);


            \draw (xandy7.output) -- ([xshift=1.60cm]xandy7.output) |- (xory7.input 3);

\draw (xandy8.output) -- ([xshift=1.55cm]xandy8.output) |- (xory7.input 2);

\draw (xandy9.output) -- ([xshift=1.60cm]xandy9.output) |- (xory7.input 1);

 \end{tikzpicture}

\end{enumerate}
\pagebreak
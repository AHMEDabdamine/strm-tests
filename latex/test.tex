
\section{Question}


\paragraph{Q1}

    \tikzset{flipflop DMod/.style={flipflop, dot on notQ,
        flipflop def={t1=D, t6=$Q2$,t4={\ctikztextnot{$Q2$}},
           c3=1},
    },
    }

\tikzset{flipflop JKMod/.style={flipflop,  dot on notQ,
            flipflop def={t1=J, t3=K,t6=$Q0$, t4={\ctikztextnot{$Q0$}},
               c2=1},
        },
}

\tikzset{flipflop RSMod/.style={flipflop,  dot on notQ,
                    flipflop def={t1=R,t3=S,t6=$Q1$, t4={\ctikztextnot{$Q1$}},
                       n4=1, c2=0},
                },
}

\tikzset{flipflop JKAMod/.style={flipflop,  dot on notQ,
            flipflop def={t1=J, t3=K,t6=Q, t4={\ctikztextnot{Q}},
                tu={\ctikztextnot{pr}}, nu=1, td={\ctikztextnot{cl}}, nd=1, c2=1},
        }}



\section{Flip}
\textbf{ Truth table of D flipflop }\\
  \begin{tabular}{|c|c|c|}
   \hline Ck & D & $Q_t$\\
   \hline 0/1 & X & $Q_{t-1}$\\
   \hline \frontmontant&  0 & 0\\
   \hline \frontmontant&   1 & 1\\
   \hline
    \end{tabular}


\textbf{ Truth table of JK flipflop }\\
\begin{tabular}{|c|c|c||c|l|}
  \hline Ck & J & K &  $Q_t$ &\\
  \hline 0 & X & X & $Q_{t-1}$ &\\
  \hline \frontmontant&0 & 0 & $Q_{t-1}$ &\\
  \hline \frontmontant& 0 & 1 & 0&\\
  \hline  \frontmontant&1 & 0 & 1 &\\
  \hline  \frontmontant& 1 & 1 & $\overline{Q_{t-1}}$ & \textcolor{red}{Toggle}\\
  \hline
  \end{tabular}


\textbf{ Truth table of JK flipflop }\\
\begin{tabular}{|c|c|c||c|l|}
  \hline Ck & J & K &  $Q_t$ &\\
  \hline 0 & X & X & $Q_{t-1}$ &\\
  \hline \frontmontant&0 & 0 & $Q_{t-1}$ &\\
  \hline \frontmontant& 0 & 1 & 0&\\
  \hline  \frontmontant&1 & 0 & 1 &\\
  \hline  \frontmontant& 1 & 1 & $\overline{Q_{t-1}}$ & \textcolor{red}{Toggle}\\
  \hline
  \end{tabular}


\textbf{ Truth table of RS flipflop }\\
\begin{tabular}{|c|c||c|l|}
	\hline R & S & $Q_t$ &\\
	\hline 0 & 0 & $Q_{t}$ &Memory State \aRL{ذاكرة}\\
	\hline 0 & 1 & 1 &Set to 1 \aRL{توحيد}\\
	\hline 1 & 0 & 0 & Reset to 0 \aRL{تصفير}\\
	\hline 1 & 1 & X &  {\color{red}Forbidden \aRL{ممنوعة}}\\
	\hline
 \end{tabular}


\textbf{ Truth table of Dlatch flipflop }\\
 \begin{tabular}{|c|c|c|}
  \hline V & D & $Q_t$\\
  \hline 0 & X & $Q_{t-1}$\\
  \hline 1 &0 & 0\\
  \hline 1 & 1 & 1\\
  \hline
   \end{tabular}


\textbf{ Truth table of jkasyn flipflop }\\
\begin{tabular}{|l|c|c|c|c|c||c|l|}
  \hline
  Mode & Pr & Cl & Ck & J & K & Q+ & Remark \hfill\aRL{ملاحظة} \\
  \hline
  Asynchronous & 0 & 0 & X & X & X & X & Forbidden\hfill\aRL{ممنوع} \\
  \aRL{نمط غير متزامن} & 0 & 1 & X & X & X & 1 & Set to 1 \hfill\aRL{توحيد} \\
  & 1 & 0 & X & X & X & 0 & Set to 0 \hfill\aRL{تصفير} \\
  \hline
  \hline
  Synchronous & 1 &1 & 0/1 & X & X & Q & Memory State \hfill\aRL{ذاكرة} \\
  \aRL{نمط متزامن} & 1 &1 & \frontmontant&  0 & 0 & Q & Memory State \hfill\aRL{ذاكرة} \\
  & 1 &1 & \frontmontant&  0 & 1 & 0 & Set to 0 \hfill \aRL{تصفير} \\
  & 1 &1 & \frontmontant&  1 & 0 & 1 & Set to 1\hfill \aRL{توحيد} \\
  & 1 &1 & \frontmontant&  1 & 1 & $\overline{Q}$ & Toggle \hfill\aRL{قلب} \\
  \hline
  \end{tabular}


\textbf{ Truth table of custom flipflop }\\
\begin{tabular}{|c|c|c||c|}
  \hline Ck & X & Y &  $Q_t$ \\
  \hline 0 & X & X & $Q_{t-1}$ \\
  \hline \frontmontant&0 & 0 & $1$ \\
  \hline \frontmontant& 0 & 1 & $Q$\\
  \hline  \frontmontant&1 & 0 & $X$ \\
  \hline  \frontmontant& 1 & 1 & $Q'$\\
  \hline
  \end{tabular}

\textbf{ Truth table of customX flipflop }\\
    \textbf{No Table to display, please check the flip type "customX", use customized type if needed }


%---------------------------------------------------

\begin{circuitikz}
\node[flipflop JKMod] (JK0) at (2,4) {};

\node[flipflop DMod] (D0) at (2,0) {};

\node[flipflop RSMod] (RS0) at (5,4) {};


\node[flipflop DMod] (D1) at (2,8) {};

\node[flipflop RSMod] (RS1) at (5,8) {};

\node[flipflop JKMod] (JK1) at (8,8) {};

\node[flipflop JKMod] (JK2) at (11,8) {};

\node[flipflop DMod] (D2) at (14,8) {};


\draw (0,6) node[circle, fill, inner sep=1pt] {};
\draw (0,6) node[left] (Ck)  {$H$};



       \draw (Ck) -| (D1.pin 3);


       \draw (Ck) -| (JK1.pin 2);


       \draw (D1.pin 4) -- (RS1.pin 3);
       \draw (D1.pin 6) -- (RS1.pin 1);


       \draw (RS1.pin 4) -- (JK1.pin 1);
       \draw (RS1.pin 6) -- (JK1.pin 3);


       \draw (JK1.pin 6) -- (JK2.pin 2);


       \draw (JK1.pin 6) -- (JK2.pin 2);


       \draw (JK2.pin 4) -- (D2.pin 3);

       \draw (JK2.pin 6) -- (D2.pin 3);

\end{circuitikz}



\paragraph{Q2}




\section{Register}


 \textbf{Register}

\begin{circuitikz}

    \draw (0,0) node[circle, fill, inner sep=1pt] {};
\draw (0,0) node[left] (Ck)  {$Ck$};

        \node[flipflop DMod] (flip0) at (2,2) {};


        
       \draw (Ck) -| (flip0.pin 3);


        \node[flipflop JKMod] (flip1) at (5,2) {};


        
       \draw (Ck) -| (flip1.pin 2);


                   \draw (flip0.pin 4) -- (flip1.pin 3);
       \draw (flip0.pin 6) -- (flip1.pin 1);


        \node[flipflop RSMod] (flip2) at (8,2) {};


        
       \draw (Ck) -| (flip2.pin 2);


                   \draw (flip1.pin 4) -- (flip2.pin 3);
       \draw (flip1.pin 6) -- (flip2.pin 1);





\end{circuitikz}

 \textbf{Register  n bits}

\begin{circuitikz}

    \draw (0,0) node[circle, fill, inner sep=1pt] {};
\draw (0,0) node[left] (Ck)  {$Ck$};

        \node[flipflop DMod] (flip0) at (2,2) {};


        
       \draw (Ck) -| (flip0.pin 3);


        \node[flipflop DMod] (flip1) at (5,2) {};


        
       \draw (Ck) -| (flip1.pin 3);


                   \draw (flip0.pin 6) -- (flip1.pin 1);

        \node[flipflop DMod] (flip2) at (8,2) {};


        
       \draw (Ck) -| (flip2.pin 3);


                   \draw (flip1.pin 6) -- (flip2.pin 1);

        \node[flipflop DMod] (flip3) at (11,2) {};


        
       \draw (Ck) -| (flip3.pin 3);


                   \draw (flip2.pin 6) -- (flip3.pin 1);

        \node[flipflop DMod] (flip4) at (14,2) {};


        
       \draw (Ck) -| (flip4.pin 3);


                   \draw (flip3.pin 6) -- (flip4.pin 1);

        \node[flipflop DMod] (flip5) at (17,2) {};


        
       \draw (Ck) -| (flip5.pin 3);


                   \draw (flip4.pin 6) -- (flip5.pin 1);






\end{circuitikz}

 \textbf{Register  n bits Jk}

\begin{circuitikz}

    \draw (0,0) node[circle, fill, inner sep=1pt] {};
\draw (0,0) node[left] (Ck)  {$Ck$};

        \node[flipflop JKMod] (flip0) at (2,2) {};


        
       \draw (Ck) -| (flip0.pin 2);


        \node[flipflop JKMod] (flip1) at (5,2) {};


        
       \draw (Ck) -| (flip1.pin 2);


                   \draw (flip0.pin 4) -- (flip1.pin 3);
       \draw (flip0.pin 6) -- (flip1.pin 1);


        \node[flipflop JKMod] (flip2) at (8,2) {};


        
       \draw (Ck) -| (flip2.pin 2);


                   \draw (flip1.pin 4) -- (flip2.pin 3);
       \draw (flip1.pin 6) -- (flip2.pin 1);


        \node[flipflop JKMod] (flip3) at (11,2) {};


        
       \draw (Ck) -| (flip3.pin 2);


                   \draw (flip2.pin 4) -- (flip3.pin 3);
       \draw (flip2.pin 6) -- (flip3.pin 1);


        \node[flipflop JKMod] (flip4) at (14,2) {};


        
       \draw (Ck) -| (flip4.pin 2);


                   \draw (flip3.pin 4) -- (flip4.pin 3);
       \draw (flip3.pin 6) -- (flip4.pin 1);


        \node[flipflop JKMod] (flip5) at (17,2) {};


        
       \draw (Ck) -| (flip5.pin 2);


                   \draw (flip4.pin 4) -- (flip5.pin 3);
       \draw (flip4.pin 6) -- (flip5.pin 1);







\end{circuitikz}

\paragraph{Q3}


\section{Counter}


 \textbf{Counter  n bits Jk}



\begin{circuitikz}


    \draw (0,2) node[circle, fill, inner sep=1pt] {};
\draw (0,2) node[left] (Ck)  {$Ck$};

        \node[flipflop JKAMod] (flipc0) at (2,2) {};


        
       \draw (Ck) -| (flipc0.pin 2);


        \draw (0.5,4) node[circle, fill, inner sep=1pt] {};
\draw (0.5,4) node[above] (Vcc0)  {$Vcc$};

          \draw (Vcc0) |- (flipc0.up);
  \draw (Vcc0) |- (flipc0.down);
  \draw (Vcc0) |- (flipc0.pin 1);
  \draw (Vcc0) |- (flipc0.pin 3);

        \node[flipflop JKAMod] (flipc1) at (5,2) {};



        \draw (3.5,4) node[circle, fill, inner sep=1pt] {};
\draw (3.5,4) node[above] (Vcc1)  {$Vcc$};

          \draw (Vcc1) |- (flipc1.up);
  \draw (Vcc1) |- (flipc1.down);
  \draw (Vcc1) |- (flipc1.pin 1);
  \draw (Vcc1) |- (flipc1.pin 3);

                   \draw (flipc0.pin 6) -- (flipc1.pin 2);


        \node[flipflop JKAMod] (flipc2) at (8,2) {};



        \draw (6.5,4) node[circle, fill, inner sep=1pt] {};
\draw (6.5,4) node[above] (Vcc2)  {$Vcc$};

          \draw (Vcc2) |- (flipc2.up);
  \draw (Vcc2) |- (flipc2.down);
  \draw (Vcc2) |- (flipc2.pin 1);
  \draw (Vcc2) |- (flipc2.pin 3);

                   \draw (flipc1.pin 6) -- (flipc2.pin 2);


        \node[flipflop JKAMod] (flipc3) at (11,2) {};



        \draw (9.5,4) node[circle, fill, inner sep=1pt] {};
\draw (9.5,4) node[above] (Vcc3)  {$Vcc$};

          \draw (Vcc3) |- (flipc3.up);
  \draw (Vcc3) |- (flipc3.down);
  \draw (Vcc3) |- (flipc3.pin 1);
  \draw (Vcc3) |- (flipc3.pin 3);

                   \draw (flipc2.pin 6) -- (flipc3.pin 2);







\end{circuitikz}

 \textbf{flipfop syn/asyn Jk}

\begin{circuitikz}
\node[flipflop JKAMod] (jk4) at (5,4) {};

\end{circuitikz}

\paragraph{Q4}

\section{Misc}


\hrule width 1\linewidth
\pagebreak

\subsection{Correction}


\paragraph{Q1}

    \tikzset{flipflop DMod/.style={flipflop, dot on notQ,
        flipflop def={t1=D, t6=$Q2$,t4={\ctikztextnot{$Q2$}},
           c3=1},
    },
    }

\tikzset{flipflop JKMod/.style={flipflop,  dot on notQ,
            flipflop def={t1=J, t3=K,t6=$Q0$, t4={\ctikztextnot{$Q0$}},
               c2=1},
        },
}

\tikzset{flipflop RSMod/.style={flipflop,  dot on notQ,
                    flipflop def={t1=R,t3=S,t6=$Q1$, t4={\ctikztextnot{$Q1$}},
                       n4=1, c2=0},
                },
}

\tikzset{flipflop JKAMod/.style={flipflop,  dot on notQ,
            flipflop def={t1=J, t3=K,t6=Q, t4={\ctikztextnot{Q}},
                tu={\ctikztextnot{pr}}, nu=1, td={\ctikztextnot{cl}}, nd=1, c2=1},
        }}



\section{Flip}
\textbf{ Truth table of D flipflop }\\
  \begin{tabular}{|c|c|c|}
   \hline Ck & D & $Q_t$\\
   \hline 0/1 & X & $Q_{t-1}$\\
   \hline \frontmontant&  0 & 0\\
   \hline \frontmontant&   1 & 1\\
   \hline
    \end{tabular}


\textbf{ Truth table of JK flipflop }\\
\begin{tabular}{|c|c|c||c|l|}
  \hline Ck & J & K &  $Q_t$ &\\
  \hline 0 & X & X & $Q_{t-1}$ &\\
  \hline \frontmontant&0 & 0 & $Q_{t-1}$ &\\
  \hline \frontmontant& 0 & 1 & 0&\\
  \hline  \frontmontant&1 & 0 & 1 &\\
  \hline  \frontmontant& 1 & 1 & $\overline{Q_{t-1}}$ & \textcolor{red}{Toggle}\\
  \hline
  \end{tabular}


\textbf{ Truth table of JK flipflop }\\
\begin{tabular}{|c|c|c||c|l|}
  \hline Ck & J & K &  $Q_t$ &\\
  \hline 0 & X & X & $Q_{t-1}$ &\\
  \hline \frontmontant&0 & 0 & $Q_{t-1}$ &\\
  \hline \frontmontant& 0 & 1 & 0&\\
  \hline  \frontmontant&1 & 0 & 1 &\\
  \hline  \frontmontant& 1 & 1 & $\overline{Q_{t-1}}$ & \textcolor{red}{Toggle}\\
  \hline
  \end{tabular}


\textbf{ Truth table of RS flipflop }\\
\begin{tabular}{|c|c||c|l|}
	\hline R & S & $Q_t$ &\\
	\hline 0 & 0 & $Q_{t}$ &Memory State \aRL{ذاكرة}\\
	\hline 0 & 1 & 1 &Set to 1 \aRL{توحيد}\\
	\hline 1 & 0 & 0 & Reset to 0 \aRL{تصفير}\\
	\hline 1 & 1 & X &  {\color{red}Forbidden \aRL{ممنوعة}}\\
	\hline
 \end{tabular}


\textbf{ Truth table of Dlatch flipflop }\\
 \begin{tabular}{|c|c|c|}
  \hline V & D & $Q_t$\\
  \hline 0 & X & $Q_{t-1}$\\
  \hline 1 &0 & 0\\
  \hline 1 & 1 & 1\\
  \hline
   \end{tabular}


\textbf{ Truth table of jkasyn flipflop }\\
\begin{tabular}{|l|c|c|c|c|c||c|l|}
  \hline
  Mode & Pr & Cl & Ck & J & K & Q+ & Remark \hfill\aRL{ملاحظة} \\
  \hline
  Asynchronous & 0 & 0 & X & X & X & X & Forbidden\hfill\aRL{ممنوع} \\
  \aRL{نمط غير متزامن} & 0 & 1 & X & X & X & 1 & Set to 1 \hfill\aRL{توحيد} \\
  & 1 & 0 & X & X & X & 0 & Set to 0 \hfill\aRL{تصفير} \\
  \hline
  \hline
  Synchronous & 1 &1 & 0/1 & X & X & Q & Memory State \hfill\aRL{ذاكرة} \\
  \aRL{نمط متزامن} & 1 &1 & \frontmontant&  0 & 0 & Q & Memory State \hfill\aRL{ذاكرة} \\
  & 1 &1 & \frontmontant&  0 & 1 & 0 & Set to 0 \hfill \aRL{تصفير} \\
  & 1 &1 & \frontmontant&  1 & 0 & 1 & Set to 1\hfill \aRL{توحيد} \\
  & 1 &1 & \frontmontant&  1 & 1 & $\overline{Q}$ & Toggle \hfill\aRL{قلب} \\
  \hline
  \end{tabular}


\textbf{ Truth table of custom flipflop }\\
\begin{tabular}{|c|c|c||c|}
  \hline Ck & X & Y &  $Q_t$ \\
  \hline 0 & X & X & $Q_{t-1}$ \\
  \hline \frontmontant&0 & 0 & $1$ \\
  \hline \frontmontant& 0 & 1 & $Q$\\
  \hline  \frontmontant&1 & 0 & $X$ \\
  \hline  \frontmontant& 1 & 1 & $Q'$\\
  \hline
  \end{tabular}

\textbf{ Truth table of customX flipflop }\\
    \textbf{No Table to display, please check the flip type "customX", use customized type if needed }


%---------------------------------------------------

\begin{circuitikz}
\node[flipflop JKMod] (JK0) at (2,4) {};

\node[flipflop DMod] (D0) at (2,0) {};

\node[flipflop RSMod] (RS0) at (5,4) {};


\node[flipflop DMod] (D1) at (2,8) {};

\node[flipflop RSMod] (RS1) at (5,8) {};

\node[flipflop JKMod] (JK1) at (8,8) {};

\node[flipflop JKMod] (JK2) at (11,8) {};

\node[flipflop DMod] (D2) at (14,8) {};


\draw (0,6) node[circle, fill, inner sep=1pt] {};
\draw (0,6) node[left] (Ck)  {$H$};



       \draw (Ck) -| (D1.pin 3);


       \draw (Ck) -| (JK1.pin 2);


       \draw (D1.pin 4) -- (RS1.pin 3);
       \draw (D1.pin 6) -- (RS1.pin 1);


       \draw (RS1.pin 4) -- (JK1.pin 1);
       \draw (RS1.pin 6) -- (JK1.pin 3);


       \draw (JK1.pin 6) -- (JK2.pin 2);


       \draw (JK1.pin 6) -- (JK2.pin 2);


       \draw (JK2.pin 4) -- (D2.pin 3);

       \draw (JK2.pin 6) -- (D2.pin 3);

\end{circuitikz}



\paragraph{Q2}




\section{Register}


 \textbf{Register}

\begin{circuitikz}

    \draw (0,0) node[circle, fill, inner sep=1pt] {};
\draw (0,0) node[left] (Ck)  {$Ck$};

        \node[flipflop DMod] (flip0) at (2,2) {};


        
       \draw (Ck) -| (flip0.pin 3);


        \node[flipflop JKMod] (flip1) at (5,2) {};


        
       \draw (Ck) -| (flip1.pin 2);


                   \draw (flip0.pin 4) -- (flip1.pin 3);
       \draw (flip0.pin 6) -- (flip1.pin 1);


        \node[flipflop RSMod] (flip2) at (8,2) {};


        
       \draw (Ck) -| (flip2.pin 2);


                   \draw (flip1.pin 4) -- (flip2.pin 3);
       \draw (flip1.pin 6) -- (flip2.pin 1);





\end{circuitikz}

 \textbf{Register  n bits}

\begin{circuitikz}

    \draw (0,0) node[circle, fill, inner sep=1pt] {};
\draw (0,0) node[left] (Ck)  {$Ck$};

        \node[flipflop DMod] (flip0) at (2,2) {};


        
       \draw (Ck) -| (flip0.pin 3);


        \node[flipflop DMod] (flip1) at (5,2) {};


        
       \draw (Ck) -| (flip1.pin 3);


                   \draw (flip0.pin 6) -- (flip1.pin 1);

        \node[flipflop DMod] (flip2) at (8,2) {};


        
       \draw (Ck) -| (flip2.pin 3);


                   \draw (flip1.pin 6) -- (flip2.pin 1);

        \node[flipflop DMod] (flip3) at (11,2) {};


        
       \draw (Ck) -| (flip3.pin 3);


                   \draw (flip2.pin 6) -- (flip3.pin 1);

        \node[flipflop DMod] (flip4) at (14,2) {};


        
       \draw (Ck) -| (flip4.pin 3);


                   \draw (flip3.pin 6) -- (flip4.pin 1);

        \node[flipflop DMod] (flip5) at (17,2) {};


        
       \draw (Ck) -| (flip5.pin 3);


                   \draw (flip4.pin 6) -- (flip5.pin 1);






\end{circuitikz}

 \textbf{Register  n bits Jk}

\begin{circuitikz}

    \draw (0,0) node[circle, fill, inner sep=1pt] {};
\draw (0,0) node[left] (Ck)  {$Ck$};

        \node[flipflop JKMod] (flip0) at (2,2) {};


        
       \draw (Ck) -| (flip0.pin 2);


        \node[flipflop JKMod] (flip1) at (5,2) {};


        
       \draw (Ck) -| (flip1.pin 2);


                   \draw (flip0.pin 4) -- (flip1.pin 3);
       \draw (flip0.pin 6) -- (flip1.pin 1);


        \node[flipflop JKMod] (flip2) at (8,2) {};


        
       \draw (Ck) -| (flip2.pin 2);


                   \draw (flip1.pin 4) -- (flip2.pin 3);
       \draw (flip1.pin 6) -- (flip2.pin 1);


        \node[flipflop JKMod] (flip3) at (11,2) {};


        
       \draw (Ck) -| (flip3.pin 2);


                   \draw (flip2.pin 4) -- (flip3.pin 3);
       \draw (flip2.pin 6) -- (flip3.pin 1);


        \node[flipflop JKMod] (flip4) at (14,2) {};


        
       \draw (Ck) -| (flip4.pin 2);


                   \draw (flip3.pin 4) -- (flip4.pin 3);
       \draw (flip3.pin 6) -- (flip4.pin 1);


        \node[flipflop JKMod] (flip5) at (17,2) {};


        
       \draw (Ck) -| (flip5.pin 2);


                   \draw (flip4.pin 4) -- (flip5.pin 3);
       \draw (flip4.pin 6) -- (flip5.pin 1);







\end{circuitikz}

\paragraph{Q3}


\section{Counter}


 \textbf{Counter  n bits Jk}



\begin{circuitikz}


    \draw (0,2) node[circle, fill, inner sep=1pt] {};
\draw (0,2) node[left] (Ck)  {$Ck$};

        \node[flipflop JKAMod] (flipc0) at (2,2) {};


        
       \draw (Ck) -| (flipc0.pin 2);


        \draw (0.5,4) node[circle, fill, inner sep=1pt] {};
\draw (0.5,4) node[above] (Vcc0)  {$Vcc$};

          \draw (Vcc0) |- (flipc0.up);
  \draw (Vcc0) |- (flipc0.down);
  \draw (Vcc0) |- (flipc0.pin 1);
  \draw (Vcc0) |- (flipc0.pin 3);

        \node[flipflop JKAMod] (flipc1) at (5,2) {};



        \draw (3.5,4) node[circle, fill, inner sep=1pt] {};
\draw (3.5,4) node[above] (Vcc1)  {$Vcc$};

          \draw (Vcc1) |- (flipc1.up);
  \draw (Vcc1) |- (flipc1.down);
  \draw (Vcc1) |- (flipc1.pin 1);
  \draw (Vcc1) |- (flipc1.pin 3);

                   \draw (flipc0.pin 6) -- (flipc1.pin 2);


        \node[flipflop JKAMod] (flipc2) at (8,2) {};



        \draw (6.5,4) node[circle, fill, inner sep=1pt] {};
\draw (6.5,4) node[above] (Vcc2)  {$Vcc$};

          \draw (Vcc2) |- (flipc2.up);
  \draw (Vcc2) |- (flipc2.down);
  \draw (Vcc2) |- (flipc2.pin 1);
  \draw (Vcc2) |- (flipc2.pin 3);

                   \draw (flipc1.pin 6) -- (flipc2.pin 2);


        \node[flipflop JKAMod] (flipc3) at (11,2) {};



        \draw (9.5,4) node[circle, fill, inner sep=1pt] {};
\draw (9.5,4) node[above] (Vcc3)  {$Vcc$};

          \draw (Vcc3) |- (flipc3.up);
  \draw (Vcc3) |- (flipc3.down);
  \draw (Vcc3) |- (flipc3.pin 1);
  \draw (Vcc3) |- (flipc3.pin 3);

                   \draw (flipc2.pin 6) -- (flipc3.pin 2);







\end{circuitikz}

 \textbf{flipfop syn/asyn Jk}

\begin{circuitikz}
\node[flipflop JKAMod] (jk4) at (5,4) {};

\end{circuitikz}

\paragraph{Q4}

\section{Misc}
\pagebreak

\section{Question}


\paragraph{Q1}

Representer sous la norme IEEE-754 32 bits le nombre suivant
248.212

\hrule width 1\linewidth
\pagebreak

\subsection{Correction}


\paragraph{Q1}

Representer sous la norme IEEE-754 32 bits le nombre suivant : 248.212

\begin{verbatim}248.2120 = 11111000.001101100111101100110011001100
 Forme normalise 1.11110000011011001111011 x2\^7
Signe + => 0
 exposant 7 +127 = 134 =(10000110)
  pseudo mantisse  11110000011011001111011 
 La representation 
 01000011011110000011011001111011
 Hex 4378367B

\end{verbatim}
\pagebreak

\paragraph{Q1}

Representer sous la norme IEEE-754 32 bits le nombre suivant
26.817

\hrule width 1\linewidth
\pagebreak

\subsection{Correction}


\paragraph{Q1}

Representer sous la norme IEEE-754 32 bits le nombre suivant : 26.817

\begin{verbatim}26.8170 = 11010.110110111011001100110011001100
 Forme normalise 1.10101101101110110011001 x2\^4
Signe + => 0
 exposant 4 +127 = 131 =(10000011)
  pseudo mantisse  10101101101110110011001 
 La representation 
 01000001110101101101110110011001
 Hex 41D6DD99

\end{verbatim}
\pagebreak

\paragraph{Q1}

Representer sous la norme IEEE-754 32 bits le nombre suivant
94.413

\hrule width 1\linewidth
\pagebreak

\subsection{Correction}


\paragraph{Q1}

Representer sous la norme IEEE-754 32 bits le nombre suivant : 94.413

\begin{verbatim}94.4130 = 1011110.011010011011101001011110001101
 Forme normalise 1.01111001101001101110100 x2\^6
Signe + => 0
 exposant 6 +127 = 133 =(10000101)
  pseudo mantisse  01111001101001101110100 
 La representation 
 01000010101111001101001101110100
 Hex 42BCD374

\end{verbatim}
\pagebreak

\paragraph{Q1}

Representer sous la norme IEEE-754 32 bits le nombre suivant
-217.511

\hrule width 1\linewidth
\pagebreak

\subsection{Correction}


\paragraph{Q1}

Representer sous la norme IEEE-754 32 bits le nombre suivant : -217.511

\begin{verbatim}217.5110 = 11011001.100111011001100110011001100110
 Forme normalise 1.10110011001110110011001 x2\^7
Signe + => 0
 exposant 7 +127 = 134 =(10000110)
  pseudo mantisse  10110011001110110011001 
 La representation 
 11000011010110011001110110011001
 Hex C3599D99

\end{verbatim}
\pagebreak

\paragraph{Q1}

Representer sous la norme IEEE-754 32 bits le nombre suivant
-224.207

\hrule width 1\linewidth
\pagebreak

\subsection{Correction}


\paragraph{Q1}

Representer sous la norme IEEE-754 32 bits le nombre suivant : -224.207

\begin{verbatim}224.2070 = 11100000.001101001111110111110011101101
 Forme normalise 1.11000000011010011111101 x2\^7
Signe + => 0
 exposant 7 +127 = 134 =(10000110)
  pseudo mantisse  11000000011010011111101 
 La representation 
 11000011011000000011010011111101
 Hex C36034FD

\end{verbatim}
\pagebreak

\paragraph{Q1}

Donner les intervalles qu'on peut représenter en nombre positifs, valeur absolue, complément à 1 et complément à 2  sur 65 bits



\hrule width 1\linewidth
\pagebreak

\subsection{Correction}


\paragraph{Q1}

Les intervalles sur 65 bitsPositifs $[0; 2^{65}-1]$ [0; 36893488147419103231]

Valeur absolue [$-(2^{64}-1);2^{64}-1$] = [-18446744073709551615; 18446744073709551615]

Compelement à 1 [$-(2^{64}-1);2^{64}-1$] = [-18446744073709551615; 18446744073709551615]

Compelement à 2 [$-(2^{65}-1);2^{64}-1$] = [-18446744073709551616; 18446744073709551615]
\pagebreak

\paragraph{Q1}

Donner les intervalles qu'on peut représenter en nombre positifs, valeur absolue, complément à 1 et complément à 2  sur 10 bits



\hrule width 1\linewidth
\pagebreak

\subsection{Correction}


\paragraph{Q1}

Les intervalles sur 10 bitsPositifs $[0; 2^{10}-1]$ [0; 1023]

Valeur absolue [$-(2^{9}-1);2^{9}-1$] = [-511; 511]

Compelement à 1 [$-(2^{9}-1);2^{9}-1$] = [-511; 511]

Compelement à 2 [$-(2^{10}-1);2^{9}-1$] = [-512; 511]
\pagebreak

\paragraph{Q1}

Donner les intervalles qu'on peut représenter en nombre positifs, valeur absolue, complément à 1 et complément à 2  sur 94 bits



\hrule width 1\linewidth
\pagebreak

\subsection{Correction}


\paragraph{Q1}

Les intervalles sur 94 bitsPositifs $[0; 2^{94}-1]$ [0; 19807040628566084398385987583]

Valeur absolue [$-(2^{93}-1);2^{93}-1$] = [-9903520314283042199192993791; 9903520314283042199192993791]

Compelement à 1 [$-(2^{93}-1);2^{93}-1$] = [-9903520314283042199192993791; 9903520314283042199192993791]

Compelement à 2 [$-(2^{94}-1);2^{93}-1$] = [-9903520314283042199192993792; 9903520314283042199192993791]
\pagebreak

\paragraph{Q1}

Donner les intervalles qu'on peut représenter en nombre positifs, valeur absolue, complément à 1 et complément à 2  sur 67 bits



\hrule width 1\linewidth
\pagebreak

\subsection{Correction}


\paragraph{Q1}

Les intervalles sur 67 bitsPositifs $[0; 2^{67}-1]$ [0; 147573952589676412927]

Valeur absolue [$-(2^{66}-1);2^{66}-1$] = [-73786976294838206463; 73786976294838206463]

Compelement à 1 [$-(2^{66}-1);2^{66}-1$] = [-73786976294838206463; 73786976294838206463]

Compelement à 2 [$-(2^{67}-1);2^{66}-1$] = [-73786976294838206464; 73786976294838206463]
\pagebreak

\paragraph{Q1}

Donner les intervalles qu'on peut représenter en nombre positifs, valeur absolue, complément à 1 et complément à 2  sur 39 bits



\hrule width 1\linewidth
\pagebreak

\subsection{Correction}


\paragraph{Q1}

Les intervalles sur 39 bitsPositifs $[0; 2^{39}-1]$ [0; 549755813887]

Valeur absolue [$-(2^{38}-1);2^{38}-1$] = [-274877906943; 274877906943]

Compelement à 1 [$-(2^{38}-1);2^{38}-1$] = [-274877906943; 274877906943]

Compelement à 2 [$-(2^{39}-1);2^{38}-1$] = [-274877906944; 274877906943]
\pagebreak

\paragraph{Q1}

Representer en complément à 1 et à 2 le nombre  : -116


\hrule width 1\linewidth
\pagebreak

\subsection{Correction}


\paragraph{Q1}

Representer en complément à 1 et à 2 le nombre  : -116

$-116$

$ -1110100_{2}$

$ 10001011_{cp1}$

$ 10001100_{cp2}$
\pagebreak

\paragraph{Q1}

Representer en complément à 1 et à 2 le nombre  : -76


\hrule width 1\linewidth
\pagebreak

\subsection{Correction}


\paragraph{Q1}

Representer en complément à 1 et à 2 le nombre  : -76

$-76$

$ -1001100_{2}$

$ 10110011_{cp1}$

$ 10110100_{cp2}$
\pagebreak

\paragraph{Q1}

Representer en complément à 1 et à 2 le nombre  : -128


\hrule width 1\linewidth
\pagebreak

\subsection{Correction}


\paragraph{Q1}

Representer en complément à 1 et à 2 le nombre  : -128

$-128$

$-10000000_{2}$

$101111111_{cp1}$

$ 10000000_{cp2}$
\pagebreak

\paragraph{Q1}

Representer en complément à 1 et à 2 le nombre  : -61


\hrule width 1\linewidth
\pagebreak

\subsection{Correction}


\paragraph{Q1}

Representer en complément à 1 et à 2 le nombre  : -61

$-61$

$  -111101_{2}$

$ 11000010_{cp1}$

$ 11000011_{cp2}$
\pagebreak

\paragraph{Q1}

Representer en complément à 1 et à 2 le nombre  : -25


\hrule width 1\linewidth
\pagebreak

\subsection{Correction}


\paragraph{Q1}

Representer en complément à 1 et à 2 le nombre  : -25

$-25$

$   -11001_{2}$

$ 11100110_{cp1}$

$ 11100111_{cp2}$
\pagebreak

\paragraph{Q1}

Simplifier l'expression suivante par les proprietés algébreiques 

S = $ \bar a.c.d + \bar a.\bar b.c + a.b.c.\bar d + a.\bar b.\bar c.d  +  \bar a.d + a.c.\bar d + a.\bar b.\bar c + \bar a.\bar b.c  +  b.c.d + a.b.\bar c  +  a.b.c.d + a.b.\bar c.\bar d + \bar a.b.\bar c.d $


\hrule width 1\linewidth
\pagebreak

\subsection{Correction}


\paragraph{Q1}

Simplifier l'expression suivante

S = $ \bar a.c.d + \bar a.\bar b.c + a.b.c.\bar d + a.\bar b.\bar c.d  +  \bar a.d + a.c.\bar d + a.\bar b.\bar c + \bar a.\bar b.c  +  b.c.d + a.b.\bar c  +  a.b.c.d + a.b.\bar c.\bar d + \bar a.b.\bar c.d $

 = $ a.b + b.d + a.\bar c + a.\bar d + \bar a.d + \bar c.d + \bar a.\bar b.c + \bar b.c.\bar d $


Karnough map
\begin{karnaugh-map}[4][4][1][CD][AB]
          \minterms{1, 2, 3, 5, 7, 8, 9, 10, 12, 13, 14, 15}
          \maxterms{0, 4, 6, 11}
        %\autoterms[0]
         \terms{}{X}
        % simplification
        \implicant{12}{14}
\implicant{5}{15}
\implicant{12}{9}
\implicantedge{12}{8}{14}{10}
\implicant{1}{7}
\implicant{1}{9}
\implicant{3}{2}
\implicantedge{2}{2}{10}{10}
          %\implicant{5}{15}
          %\implicantedge{8}{8}{10}{10}
          %\implicantedge{8}{8}{10}{10}[8,10]
        \end{karnaugh-map}


\pagebreak

\paragraph{Q1}

Simplifier l'expression suivante par les proprietés algébreiques 

S = $ a.c.d + \bar b.c.d + a.\bar c.\bar d + \bar b.\bar c.\bar d + \bar a.b.\bar c.d  +  a.d + \bar a.\bar d  +  \bar a.\bar b.d + a.\bar b.\bar c.\bar d  +  a.c.d + \bar a.b.\bar d + b.\bar c.\bar d $


\hrule width 1\linewidth
\pagebreak

\subsection{Correction}


\paragraph{Q1}

Simplifier l'expression suivante

S = $ a.c.d + \bar b.c.d + a.\bar c.\bar d + \bar b.\bar c.\bar d + \bar a.b.\bar c.d  +  a.d + \bar a.\bar d  +  \bar a.\bar b.d + a.\bar b.\bar c.\bar d  +  a.c.d + \bar a.b.\bar d + b.\bar c.\bar d $

 = $ \bar c + a.d + \bar a.\bar b + \bar a.\bar d $


Karnough map
\begin{karnaugh-map}[4][4][1][CD][AB]
          \minterms{0, 1, 2, 3, 4, 5, 6, 8, 9, 11, 12, 13, 15}
          \maxterms{7, 10, 14}
        %\autoterms[0]
         \terms{}{X}
        % simplification
        \implicant{0}{9}
\implicant{13}{11}
\implicant{0}{2}
\implicantedge{0}{4}{2}{6}
          %\implicant{5}{15}
          %\implicantedge{8}{8}{10}{10}
          %\implicantedge{8}{8}{10}{10}[8,10]
        \end{karnaugh-map}


\pagebreak

\paragraph{Q1}

Simplifier l'expression suivante par les proprietés algébreiques 

S = $ b.c + \bar a.c + \bar a.\bar d + a.\bar b.d  +  a.d + b.c + a.\bar b + \bar b.\bar c  +  c.d + \bar a.d + \bar b.d + a.\bar b.\bar c + a.\bar c.\bar d + \bar a.b.\bar c + b.\bar c.\bar d + \bar a.\bar b.c  +  \bar a.b.c + a.\bar b.\bar c + a.\bar b.\bar d + a.\bar c.\bar d $


\hrule width 1\linewidth
\pagebreak

\subsection{Correction}


\paragraph{Q1}

Simplifier l'expression suivante

S = $ b.c + \bar a.c + \bar a.\bar d + a.\bar b.d  +  a.d + b.c + a.\bar b + \bar b.\bar c  +  c.d + \bar a.d + \bar b.d + a.\bar b.\bar c + a.\bar c.\bar d + \bar a.b.\bar c + b.\bar c.\bar d + \bar a.\bar b.c  +  \bar a.b.c + a.\bar b.\bar c + a.\bar b.\bar d + a.\bar c.\bar d $

 = $ True $


Karnough map
\begin{karnaugh-map}[4][4][1][CD][AB]
          \minterms{0, 1, 2, 3, 4, 5, 6, 7, 8, 9, 10, 11, 12, 13, 14, 15}
          \maxterms{}
        %\autoterms[0]
         \terms{}{X}
        % simplification
        
          %\implicant{5}{15}
          %\implicantedge{8}{8}{10}{10}
          %\implicantedge{8}{8}{10}{10}[8,10]
        \end{karnaugh-map}


\pagebreak

\paragraph{Q1}

Simplifier l'expression suivante par les proprietés algébreiques 

S = $ a.b.\bar c + \bar a.\bar b.c.d  +  a.b.\bar d + \bar a.\bar b.\bar c + \bar a.\bar b.\bar d + \bar a.\bar c.\bar d  +  \bar a.b.c.d + \bar a.b.\bar c.\bar d + \bar a.\bar b.c.\bar d  +  a.\bar b.c + b.\bar c.d + \bar b.c.\bar d $


\hrule width 1\linewidth
\pagebreak

\subsection{Correction}


\paragraph{Q1}

Simplifier l'expression suivante

S = $ a.b.\bar c + \bar a.\bar b.c.d  +  a.b.\bar d + \bar a.\bar b.\bar c + \bar a.\bar b.\bar d + \bar a.\bar c.\bar d  +  \bar a.b.c.d + \bar a.b.\bar c.\bar d + \bar a.\bar b.c.\bar d  +  a.\bar b.c + b.\bar c.d + \bar b.c.\bar d $

 = $ b.\bar c + \bar b.c + \bar a.d + \bar a.\bar b + a.c.\bar d $


Karnough map
\begin{karnaugh-map}[4][4][1][CD][AB]
          \minterms{0, 1, 2, 3, 4, 5, 7, 10, 11, 12, 13, 14}
          \maxterms{6, 8, 9, 15}
        %\autoterms[0]
         \terms{}{X}
        % simplification
        \implicant{4}{13}
\implicantedge{3}{2}{11}{10}
\implicant{1}{7}
\implicant{0}{2}
\implicant{14}{10}
          %\implicant{5}{15}
          %\implicantedge{8}{8}{10}{10}
          %\implicantedge{8}{8}{10}{10}[8,10]
        \end{karnaugh-map}


\pagebreak

\paragraph{Q1}

Simplifier l'expression suivante par les proprietés algébreiques 

S = $ a.b.c.d + a.\bar b.c.\bar d + \bar a.b.\bar c.d + \bar a.\bar b.\bar c.\bar d  +  a.\bar b.c + \bar a.b.\bar d + \bar a.\bar b.\bar c.d  +  a.\bar b.c + \bar a.b.c + \bar a.c.d + a.b.\bar c.d + \bar a.\bar c.\bar d  +  \bar a.b.d + \bar b.\bar c.\bar d $


\hrule width 1\linewidth
\pagebreak

\subsection{Correction}


\paragraph{Q1}

Simplifier l'expression suivante

S = $ a.b.c.d + a.\bar b.c.\bar d + \bar a.b.\bar c.d + \bar a.\bar b.\bar c.\bar d  +  a.\bar b.c + \bar a.b.\bar d + \bar a.\bar b.\bar c.d  +  a.\bar b.c + \bar a.b.c + \bar a.c.d + a.b.\bar c.d + \bar a.\bar c.\bar d  +  \bar a.b.d + \bar b.\bar c.\bar d $

 = $ b.d + c.d + \bar a.b + \bar a.d + \bar a.\bar c + a.\bar b.c + a.\bar b.\bar d + \bar b.\bar c.\bar d $


Karnough map
\begin{karnaugh-map}[4][4][1][CD][AB]
          \minterms{0, 1, 3, 4, 5, 6, 7, 8, 10, 11, 13, 15}
          \maxterms{2, 9, 12, 14}
        %\autoterms[0]
         \terms{}{X}
        % simplification
        \implicant{5}{15}
\implicant{3}{11}
\implicant{4}{6}
\implicant{1}{7}
\implicant{0}{5}
\implicant{10}{10}
\implicantedge{8}{8}{10}{10}
\implicantedge{0}{0}{8}{8}
          %\implicant{5}{15}
          %\implicantedge{8}{8}{10}{10}
          %\implicantedge{8}{8}{10}{10}[8,10]
        \end{karnaugh-map}


\pagebreak

\paragraph{Q1}

Simplifier les fonctions suivantes

\begin{karnaugh-map}[4][4][1][CD][AB]
          \minterms{3, 4, 7, 11, 15}
          \maxterms{0, 1, 2, 5, 6, 8, 9, 10, 12, 13, 14}
        %\autoterms[0]
         \terms{}{X}
        % simplification
        
          %\implicant{5}{15}
          %\implicantedge{8}{8}{10}{10}
          %\implicantedge{8}{8}{10}{10}[8,10]
        \end{karnaugh-map}\begin{karnaugh-map}[4][4][1][CD][AB]
          \minterms{0, 1, 4, 5, 8, 10, 12}
          \maxterms{2, 3, 6, 7, 9, 11, 13, 14, 15}
        %\autoterms[0]
         \terms{}{X}
        % simplification
        
          %\implicant{5}{15}
          %\implicantedge{8}{8}{10}{10}
          %\implicantedge{8}{8}{10}{10}[8,10]
        \end{karnaugh-map}\begin{karnaugh-map}[4][4][1][CD][AB]
          \minterms{0, 1, 4, 5, 7, 11, 12, 13, 15}
          \maxterms{2, 3, 6, 8, 9, 10, 14}
        %\autoterms[0]
         \terms{}{X}
        % simplification
        
          %\implicant{5}{15}
          %\implicantedge{8}{8}{10}{10}
          %\implicantedge{8}{8}{10}{10}[8,10]
        \end{karnaugh-map}

\hrule width 1\linewidth
\pagebreak

\subsection{Correction}


\paragraph{Q1}

Simplifier les fonctions suivantes
table 1

\begin{karnaugh-map}[4][4][1][CD][AB]
          \minterms{3, 4, 7, 11, 15}
          \maxterms{0, 1, 2, 5, 6, 8, 9, 10, 12, 13, 14}
        %\autoterms[0]
         \terms{}{X}
        % simplification
        \implicant{3}{11}
\implicant{4}{4}
          %\implicant{5}{15}
          %\implicantedge{8}{8}{10}{10}
          %\implicantedge{8}{8}{10}{10}[8,10]
        \end{karnaugh-map}Simplified Sum of products : $ c.d + \bar a.b.\bar c.\bar d $

table 2

\begin{karnaugh-map}[4][4][1][CD][AB]
          \minterms{0, 1, 4, 5, 8, 10, 12}
          \maxterms{2, 3, 6, 7, 9, 11, 13, 14, 15}
        %\autoterms[0]
         \terms{}{X}
        % simplification
        \implicant{0}{5}
\implicant{0}{8}
\implicantedge{8}{8}{10}{10}
          %\implicant{5}{15}
          %\implicantedge{8}{8}{10}{10}
          %\implicantedge{8}{8}{10}{10}[8,10]
        \end{karnaugh-map}Simplified Sum of products : $ \bar a.\bar c + \bar c.\bar d + a.\bar b.\bar d $

table 3

\begin{karnaugh-map}[4][4][1][CD][AB]
          \minterms{0, 1, 4, 5, 7, 11, 12, 13, 15}
          \maxterms{2, 3, 6, 8, 9, 10, 14}
        %\autoterms[0]
         \terms{}{X}
        % simplification
        \implicant{5}{15}
\implicant{4}{13}
\implicant{15}{11}
\implicant{0}{5}
          %\implicant{5}{15}
          %\implicantedge{8}{8}{10}{10}
          %\implicantedge{8}{8}{10}{10}[8,10]
        \end{karnaugh-map}Simplified Sum of products : $ b.d + b.\bar c + a.c.d + \bar a.\bar c $


\pagebreak

\paragraph{Q1}

Simplifier les fonctions suivantes

\begin{karnaugh-map}[4][4][1][CD][AB]
          \minterms{0, 1, 2, 3, 7, 8, 9, 10, 12, 13}
          \maxterms{4, 5, 6, 11, 14, 15}
        %\autoterms[0]
         \terms{}{X}
        % simplification
        
          %\implicant{5}{15}
          %\implicantedge{8}{8}{10}{10}
          %\implicantedge{8}{8}{10}{10}[8,10]
        \end{karnaugh-map}\begin{karnaugh-map}[4][4][1][CD][AB]
          \minterms{0, 4, 5, 7, 8, 9, 10}
          \maxterms{1, 2, 3, 6, 11, 12, 13, 14, 15}
        %\autoterms[0]
         \terms{}{X}
        % simplification
        
          %\implicant{5}{15}
          %\implicantedge{8}{8}{10}{10}
          %\implicantedge{8}{8}{10}{10}[8,10]
        \end{karnaugh-map}\begin{karnaugh-map}[4][4][1][CD][AB]
          \minterms{1, 4, 5, 10}
          \maxterms{0, 2, 3, 6, 7, 8, 9, 11, 12, 13, 14, 15}
        %\autoterms[0]
         \terms{}{X}
        % simplification
        
          %\implicant{5}{15}
          %\implicantedge{8}{8}{10}{10}
          %\implicantedge{8}{8}{10}{10}[8,10]
        \end{karnaugh-map}

\hrule width 1\linewidth
\pagebreak

\subsection{Correction}


\paragraph{Q1}

Simplifier les fonctions suivantes
table 1

\begin{karnaugh-map}[4][4][1][CD][AB]
          \minterms{0, 1, 2, 3, 7, 8, 9, 10, 12, 13}
          \maxterms{4, 5, 6, 11, 14, 15}
        %\autoterms[0]
         \terms{}{X}
        % simplification
        \implicant{12}{9}
\implicant{0}{2}
 \implicantcorner{0}{2}
\implicant{3}{7}
          %\implicant{5}{15}
          %\implicantedge{8}{8}{10}{10}
          %\implicantedge{8}{8}{10}{10}[8,10]
        \end{karnaugh-map}Simplified Sum of products : $ a.\bar c + \bar a.\bar b + \bar b.\bar d + \bar a.c.d $

table 2

\begin{karnaugh-map}[4][4][1][CD][AB]
          \minterms{0, 4, 5, 7, 8, 9, 10}
          \maxterms{1, 2, 3, 6, 11, 12, 13, 14, 15}
        %\autoterms[0]
         \terms{}{X}
        % simplification
        \implicant{5}{7}
\implicant{8}{9}
\implicantedge{8}{8}{10}{10}
\implicant{0}{4}
          %\implicant{5}{15}
          %\implicantedge{8}{8}{10}{10}
          %\implicantedge{8}{8}{10}{10}[8,10]
        \end{karnaugh-map}Simplified Sum of products : $ \bar a.b.d + a.\bar b.\bar c + a.\bar b.\bar d + \bar a.\bar c.\bar d $

table 3

\begin{karnaugh-map}[4][4][1][CD][AB]
          \minterms{1, 4, 5, 10}
          \maxterms{0, 2, 3, 6, 7, 8, 9, 11, 12, 13, 14, 15}
        %\autoterms[0]
         \terms{}{X}
        % simplification
        \implicant{4}{5}
\implicant{1}{5}
\implicant{10}{10}
          %\implicant{5}{15}
          %\implicantedge{8}{8}{10}{10}
          %\implicantedge{8}{8}{10}{10}[8,10]
        \end{karnaugh-map}Simplified Sum of products : $ \bar a.b.\bar c + \bar a.\bar c.d + a.\bar b.c.\bar d $


\pagebreak

\paragraph{Q1}

Simplifier les fonctions suivantes

\begin{karnaugh-map}[4][4][1][CD][AB]
          \minterms{0, 1, 7, 14}
          \maxterms{2, 3, 4, 5, 6, 8, 9, 10, 11, 12, 13, 15}
        %\autoterms[0]
         \terms{}{X}
        % simplification
        
          %\implicant{5}{15}
          %\implicantedge{8}{8}{10}{10}
          %\implicantedge{8}{8}{10}{10}[8,10]
        \end{karnaugh-map}\begin{karnaugh-map}[4][4][1][CD][AB]
          \minterms{1, 7, 8, 9, 10, 14}
          \maxterms{0, 2, 3, 4, 5, 6, 11, 12, 13, 15}
        %\autoterms[0]
         \terms{}{X}
        % simplification
        
          %\implicant{5}{15}
          %\implicantedge{8}{8}{10}{10}
          %\implicantedge{8}{8}{10}{10}[8,10]
        \end{karnaugh-map}\begin{karnaugh-map}[4][4][1][CD][AB]
          \minterms{1, 2, 3, 4, 5, 8, 10, 13, 14}
          \maxterms{0, 6, 7, 9, 11, 12, 15}
        %\autoterms[0]
         \terms{}{X}
        % simplification
        
          %\implicant{5}{15}
          %\implicantedge{8}{8}{10}{10}
          %\implicantedge{8}{8}{10}{10}[8,10]
        \end{karnaugh-map}

\hrule width 1\linewidth
\pagebreak

\subsection{Correction}


\paragraph{Q1}

Simplifier les fonctions suivantes
table 1

\begin{karnaugh-map}[4][4][1][CD][AB]
          \minterms{0, 1, 7, 14}
          \maxterms{2, 3, 4, 5, 6, 8, 9, 10, 11, 12, 13, 15}
        %\autoterms[0]
         \terms{}{X}
        % simplification
        \implicant{14}{14}
\implicant{7}{7}
\implicant{0}{1}
          %\implicant{5}{15}
          %\implicantedge{8}{8}{10}{10}
          %\implicantedge{8}{8}{10}{10}[8,10]
        \end{karnaugh-map}Simplified Sum of products : $ a.b.c.\bar d + \bar a.b.c.d + \bar a.\bar b.\bar c $

table 2

\begin{karnaugh-map}[4][4][1][CD][AB]
          \minterms{1, 7, 8, 9, 10, 14}
          \maxterms{0, 2, 3, 4, 5, 6, 11, 12, 13, 15}
        %\autoterms[0]
         \terms{}{X}
        % simplification
        \implicant{14}{10}
\implicant{8}{9}
\implicantedge{1}{1}{9}{9}
\implicant{7}{7}
          %\implicant{5}{15}
          %\implicantedge{8}{8}{10}{10}
          %\implicantedge{8}{8}{10}{10}[8,10]
        \end{karnaugh-map}Simplified Sum of products : $ a.c.\bar d + a.\bar b.\bar c + \bar b.\bar c.d + \bar a.b.c.d $

table 3

\begin{karnaugh-map}[4][4][1][CD][AB]
          \minterms{1, 2, 3, 4, 5, 8, 10, 13, 14}
          \maxterms{0, 6, 7, 9, 11, 12, 15}
        %\autoterms[0]
         \terms{}{X}
        % simplification
        \implicant{14}{10}
\implicant{5}{13}
\implicantedge{8}{8}{10}{10}
\implicant{4}{5}
\implicant{3}{2}
\implicant{1}{3}
          %\implicant{5}{15}
          %\implicantedge{8}{8}{10}{10}
          %\implicantedge{8}{8}{10}{10}[8,10]
        \end{karnaugh-map}Simplified Sum of products : $ a.c.\bar d + b.\bar c.d + a.\bar b.\bar d + \bar a.b.\bar c + \bar a.\bar b.c + \bar a.\bar b.d $


\pagebreak

\paragraph{Q1}

Simplifier les fonctions suivantes

\begin{karnaugh-map}[4][4][1][CD][AB]
          \minterms{3, 5, 6, 7, 8, 9, 10}
          \maxterms{0, 1, 2, 4, 11, 12, 13, 14, 15}
        %\autoterms[0]
         \terms{}{X}
        % simplification
        
          %\implicant{5}{15}
          %\implicantedge{8}{8}{10}{10}
          %\implicantedge{8}{8}{10}{10}[8,10]
        \end{karnaugh-map}\begin{karnaugh-map}[4][4][1][CD][AB]
          \minterms{0, 2, 7, 12, 14}
          \maxterms{1, 3, 4, 5, 6, 8, 9, 10, 11, 13, 15}
        %\autoterms[0]
         \terms{}{X}
        % simplification
        
          %\implicant{5}{15}
          %\implicantedge{8}{8}{10}{10}
          %\implicantedge{8}{8}{10}{10}[8,10]
        \end{karnaugh-map}\begin{karnaugh-map}[4][4][1][CD][AB]
          \minterms{0, 2, 7, 8, 9, 10, 12, 14, 15}
          \maxterms{1, 3, 4, 5, 6, 11, 13}
        %\autoterms[0]
         \terms{}{X}
        % simplification
        
          %\implicant{5}{15}
          %\implicantedge{8}{8}{10}{10}
          %\implicantedge{8}{8}{10}{10}[8,10]
        \end{karnaugh-map}

\hrule width 1\linewidth
\pagebreak

\subsection{Correction}


\paragraph{Q1}

Simplifier les fonctions suivantes
table 1

\begin{karnaugh-map}[4][4][1][CD][AB]
          \minterms{3, 5, 6, 7, 8, 9, 10}
          \maxterms{0, 1, 2, 4, 11, 12, 13, 14, 15}
        %\autoterms[0]
         \terms{}{X}
        % simplification
        \implicant{7}{6}
\implicant{5}{7}
\implicant{3}{7}
\implicant{8}{9}
\implicantedge{8}{8}{10}{10}
          %\implicant{5}{15}
          %\implicantedge{8}{8}{10}{10}
          %\implicantedge{8}{8}{10}{10}[8,10]
        \end{karnaugh-map}Simplified Sum of products : $ \bar a.b.c + \bar a.b.d + \bar a.c.d + a.\bar b.\bar c + a.\bar b.\bar d $

table 2

\begin{karnaugh-map}[4][4][1][CD][AB]
          \minterms{0, 2, 7, 12, 14}
          \maxterms{1, 3, 4, 5, 6, 8, 9, 10, 11, 13, 15}
        %\autoterms[0]
         \terms{}{X}
        % simplification
        \implicantedge{12}{12}{14}{14}
\implicant{7}{7}
\implicantedge{0}{0}{2}{2}
          %\implicant{5}{15}
          %\implicantedge{8}{8}{10}{10}
          %\implicantedge{8}{8}{10}{10}[8,10]
        \end{karnaugh-map}Simplified Sum of products : $ a.b.\bar d + \bar a.b.c.d + \bar a.\bar b.\bar d $

table 3

\begin{karnaugh-map}[4][4][1][CD][AB]
          \minterms{0, 2, 7, 8, 9, 10, 12, 14, 15}
          \maxterms{1, 3, 4, 5, 6, 11, 13}
        %\autoterms[0]
         \terms{}{X}
        % simplification
        \implicantedge{12}{8}{14}{10}
\implicant{7}{15}
 \implicantcorner{0}{2}
\implicant{8}{9}
          %\implicant{5}{15}
          %\implicantedge{8}{8}{10}{10}
          %\implicantedge{8}{8}{10}{10}[8,10]
        \end{karnaugh-map}Simplified Sum of products : $ a.\bar d + b.c.d + \bar b.\bar d + a.\bar b.\bar c $


\pagebreak

\paragraph{Q1}

Simplifier les fonctions suivantes

\begin{karnaugh-map}[4][4][1][CD][AB]
          \minterms{0, 5, 8, 9, 11, 15}
          \maxterms{1, 2, 3, 4, 6, 7, 10, 12, 13, 14}
        %\autoterms[0]
         \terms{}{X}
        % simplification
        
          %\implicant{5}{15}
          %\implicantedge{8}{8}{10}{10}
          %\implicantedge{8}{8}{10}{10}[8,10]
        \end{karnaugh-map}\begin{karnaugh-map}[4][4][1][CD][AB]
          \minterms{2, 8, 10, 12, 13}
          \maxterms{0, 1, 3, 4, 5, 6, 7, 9, 11, 14, 15}
        %\autoterms[0]
         \terms{}{X}
        % simplification
        
          %\implicant{5}{15}
          %\implicantedge{8}{8}{10}{10}
          %\implicantedge{8}{8}{10}{10}[8,10]
        \end{karnaugh-map}\begin{karnaugh-map}[4][4][1][CD][AB]
          \minterms{1, 4, 7, 10, 12, 14}
          \maxterms{0, 2, 3, 5, 6, 8, 9, 11, 13, 15}
        %\autoterms[0]
         \terms{}{X}
        % simplification
        
          %\implicant{5}{15}
          %\implicantedge{8}{8}{10}{10}
          %\implicantedge{8}{8}{10}{10}[8,10]
        \end{karnaugh-map}

\hrule width 1\linewidth
\pagebreak

\subsection{Correction}


\paragraph{Q1}

Simplifier les fonctions suivantes
table 1

\begin{karnaugh-map}[4][4][1][CD][AB]
          \minterms{0, 5, 8, 9, 11, 15}
          \maxterms{1, 2, 3, 4, 6, 7, 10, 12, 13, 14}
        %\autoterms[0]
         \terms{}{X}
        % simplification
        \implicant{15}{11}
\implicant{8}{9}
\implicantedge{0}{0}{8}{8}
\implicant{5}{5}
          %\implicant{5}{15}
          %\implicantedge{8}{8}{10}{10}
          %\implicantedge{8}{8}{10}{10}[8,10]
        \end{karnaugh-map}Simplified Sum of products : $ a.c.d + a.\bar b.\bar c + \bar b.\bar c.\bar d + \bar a.b.\bar c.d $

table 2

\begin{karnaugh-map}[4][4][1][CD][AB]
          \minterms{2, 8, 10, 12, 13}
          \maxterms{0, 1, 3, 4, 5, 6, 7, 9, 11, 14, 15}
        %\autoterms[0]
         \terms{}{X}
        % simplification
        \implicant{12}{13}
\implicantedge{8}{8}{10}{10}
\implicantedge{2}{2}{10}{10}
          %\implicant{5}{15}
          %\implicantedge{8}{8}{10}{10}
          %\implicantedge{8}{8}{10}{10}[8,10]
        \end{karnaugh-map}Simplified Sum of products : $ a.b.\bar c + a.\bar b.\bar d + \bar b.c.\bar d $

table 3

\begin{karnaugh-map}[4][4][1][CD][AB]
          \minterms{1, 4, 7, 10, 12, 14}
          \maxterms{0, 2, 3, 5, 6, 8, 9, 11, 13, 15}
        %\autoterms[0]
         \terms{}{X}
        % simplification
        \implicant{14}{10}
\implicant{4}{12}
\implicant{7}{7}
\implicant{1}{1}
          %\implicant{5}{15}
          %\implicantedge{8}{8}{10}{10}
          %\implicantedge{8}{8}{10}{10}[8,10]
        \end{karnaugh-map}Simplified Sum of products : $ a.c.\bar d + b.\bar c.\bar d + \bar a.b.c.d + \bar a.\bar b.\bar c.d $


\pagebreak

\paragraph{Q1}

Soit la fonction donnée par sa forme canonique, Tracer la table de karnaugh et simplifier.

$$F0(a, b, c, d) = $\bar A.\bar B.C.D + \bar A.B.\bar C.\bar D + \bar A.B.\bar C.D + A.\bar B.\bar C.\bar D + A.\bar B.C.D + A.B.\bar C.\bar D$

$$$$F0(a, b, c, d) = $\sum([3, 4, 5, 8, 11, 12])$

$$$$F1(a, b, c, d) = $\bar A.\bar B.C.D + \bar A.B.C.\bar D + A.B.\bar C.\bar D + A.B.C.D$

$$$$F1(a, b, c, d) = $\sum([3, 6, 12, 15])$

$$

\hrule width 1\linewidth
\pagebreak

\subsection{Correction}


\paragraph{Q1}

Simplifier les fonctions suivantes

$$F0(a, b, c, d) = $\bar A.\bar B.C.D + \bar A.B.\bar C.\bar D + \bar A.B.\bar C.D + A.\bar B.\bar C.\bar D + A.\bar B.C.D + A.B.\bar C.\bar D$

$$$$F0(a, b, c, d) = $\sum([3, 4, 5, 8, 11, 12])$

$$\begin{karnaugh-map}[4][4][1][CD][AB]
          \minterms{3, 4, 5, 8, 11, 12}
          \maxterms{0, 1, 2, 6, 7, 9, 10, 13, 14, 15}
        %\autoterms[0]
         \terms{}{X}
        % simplification
        \implicantedge{3}{3}{11}{11}
\implicant{12}{8}
\implicant{4}{5}
          %\implicant{5}{15}
          %\implicantedge{8}{8}{10}{10}
          %\implicantedge{8}{8}{10}{10}[8,10]
        \end{karnaugh-map}$$Simplified Sum of products : $ b'.c.d + a.c'.d' + a'.b.c' $

$$$$F1(a, b, c, d) = $\bar A.\bar B.C.D + \bar A.B.C.\bar D + A.B.\bar C.\bar D + A.B.C.D$

$$$$F1(a, b, c, d) = $\sum([3, 6, 12, 15])$

$$\begin{karnaugh-map}[4][4][1][CD][AB]
          \minterms{3, 6, 12, 15}
          \maxterms{0, 1, 2, 4, 5, 7, 8, 9, 10, 11, 13, 14}
        %\autoterms[0]
         \terms{}{X}
        % simplification
        \implicant{15}{15}
\implicant{12}{12}
\implicant{6}{6}
\implicant{3}{3}
          %\implicant{5}{15}
          %\implicantedge{8}{8}{10}{10}
          %\implicantedge{8}{8}{10}{10}[8,10]
        \end{karnaugh-map}$$Simplified Sum of products : $ a.b.c.d + a.b.c'.d' + a'.b.c.d' + a'.b'.c.d $

$$
\pagebreak

\paragraph{Q1}

Soit la fonction donnée par sa forme canonique, Tracer la table de karnaugh et simplifier.

$$F0(a, b, c, d) = $\bar A.\bar B.\bar C.\bar D + \bar A.\bar B.C.\bar D + \bar A.B.C.\bar D + \bar A.B.C.D + A.\bar B.C.D + A.B.\bar C.\bar D + A.B.\bar C.D$

$$$$F0(a, b, c, d) = $\sum([0, 2, 6, 7, 11, 12, 13])$

$$$$F1(a, b, c, d) = $\bar A.\bar B.\bar C.\bar D + \bar A.\bar B.\bar C.D + \bar A.\bar B.C.D + \bar A.B.\bar C.\bar D + \bar A.B.C.\bar D + \bar A.B.C.D + A.\bar B.\bar C.D + A.\bar B.C.D + A.B.\bar C.\bar D + A.B.\bar C.D + A.B.C.\bar D$

$$$$F1(a, b, c, d) = $\sum([0, 1, 3, 4, 6, 7, 9, 11, 12, 13, 14])$

$$

\hrule width 1\linewidth
\pagebreak

\subsection{Correction}


\paragraph{Q1}

Simplifier les fonctions suivantes

$$F0(a, b, c, d) = $\bar A.\bar B.\bar C.\bar D + \bar A.\bar B.C.\bar D + \bar A.B.C.\bar D + \bar A.B.C.D + A.\bar B.C.D + A.B.\bar C.\bar D + A.B.\bar C.D$

$$$$F0(a, b, c, d) = $\sum([0, 2, 6, 7, 11, 12, 13])$

$$\begin{karnaugh-map}[4][4][1][CD][AB]
          \minterms{0, 2, 6, 7, 11, 12, 13}
          \maxterms{1, 3, 4, 5, 8, 9, 10, 14, 15}
        %\autoterms[0]
         \terms{}{X}
        % simplification
        \implicant{12}{13}
\implicant{7}{6}
\implicant{11}{11}
\implicantedge{0}{0}{2}{2}
          %\implicant{5}{15}
          %\implicantedge{8}{8}{10}{10}
          %\implicantedge{8}{8}{10}{10}[8,10]
        \end{karnaugh-map}$$Simplified Sum of products : $ a.b.c' + a'.b.c + a.b'.c.d + a'.b'.d' $

$$$$F1(a, b, c, d) = $\bar A.\bar B.\bar C.\bar D + \bar A.\bar B.\bar C.D + \bar A.\bar B.C.D + \bar A.B.\bar C.\bar D + \bar A.B.C.\bar D + \bar A.B.C.D + A.\bar B.\bar C.D + A.\bar B.C.D + A.B.\bar C.\bar D + A.B.\bar C.D + A.B.C.\bar D$

$$$$F1(a, b, c, d) = $\sum([0, 1, 3, 4, 6, 7, 9, 11, 12, 13, 14])$

$$\begin{karnaugh-map}[4][4][1][CD][AB]
          \minterms{0, 1, 3, 4, 6, 7, 9, 11, 12, 13, 14}
          \maxterms{2, 5, 8, 10, 15}
        %\autoterms[0]
         \terms{}{X}
        % simplification
        \implicantedge{4}{12}{6}{14}
\implicantedge{1}{3}{9}{11}
\implicant{13}{9}
\implicant{3}{7}
\implicant{0}{1}
          %\implicant{5}{15}
          %\implicantedge{8}{8}{10}{10}
          %\implicantedge{8}{8}{10}{10}[8,10]
        \end{karnaugh-map}$$Simplified Sum of products : $ b.d' + b'.d + a.c'.d + a'.c.d + a'.b'.c' $

$$
\pagebreak

\paragraph{Q1}

Soit la fonction donnée par sa forme canonique, Tracer la table de karnaugh et simplifier.

$$F0(a, b, c, d) = $\bar A.\bar B.\bar C.\bar D + \bar A.\bar B.C.\bar D + \bar A.B.\bar C.\bar D + A.\bar B.\bar C.D + A.\bar B.C.D + A.B.C.D$

$$$$F0(a, b, c, d) = $\sum([0, 2, 4, 9, 11, 15])$

$$$$F1(a, b, c, d) = $\bar A.\bar B.\bar C.\bar D + \bar A.\bar B.C.\bar D + \bar A.\bar B.C.D + \bar A.B.\bar C.\bar D + \bar A.B.C.\bar D + \bar A.B.C.D + A.\bar B.\bar C.\bar D + A.\bar B.\bar C.D + A.\bar B.C.D + A.B.C.D$

$$$$F1(a, b, c, d) = $\sum([0, 2, 3, 4, 6, 7, 8, 9, 11, 15])$

$$

\hrule width 1\linewidth
\pagebreak

\subsection{Correction}


\paragraph{Q1}

Simplifier les fonctions suivantes

$$F0(a, b, c, d) = $\bar A.\bar B.\bar C.\bar D + \bar A.\bar B.C.\bar D + \bar A.B.\bar C.\bar D + A.\bar B.\bar C.D + A.\bar B.C.D + A.B.C.D$

$$$$F0(a, b, c, d) = $\sum([0, 2, 4, 9, 11, 15])$

$$\begin{karnaugh-map}[4][4][1][CD][AB]
          \minterms{0, 2, 4, 9, 11, 15}
          \maxterms{1, 3, 5, 6, 7, 8, 10, 12, 13, 14}
        %\autoterms[0]
         \terms{}{X}
        % simplification
        \implicant{15}{11}
\implicant{9}{11}
\implicantedge{0}{0}{2}{2}
\implicant{0}{4}
          %\implicant{5}{15}
          %\implicantedge{8}{8}{10}{10}
          %\implicantedge{8}{8}{10}{10}[8,10]
        \end{karnaugh-map}$$Simplified Sum of products : $ a.c.d + a.b'.d + a'.b'.d' + a'.c'.d' $

$$$$F1(a, b, c, d) = $\bar A.\bar B.\bar C.\bar D + \bar A.\bar B.C.\bar D + \bar A.\bar B.C.D + \bar A.B.\bar C.\bar D + \bar A.B.C.\bar D + \bar A.B.C.D + A.\bar B.\bar C.\bar D + A.\bar B.\bar C.D + A.\bar B.C.D + A.B.C.D$

$$$$F1(a, b, c, d) = $\sum([0, 2, 3, 4, 6, 7, 8, 9, 11, 15])$

$$\begin{karnaugh-map}[4][4][1][CD][AB]
          \minterms{0, 2, 3, 4, 6, 7, 8, 9, 11, 15}
          \maxterms{1, 5, 10, 12, 13, 14}
        %\autoterms[0]
         \terms{}{X}
        % simplification
        \implicant{3}{11}
\implicantedge{0}{4}{2}{6}
\implicant{8}{9}
          %\implicant{5}{15}
          %\implicantedge{8}{8}{10}{10}
          %\implicantedge{8}{8}{10}{10}[8,10]
        \end{karnaugh-map}$$Simplified Sum of products : $ c.d + a'.d' + a.b'.c' $

$$
\pagebreak

\paragraph{Q1}

Soit la fonction donnée par sa forme canonique, Tracer la table de karnaugh et simplifier.

$$F0(a, b, c, d) = $\bar A.\bar B.\bar C.\bar D + \bar A.\bar B.C.D + \bar A.B.C.D + A.\bar B.\bar C.D$

$$$$F0(a, b, c, d) = $\sum([0, 3, 7, 9])$

$$$$F1(a, b, c, d) = $\bar A.\bar B.\bar C.D + \bar A.\bar B.C.D + \bar A.B.\bar C.\bar D + \bar A.B.\bar C.D + A.\bar B.\bar C.D + A.\bar B.C.\bar D + A.B.\bar C.\bar D + A.B.\bar C.D + A.B.C.\bar D$

$$$$F1(a, b, c, d) = $\sum([1, 3, 4, 5, 9, 10, 12, 13, 14])$

$$

\hrule width 1\linewidth
\pagebreak

\subsection{Correction}


\paragraph{Q1}

Simplifier les fonctions suivantes

$$F0(a, b, c, d) = $\bar A.\bar B.\bar C.\bar D + \bar A.\bar B.C.D + \bar A.B.C.D + A.\bar B.\bar C.D$

$$$$F0(a, b, c, d) = $\sum([0, 3, 7, 9])$

$$\begin{karnaugh-map}[4][4][1][CD][AB]
          \minterms{0, 3, 7, 9}
          \maxterms{1, 2, 4, 5, 6, 8, 10, 11, 12, 13, 14, 15}
        %\autoterms[0]
         \terms{}{X}
        % simplification
        \implicant{3}{7}
\implicant{9}{9}
\implicant{0}{0}
          %\implicant{5}{15}
          %\implicantedge{8}{8}{10}{10}
          %\implicantedge{8}{8}{10}{10}[8,10]
        \end{karnaugh-map}$$Simplified Sum of products : $ a'.c.d + a.b'.c'.d + a'.b'.c'.d' $

$$$$F1(a, b, c, d) = $\bar A.\bar B.\bar C.D + \bar A.\bar B.C.D + \bar A.B.\bar C.\bar D + \bar A.B.\bar C.D + A.\bar B.\bar C.D + A.\bar B.C.\bar D + A.B.\bar C.\bar D + A.B.\bar C.D + A.B.C.\bar D$

$$$$F1(a, b, c, d) = $\sum([1, 3, 4, 5, 9, 10, 12, 13, 14])$

$$\begin{karnaugh-map}[4][4][1][CD][AB]
          \minterms{1, 3, 4, 5, 9, 10, 12, 13, 14}
          \maxterms{0, 2, 6, 7, 8, 11, 15}
        %\autoterms[0]
         \terms{}{X}
        % simplification
        \implicant{4}{13}
\implicant{1}{9}
\implicant{14}{10}
\implicant{1}{3}
          %\implicant{5}{15}
          %\implicantedge{8}{8}{10}{10}
          %\implicantedge{8}{8}{10}{10}[8,10]
        \end{karnaugh-map}$$Simplified Sum of products : $ b.c' + c'.d + a.c.d' + a'.b'.d $

$$
\pagebreak

\paragraph{Q1}

Soit la fonction donnée par sa forme canonique, Tracer la table de karnaugh et simplifier.

$$F0(a, b, c, d) = $\bar A.\bar B.C.D + \bar A.B.\bar C.\bar D + \bar A.B.\bar C.D + \bar A.B.C.D + A.B.\bar C.D$

$$$$F0(a, b, c, d) = $\sum([3, 4, 5, 7, 13])$

$$$$F1(a, b, c, d) = $\bar A.\bar B.\bar C.\bar D + \bar A.B.C.\bar D + \bar A.B.C.D + A.\bar B.\bar C.\bar D + A.\bar B.\bar C.D + A.\bar B.C.\bar D + A.B.\bar C.D + A.B.C.\bar D + A.B.C.D$

$$$$F1(a, b, c, d) = $\sum([0, 6, 7, 8, 9, 10, 13, 14, 15])$

$$

\hrule width 1\linewidth
\pagebreak

\subsection{Correction}


\paragraph{Q1}

Simplifier les fonctions suivantes

$$F0(a, b, c, d) = $\bar A.\bar B.C.D + \bar A.B.\bar C.\bar D + \bar A.B.\bar C.D + \bar A.B.C.D + A.B.\bar C.D$

$$$$F0(a, b, c, d) = $\sum([3, 4, 5, 7, 13])$

$$\begin{karnaugh-map}[4][4][1][CD][AB]
          \minterms{3, 4, 5, 7, 13}
          \maxterms{0, 1, 2, 6, 8, 9, 10, 11, 12, 14, 15}
        %\autoterms[0]
         \terms{}{X}
        % simplification
        \implicant{5}{13}
\implicant{3}{7}
\implicant{4}{5}
          %\implicant{5}{15}
          %\implicantedge{8}{8}{10}{10}
          %\implicantedge{8}{8}{10}{10}[8,10]
        \end{karnaugh-map}$$Simplified Sum of products : $ b.c'.d + a'.c.d + a'.b.c' $

$$$$F1(a, b, c, d) = $\bar A.\bar B.\bar C.\bar D + \bar A.B.C.\bar D + \bar A.B.C.D + A.\bar B.\bar C.\bar D + A.\bar B.\bar C.D + A.\bar B.C.\bar D + A.B.\bar C.D + A.B.C.\bar D + A.B.C.D$

$$$$F1(a, b, c, d) = $\sum([0, 6, 7, 8, 9, 10, 13, 14, 15])$

$$\begin{karnaugh-map}[4][4][1][CD][AB]
          \minterms{0, 6, 7, 8, 9, 10, 13, 14, 15}
          \maxterms{1, 2, 3, 4, 5, 11, 12}
        %\autoterms[0]
         \terms{}{X}
        % simplification
        \implicant{7}{14}
\implicant{13}{9}
\implicantedge{8}{8}{10}{10}
\implicantedge{0}{0}{8}{8}
          %\implicant{5}{15}
          %\implicantedge{8}{8}{10}{10}
          %\implicantedge{8}{8}{10}{10}[8,10]
        \end{karnaugh-map}$$Simplified Sum of products : $ b.c + a.c'.d + a.b'.d' + b'.c'.d' $

$$
\pagebreak

\paragraph{Q1}

Etudier la fonction suivante

f(a,b,c,d)= $ a.c.d + b.c.d + a.\bar b.c + a.\bar b.d + \bar a.b.c + \bar a.b.d + \bar a.c.\bar d + \bar b.c.\bar d + \bar a.\bar c.d + \bar b.\bar c.d + a.b.\bar c.\bar d  +  a.\bar c.\bar d + \bar a.\bar b.c.\bar d $


\hrule width 1\linewidth
\pagebreak

\subsection{Correction}


\paragraph{Q1}

$$f(a,b,c,d)=$ a.c.d + b.c.d + a.b'.c + a.b'.d + a'.b.c + a'.b.d + a'.c.d' + b'.c.d' + a'.c'.d + b'.c'.d + a.b.c'.d'  +  a.c'.d' + a'.b'.c.d' $
$$$$f(a,b,c,d)=$ \sum(a.c.d + b.c.d + a.b'.c + a.b'.d + a'.b.c + a'.b.d + a'.c.d' + b'.c.d' + a'.c'.d + b'.c'.d + a.b.c'.d'  +  a.c'.d' + a'.b'.c.d') $ 
$$
%%\begin{table}
        \begin{tabular}{|c|c|c|c|c||c|}
    \toprule
        N° & A & B & C & D & S0\\ \midrule0 & 0 & 0 & 0 & 0 & 0\\1 & 0 & 0 & 0 & 1 & 1\\2 & 0 & 0 & 1 & 0 & 1\\3 & 0 & 0 & 1 & 1 & 0\\\midrule4 & 0 & 1 & 0 & 0 & 0\\5 & 0 & 1 & 0 & 1 & 1\\6 & 0 & 1 & 1 & 0 & 1\\7 & 0 & 1 & 1 & 1 & 1\\\midrule8 & 1 & 0 & 0 & 0 & 1\\9 & 1 & 0 & 0 & 1 & 1\\10 & 1 & 0 & 1 & 0 & 1\\11 & 1 & 0 & 1 & 1 & 1\\\midrule12 & 1 & 1 & 0 & 0 & 1\\13 & 1 & 1 & 0 & 1 & 0\\14 & 1 & 1 & 1 & 0 & 0\\15 & 1 & 1 & 1 & 1 & 1\\\bottomrule
        \end{tabular}
        %%\end{table}
        $$
Sum of products 
 f(a,b,c,d) = \sum($\bar A.\bar B.\bar C.D + \bar A.\bar B.C.\bar D + \bar A.B.\bar C.D + \bar A.B.C.\bar D + \bar A.B.C.D + A.\bar B.\bar C.\bar D + A.\bar B.\bar C.D + A.\bar B.C.\bar D + A.\bar B.C.D + A.B.\bar C.\bar D + A.B.C.D$)
$$$$
Product of sums 
 f(a,b,c,d) = \prod($(A+B+C+D) . (A+B+\bar C+\bar D) . (A+\bar B+C+D) . (\bar A+\bar B+C+\bar D) . (\bar A+\bar B+\bar C+D)$)
$$
Karnough map
\begin{karnaugh-map}[4][4][1][CD][AB]
          \minterms{1, 2, 5, 6, 7, 8, 9, 10, 11, 12, 15}
          \maxterms{0, 3, 4, 13, 14}
        %\autoterms[0]
         \terms{}{X}
        % simplification
        \implicant{8}{10}
\implicant{15}{11}
\implicant{7}{15}
\implicant{7}{6}
\implicant{5}{7}
\implicant{12}{8}
\implicant{2}{6}
\implicantedge{2}{2}{10}{10}
\implicant{1}{5}
\implicantedge{1}{1}{9}{9}
          %\implicant{5}{15}
          %\implicantedge{8}{8}{10}{10}
          %\implicantedge{8}{8}{10}{10}[8,10]
        \end{karnaugh-map}

$$Simplified Sum of products: $ a.b' + a.c.d + b.c.d + a'.b.c + a'.b.d + a.c'.d' + a'.c.d' + b'.c.d' + a'.c'.d + b'.c'.d $
$$$$
Simplified Product of sums: $(a+c+d).(a+b+c'+d').(a'+b'+c+d').(a'+b'+c'+d)$
$$\paragraph{Logigramme} de la fonction\\
        %%\missingfigure[figwidth=6cm]{Logigramme}

 \label{logigram-F}
\begin{tikzpicture}

 %%Paramaters
%% var position, can be modified
\def\varPos{10.00}
\node (x) at (0, \varPos) {$A$};
\node (y) at (0.5, \varPos) {$B$};
\node (z) at (1, \varPos) {$C$};
\node (w) at (1.5, \varPos) {$D$};

            \node[not gate US, draw, rotate=270] at ($(x) + (0.25, -0.6)$) (notx) {};
            \draw ($(x)+(0,-1ex)$) -| (notx.input); 
            \node[not gate US, draw, rotate=270] at ($(y) + (0.25, -0.6)$) (noty) {};
            \draw ($(y)+(0,-1ex)$) -| (noty.input); 
            \node[not gate US, draw, rotate=270] at ($(z) + (0.25, -0.6)$) (notz) {};
            \draw ($(z)+(0,-1ex)$) -| (notz.input);
            \node[not gate US, draw, rotate=270] at ($(w) + (0.25, -0.6)$) (notw) {};
            \draw ($(w)+(0,-1ex)$) -| (notw.input);
         

      
                \node[and gate US, draw, rotate=0, logic gate inputs=nnnn] at (2.5, 0.00) (xandy0) {};\draw (xandy0.output) -- node[above]{\scriptsize $ A.\bar B $} ($(xandy0) + (1.8, 0)$);
                % X
\draw ($(x) + (0, -1ex)$)|- (xandy0.input 1);
% Y
\draw [line width=0.25mm,   red] (noty.output)
            -- ([xshift=0cm]noty.output) |- (xandy0.input 2);
 

      
                \node[and gate US, draw, rotate=0, logic gate inputs=nnnn] at (2.5, 1.20) (xandy1) {};\draw (xandy1.output) -- node[above]{\scriptsize $ A.C.D $} ($(xandy1) + (1.8, 0)$);
                % X
\draw ($(x) + (0, -1ex)$)|- (xandy1.input 1);
% Z
\draw ($(z) + (0, -1ex)$)|- (xandy1.input 3);
% W
\draw ($(w) + (0, -1ex)$)|- (xandy1.input 4);
 

      
                \node[and gate US, draw, rotate=0, logic gate inputs=nnnn] at (2.5, 2.40) (xandy2) {};\draw (xandy2.output) -- node[above]{\scriptsize $ B.C.D $} ($(xandy2) + (1.8, 0)$);
                % Y
\draw ($(y) + (0, -1ex)$)|- (xandy2.input 2);
% Z
\draw ($(z) + (0, -1ex)$)|- (xandy2.input 3);
% W
\draw ($(w) + (0, -1ex)$)|- (xandy2.input 4);
 

      
                \node[and gate US, draw, rotate=0, logic gate inputs=nnnn] at (2.5, 3.60) (xandy3) {};\draw (xandy3.output) -- node[above]{\scriptsize $ \bar A.B.C $} ($(xandy3) + (1.8, 0)$);
                % X
\draw [line width=0.25mm,   red] (notx.output)
            -- ([xshift=0cm]notx.output) |- (xandy3.input 1);
% Y
\draw ($(y) + (0, -1ex)$)|- (xandy3.input 2);
% Z
\draw ($(z) + (0, -1ex)$)|- (xandy3.input 3);
 

      
                \node[and gate US, draw, rotate=0, logic gate inputs=nnnn] at (2.5, 4.80) (xandy4) {};\draw (xandy4.output) -- node[above]{\scriptsize $ \bar A.B.D $} ($(xandy4) + (1.8, 0)$);
                % X
\draw [line width=0.25mm,   red] (notx.output)
            -- ([xshift=0cm]notx.output) |- (xandy4.input 1);
% Y
\draw ($(y) + (0, -1ex)$)|- (xandy4.input 2);
% W
\draw ($(w) + (0, -1ex)$)|- (xandy4.input 4);
 

      
                \node[and gate US, draw, rotate=0, logic gate inputs=nnnn] at (2.5, 6.00) (xandy5) {};\draw (xandy5.output) -- node[above]{\scriptsize $ A.\bar C.\bar D $} ($(xandy5) + (1.8, 0)$);
                % X
\draw ($(x) + (0, -1ex)$)|- (xandy5.input 1);
% Z
\draw [line width=0.25mm,   red] (notz.output)
            -- ([xshift=0cm]notz.output) |- (xandy5.input 3);
% W
\draw [line width=0.25mm,   red] (notw.output)
            -- ([xshift=0cm]notw.output) |- (xandy5.input 4);
 

      
                \node[and gate US, draw, rotate=0, logic gate inputs=nnnn] at (2.5, 7.20) (xandy6) {};\draw (xandy6.output) -- node[above]{\scriptsize $ \bar A.C.\bar D $} ($(xandy6) + (1.8, 0)$);
                % X
\draw [line width=0.25mm,   red] (notx.output)
            -- ([xshift=0cm]notx.output) |- (xandy6.input 1);
% Z
\draw ($(z) + (0, -1ex)$)|- (xandy6.input 3);
% W
\draw [line width=0.25mm,   red] (notw.output)
            -- ([xshift=0cm]notw.output) |- (xandy6.input 4);
 

      
                \node[and gate US, draw, rotate=0, logic gate inputs=nnnn] at (2.5, 8.40) (xandy7) {};\draw (xandy7.output) -- node[above]{\scriptsize $ \bar B.C.\bar D $} ($(xandy7) + (1.8, 0)$);
                % Y
\draw [line width=0.25mm,   red] (noty.output)
            -- ([xshift=0cm]noty.output) |- (xandy7.input 2);
% Z
\draw ($(z) + (0, -1ex)$)|- (xandy7.input 3);
% W
\draw [line width=0.25mm,   red] (notw.output)
            -- ([xshift=0cm]notw.output) |- (xandy7.input 4);
 

      
                \node[and gate US, draw, rotate=0, logic gate inputs=nnnn] at (2.5, 9.60) (xandy8) {};\draw (xandy8.output) -- node[above]{\scriptsize $ \bar A.\bar C.D $} ($(xandy8) + (1.8, 0)$);
                % X
\draw [line width=0.25mm,   red] (notx.output)
            -- ([xshift=0cm]notx.output) |- (xandy8.input 1);
% Z
\draw [line width=0.25mm,   red] (notz.output)
            -- ([xshift=0cm]notz.output) |- (xandy8.input 3);
% W
\draw ($(w) + (0, -1ex)$)|- (xandy8.input 4);
 

      
                \node[and gate US, draw, rotate=0, logic gate inputs=nnnn] at (2.5, 10.80) (xandy9) {};\draw (xandy9.output) -- node[above]{\scriptsize $ \bar B.\bar C.D $} ($(xandy9) + (1.8, 0)$);
                % Y
\draw [line width=0.25mm,   red] (noty.output)
            -- ([xshift=0cm]noty.output) |- (xandy9.input 2);
% Z
\draw [line width=0.25mm,   red] (notz.output)
            -- ([xshift=0cm]notz.output) |- (xandy9.input 3);
% W
\draw ($(w) + (0, -1ex)$)|- (xandy9.input 4);
\node[or gate US, draw, rotate=0, logic gate inputs=nnnnnnnnnnn] at (6, 5.40) (xory0) {};


            \draw (xory0.output) -- node[above]{\scriptsize $F$} ($(xory0.east) + (+3ex, 0)$);


            \draw (xandy0.output) -- ([xshift=1.60cm]xandy0.output) |- (xory0.input 10);

\draw (xandy1.output) -- ([xshift=1.55cm]xandy1.output) |- (xory0.input 9);

\draw (xandy2.output) -- ([xshift=1.50cm]xandy2.output) |- (xory0.input 8);

\draw (xandy3.output) -- ([xshift=1.45cm]xandy3.output) |- (xory0.input 7);

\draw (xandy4.output) -- ([xshift=1.40cm]xandy4.output) |- (xory0.input 6);

\draw (xandy5.output) -- ([xshift=1.35cm]xandy5.output) |- (xory0.input 5);

\draw (xandy6.output) -- ([xshift=1.40cm]xandy6.output) |- (xory0.input 4);

\draw (xandy7.output) -- ([xshift=1.45cm]xandy7.output) |- (xory0.input 3);

\draw (xandy8.output) -- ([xshift=1.50cm]xandy8.output) |- (xory0.input 2);

\draw (xandy9.output) -- ([xshift=1.55cm]xandy9.output) |- (xory0.input 1);

 \end{tikzpicture}


\pagebreak

\paragraph{Q1}

Etudier la fonction suivante

f(a,b,c,d)= $ a.\bar c + a.\bar d + \bar a.b + \bar a.c.d  +  a.c.d + a.b.\bar c + \bar b.c.d + \bar a.\bar c.d + \bar a.\bar b.\bar c $


\hrule width 1\linewidth
\pagebreak

\subsection{Correction}


\paragraph{Q1}

$$f(a,b,c,d)=$ a.c' + a.d' + a'.b + a'.c.d  +  a.c.d + a.b.c' + b'.c.d + a'.c'.d + a'.b'.c' $
$$$$f(a,b,c,d)=$ \sum(a.c' + a.d' + a'.b + a'.c.d  +  a.c.d + a.b.c' + b'.c.d + a'.c'.d + a'.b'.c') $ 
$$
%%\begin{table}
        \begin{tabular}{|c|c|c|c|c||c|}
    \toprule
        N° & A & B & C & D & S0\\ \midrule0 & 0 & 0 & 0 & 0 & 1\\1 & 0 & 0 & 0 & 1 & 1\\2 & 0 & 0 & 1 & 0 & 0\\3 & 0 & 0 & 1 & 1 & 1\\\midrule4 & 0 & 1 & 0 & 0 & 1\\5 & 0 & 1 & 0 & 1 & 1\\6 & 0 & 1 & 1 & 0 & 1\\7 & 0 & 1 & 1 & 1 & 1\\\midrule8 & 1 & 0 & 0 & 0 & 1\\9 & 1 & 0 & 0 & 1 & 1\\10 & 1 & 0 & 1 & 0 & 1\\11 & 1 & 0 & 1 & 1 & 1\\\midrule12 & 1 & 1 & 0 & 0 & 1\\13 & 1 & 1 & 0 & 1 & 1\\14 & 1 & 1 & 1 & 0 & 1\\15 & 1 & 1 & 1 & 1 & 1\\\bottomrule
        \end{tabular}
        %%\end{table}
        $$
Sum of products 
 f(a,b,c,d) = \sum($\bar A.\bar B.\bar C.\bar D + \bar A.\bar B.\bar C.D + \bar A.\bar B.C.D + \bar A.B.\bar C.\bar D + \bar A.B.\bar C.D + \bar A.B.C.\bar D + \bar A.B.C.D + A.\bar B.\bar C.\bar D + A.\bar B.\bar C.D + A.\bar B.C.\bar D + A.\bar B.C.D + A.B.\bar C.\bar D + A.B.\bar C.D + A.B.C.\bar D + A.B.C.D$)
$$$$
Product of sums 
 f(a,b,c,d) = \prod($(A+B+\bar C+D)$)
$$
Karnough map
\begin{karnaugh-map}[4][4][1][CD][AB]
          \minterms{0, 1, 3, 4, 5, 6, 7, 8, 9, 10, 11, 12, 13, 14, 15}
          \maxterms{2}
        %\autoterms[0]
         \terms{}{X}
        % simplification
        \implicant{12}{10}
\implicant{4}{14}
\implicant{1}{11}
\implicant{0}{9}
          %\implicant{5}{15}
          %\implicantedge{8}{8}{10}{10}
          %\implicantedge{8}{8}{10}{10}[8,10]
        \end{karnaugh-map}

$$Simplified Sum of products: $ a + b + d + c' $
$$$$
Simplified Product of sums: $(a+b+c'+d)$
$$\paragraph{Logigramme} de la fonction\\
        %%\missingfigure[figwidth=6cm]{Logigramme}

 \label{logigram-F}
\begin{tikzpicture}

 %%Paramaters
%% var position, can be modified
\def\varPos{4.00}
\node (x) at (0, \varPos) {$A$};
\node (y) at (0.5, \varPos) {$B$};
\node (z) at (1, \varPos) {$C$};
\node (w) at (1.5, \varPos) {$D$};

            \node[not gate US, draw, rotate=270] at ($(x) + (0.25, -0.6)$) (notx) {};
            \draw ($(x)+(0,-1ex)$) -| (notx.input); 
            \node[not gate US, draw, rotate=270] at ($(y) + (0.25, -0.6)$) (noty) {};
            \draw ($(y)+(0,-1ex)$) -| (noty.input); 
            \node[not gate US, draw, rotate=270] at ($(z) + (0.25, -0.6)$) (notz) {};
            \draw ($(z)+(0,-1ex)$) -| (notz.input);
            \node[not gate US, draw, rotate=270] at ($(w) + (0.25, -0.6)$) (notw) {};
            \draw ($(w)+(0,-1ex)$) -| (notw.input);
         

      
                \node[and gate US, draw, rotate=0, logic gate inputs=nnnn] at (2.5, 0.00) (xandy0) {};\draw (xandy0.output) -- node[above]{\scriptsize $ A $} ($(xandy0) + (1.8, 0)$);
                % X
\draw ($(x) + (0, -1ex)$)|- (xandy0.input 1);
 

      
                \node[and gate US, draw, rotate=0, logic gate inputs=nnnn] at (2.5, 1.20) (xandy1) {};\draw (xandy1.output) -- node[above]{\scriptsize $ B $} ($(xandy1) + (1.8, 0)$);
                % Y
\draw ($(y) + (0, -1ex)$)|- (xandy1.input 2);
 

      
                \node[and gate US, draw, rotate=0, logic gate inputs=nnnn] at (2.5, 2.40) (xandy2) {};\draw (xandy2.output) -- node[above]{\scriptsize $ D $} ($(xandy2) + (1.8, 0)$);
                % W
\draw ($(w) + (0, -1ex)$)|- (xandy2.input 4);
 

      
                \node[and gate US, draw, rotate=0, logic gate inputs=nnnn] at (2.5, 3.60) (xandy3) {};\draw (xandy3.output) -- node[above]{\scriptsize $ \bar C $} ($(xandy3) + (1.8, 0)$);
                % Z
\draw [line width=0.25mm,   red] (notz.output)
            -- ([xshift=0cm]notz.output) |- (xandy3.input 3);
\node[or gate US, draw, rotate=0, logic gate inputs=nnnnn] at (6, 1.80) (xory0) {};


            \draw (xory0.output) -- node[above]{\scriptsize $F$} ($(xory0.east) + (+3ex, 0)$);


            \draw (xandy0.output) -- ([xshift=1.60cm]xandy0.output) |- (xory0.input 4);

\draw (xandy1.output) -- ([xshift=1.55cm]xandy1.output) |- (xory0.input 3);

\draw (xandy2.output) -- ([xshift=1.50cm]xandy2.output) |- (xory0.input 2);

\draw (xandy3.output) -- ([xshift=1.55cm]xandy3.output) |- (xory0.input 1);

 \end{tikzpicture}


\pagebreak

\paragraph{Q1}

Etudier la fonction suivante

f(a,b,c,d)= $ \bar a.\bar b + \bar b.c.d + a.b.c.\bar d + a.b.\bar c.d  +  a.\bar d + b.\bar c + a.\bar b.c + \bar a.b.d + \bar b.c.\bar d + \bar a.\bar c.d $


\hrule width 1\linewidth
\pagebreak

\subsection{Correction}


\paragraph{Q1}

$$f(a,b,c,d)=$ a'.b' + b'.c.d + a.b.c.d' + a.b.c'.d  +  a.d' + b.c' + a.b'.c + a'.b.d + b'.c.d' + a'.c'.d $
$$$$f(a,b,c,d)=$ \sum(a'.b' + b'.c.d + a.b.c.d' + a.b.c'.d  +  a.d' + b.c' + a.b'.c + a'.b.d + b'.c.d' + a'.c'.d) $ 
$$
%%\begin{table}
        \begin{tabular}{|c|c|c|c|c||c|}
    \toprule
        N° & A & B & C & D & S0\\ \midrule0 & 0 & 0 & 0 & 0 & 1\\1 & 0 & 0 & 0 & 1 & 1\\2 & 0 & 0 & 1 & 0 & 1\\3 & 0 & 0 & 1 & 1 & 1\\\midrule4 & 0 & 1 & 0 & 0 & 1\\5 & 0 & 1 & 0 & 1 & 1\\6 & 0 & 1 & 1 & 0 & 0\\7 & 0 & 1 & 1 & 1 & 1\\\midrule8 & 1 & 0 & 0 & 0 & 1\\9 & 1 & 0 & 0 & 1 & 0\\10 & 1 & 0 & 1 & 0 & 1\\11 & 1 & 0 & 1 & 1 & 1\\\midrule12 & 1 & 1 & 0 & 0 & 1\\13 & 1 & 1 & 0 & 1 & 1\\14 & 1 & 1 & 1 & 0 & 1\\15 & 1 & 1 & 1 & 1 & 0\\\bottomrule
        \end{tabular}
        %%\end{table}
        $$
Sum of products 
 f(a,b,c,d) = \sum($\bar A.\bar B.\bar C.\bar D + \bar A.\bar B.\bar C.D + \bar A.\bar B.C.\bar D + \bar A.\bar B.C.D + \bar A.B.\bar C.\bar D + \bar A.B.\bar C.D + \bar A.B.C.D + A.\bar B.\bar C.\bar D + A.\bar B.C.\bar D + A.\bar B.C.D + A.B.\bar C.\bar D + A.B.\bar C.D + A.B.C.\bar D$)
$$$$
Product of sums 
 f(a,b,c,d) = \prod($(A+\bar B+\bar C+D) . (\bar A+B+C+\bar D) . (\bar A+\bar B+\bar C+\bar D)$)
$$
Karnough map
\begin{karnaugh-map}[4][4][1][CD][AB]
          \minterms{0, 1, 2, 3, 4, 5, 7, 8, 10, 11, 12, 13, 14}
          \maxterms{6, 9, 15}
        %\autoterms[0]
         \terms{}{X}
        % simplification
        \implicantedge{12}{8}{14}{10}
\implicant{4}{13}
\implicantedge{3}{2}{11}{10}
\implicant{1}{7}
\implicant{0}{2}
          %\implicant{5}{15}
          %\implicantedge{8}{8}{10}{10}
          %\implicantedge{8}{8}{10}{10}[8,10]
        \end{karnaugh-map}

$$Simplified Sum of products: $ a.d' + b.c' + b'.c + a'.d + a'.b' $
$$$$
Simplified Product of sums: $(a+b'+c'+d).(a'+b+c+d').(a'+b'+c'+d')$
$$\paragraph{Logigramme} de la fonction\\
        %%\missingfigure[figwidth=6cm]{Logigramme}

 \label{logigram-F}
\begin{tikzpicture}

 %%Paramaters
%% var position, can be modified
\def\varPos{5.00}
\node (x) at (0, \varPos) {$A$};
\node (y) at (0.5, \varPos) {$B$};
\node (z) at (1, \varPos) {$C$};
\node (w) at (1.5, \varPos) {$D$};

            \node[not gate US, draw, rotate=270] at ($(x) + (0.25, -0.6)$) (notx) {};
            \draw ($(x)+(0,-1ex)$) -| (notx.input); 
            \node[not gate US, draw, rotate=270] at ($(y) + (0.25, -0.6)$) (noty) {};
            \draw ($(y)+(0,-1ex)$) -| (noty.input); 
            \node[not gate US, draw, rotate=270] at ($(z) + (0.25, -0.6)$) (notz) {};
            \draw ($(z)+(0,-1ex)$) -| (notz.input);
            \node[not gate US, draw, rotate=270] at ($(w) + (0.25, -0.6)$) (notw) {};
            \draw ($(w)+(0,-1ex)$) -| (notw.input);
         

      
                \node[and gate US, draw, rotate=0, logic gate inputs=nnnn] at (2.5, 0.00) (xandy0) {};\draw (xandy0.output) -- node[above]{\scriptsize $ A.\bar D $} ($(xandy0) + (1.8, 0)$);
                % X
\draw ($(x) + (0, -1ex)$)|- (xandy0.input 1);
% W
\draw [line width=0.25mm,   red] (notw.output)
            -- ([xshift=0cm]notw.output) |- (xandy0.input 4);
 

      
                \node[and gate US, draw, rotate=0, logic gate inputs=nnnn] at (2.5, 1.20) (xandy1) {};\draw (xandy1.output) -- node[above]{\scriptsize $ B.\bar C $} ($(xandy1) + (1.8, 0)$);
                % Y
\draw ($(y) + (0, -1ex)$)|- (xandy1.input 2);
% Z
\draw [line width=0.25mm,   red] (notz.output)
            -- ([xshift=0cm]notz.output) |- (xandy1.input 3);
 

      
                \node[and gate US, draw, rotate=0, logic gate inputs=nnnn] at (2.5, 2.40) (xandy2) {};\draw (xandy2.output) -- node[above]{\scriptsize $ \bar B.C $} ($(xandy2) + (1.8, 0)$);
                % Y
\draw [line width=0.25mm,   red] (noty.output)
            -- ([xshift=0cm]noty.output) |- (xandy2.input 2);
% Z
\draw ($(z) + (0, -1ex)$)|- (xandy2.input 3);
 

      
                \node[and gate US, draw, rotate=0, logic gate inputs=nnnn] at (2.5, 3.60) (xandy3) {};\draw (xandy3.output) -- node[above]{\scriptsize $ \bar A.D $} ($(xandy3) + (1.8, 0)$);
                % X
\draw [line width=0.25mm,   red] (notx.output)
            -- ([xshift=0cm]notx.output) |- (xandy3.input 1);
% W
\draw ($(w) + (0, -1ex)$)|- (xandy3.input 4);
 

      
                \node[and gate US, draw, rotate=0, logic gate inputs=nnnn] at (2.5, 4.80) (xandy4) {};\draw (xandy4.output) -- node[above]{\scriptsize $ \bar A.\bar B $} ($(xandy4) + (1.8, 0)$);
                % X
\draw [line width=0.25mm,   red] (notx.output)
            -- ([xshift=0cm]notx.output) |- (xandy4.input 1);
% Y
\draw [line width=0.25mm,   red] (noty.output)
            -- ([xshift=0cm]noty.output) |- (xandy4.input 2);
\node[or gate US, draw, rotate=0, logic gate inputs=nnnnnn] at (6, 2.40) (xory0) {};


            \draw (xory0.output) -- node[above]{\scriptsize $F$} ($(xory0.east) + (+3ex, 0)$);


            \draw (xandy0.output) -- ([xshift=1.60cm]xandy0.output) |- (xory0.input 5);

\draw (xandy1.output) -- ([xshift=1.55cm]xandy1.output) |- (xory0.input 4);

\draw (xandy2.output) -- ([xshift=1.50cm]xandy2.output) |- (xory0.input 3);

\draw (xandy3.output) -- ([xshift=1.55cm]xandy3.output) |- (xory0.input 2);

\draw (xandy4.output) -- ([xshift=1.60cm]xandy4.output) |- (xory0.input 1);

 \end{tikzpicture}


\pagebreak

\paragraph{Q1}

Etudier la fonction suivante

f(a,b,c,d)= $ a.\bar b.\bar c + \bar a.\bar b.c + \bar a.\bar b.d  +  \bar b.d + a.\bar b.c + a.\bar c.d + \bar a.\bar b.\bar c + \bar a.b.c.\bar d $


\hrule width 1\linewidth
\pagebreak

\subsection{Correction}


\paragraph{Q1}

$$f(a,b,c,d)=$ a.b'.c' + a'.b'.c + a'.b'.d  +  b'.d + a.b'.c + a.c'.d + a'.b'.c' + a'.b.c.d' $
$$$$f(a,b,c,d)=$ \sum(a.b'.c' + a'.b'.c + a'.b'.d  +  b'.d + a.b'.c + a.c'.d + a'.b'.c' + a'.b.c.d') $ 
$$
%%\begin{table}
        \begin{tabular}{|c|c|c|c|c||c|}
    \toprule
        N° & A & B & C & D & S0\\ \midrule0 & 0 & 0 & 0 & 0 & 1\\1 & 0 & 0 & 0 & 1 & 1\\2 & 0 & 0 & 1 & 0 & 1\\3 & 0 & 0 & 1 & 1 & 1\\\midrule4 & 0 & 1 & 0 & 0 & 0\\5 & 0 & 1 & 0 & 1 & 0\\6 & 0 & 1 & 1 & 0 & 1\\7 & 0 & 1 & 1 & 1 & 0\\\midrule8 & 1 & 0 & 0 & 0 & 1\\9 & 1 & 0 & 0 & 1 & 1\\10 & 1 & 0 & 1 & 0 & 1\\11 & 1 & 0 & 1 & 1 & 1\\\midrule12 & 1 & 1 & 0 & 0 & 0\\13 & 1 & 1 & 0 & 1 & 1\\14 & 1 & 1 & 1 & 0 & 0\\15 & 1 & 1 & 1 & 1 & 0\\\bottomrule
        \end{tabular}
        %%\end{table}
        $$
Sum of products 
 f(a,b,c,d) = \sum($\bar A.\bar B.\bar C.\bar D + \bar A.\bar B.\bar C.D + \bar A.\bar B.C.\bar D + \bar A.\bar B.C.D + \bar A.B.C.\bar D + A.\bar B.\bar C.\bar D + A.\bar B.\bar C.D + A.\bar B.C.\bar D + A.\bar B.C.D + A.B.\bar C.D$)
$$$$
Product of sums 
 f(a,b,c,d) = \prod($(A+\bar B+C+D) . (A+\bar B+C+\bar D) . (A+\bar B+\bar C+\bar D) . (\bar A+\bar B+C+D) . (\bar A+\bar B+\bar C+D) . (\bar A+\bar B+\bar C+\bar D)$)
$$
Karnough map
\begin{karnaugh-map}[4][4][1][CD][AB]
          \minterms{0, 1, 2, 3, 6, 8, 9, 10, 11, 13}
          \maxterms{4, 5, 7, 12, 14, 15}
        %\autoterms[0]
         \terms{}{X}
        % simplification
        \implicantedge{0}{2}{8}{10}
\implicant{13}{9}
\implicant{2}{6}
          %\implicant{5}{15}
          %\implicantedge{8}{8}{10}{10}
          %\implicantedge{8}{8}{10}{10}[8,10]
        \end{karnaugh-map}

$$Simplified Sum of products: $ b' + a.c'.d + a'.c.d' $
$$$$
Simplified Product of sums: $(a+b'+c).(b'+c+d).(a+b'+d').(a'+b'+d).(a'+b'+c').(b'+c'+d')$
$$\paragraph{Logigramme} de la fonction\\
        %%\missingfigure[figwidth=6cm]{Logigramme}

 \label{logigram-F}
\begin{tikzpicture}

 %%Paramaters
%% var position, can be modified
\def\varPos{3.00}
\node (x) at (0, \varPos) {$A$};
\node (y) at (0.5, \varPos) {$B$};
\node (z) at (1, \varPos) {$C$};
\node (w) at (1.5, \varPos) {$D$};

            \node[not gate US, draw, rotate=270] at ($(x) + (0.25, -0.6)$) (notx) {};
            \draw ($(x)+(0,-1ex)$) -| (notx.input); 
            \node[not gate US, draw, rotate=270] at ($(y) + (0.25, -0.6)$) (noty) {};
            \draw ($(y)+(0,-1ex)$) -| (noty.input); 
            \node[not gate US, draw, rotate=270] at ($(z) + (0.25, -0.6)$) (notz) {};
            \draw ($(z)+(0,-1ex)$) -| (notz.input);
            \node[not gate US, draw, rotate=270] at ($(w) + (0.25, -0.6)$) (notw) {};
            \draw ($(w)+(0,-1ex)$) -| (notw.input);
         

      
                \node[and gate US, draw, rotate=0, logic gate inputs=nnnn] at (2.5, 0.00) (xandy0) {};\draw (xandy0.output) -- node[above]{\scriptsize $ \bar B $} ($(xandy0) + (1.8, 0)$);
                % Y
\draw [line width=0.25mm,   red] (noty.output)
            -- ([xshift=0cm]noty.output) |- (xandy0.input 2);
 

      
                \node[and gate US, draw, rotate=0, logic gate inputs=nnnn] at (2.5, 1.20) (xandy1) {};\draw (xandy1.output) -- node[above]{\scriptsize $ A.\bar C.D $} ($(xandy1) + (1.8, 0)$);
                % X
\draw ($(x) + (0, -1ex)$)|- (xandy1.input 1);
% Z
\draw [line width=0.25mm,   red] (notz.output)
            -- ([xshift=0cm]notz.output) |- (xandy1.input 3);
% W
\draw ($(w) + (0, -1ex)$)|- (xandy1.input 4);
 

      
                \node[and gate US, draw, rotate=0, logic gate inputs=nnnn] at (2.5, 2.40) (xandy2) {};\draw (xandy2.output) -- node[above]{\scriptsize $ \bar A.C.\bar D $} ($(xandy2) + (1.8, 0)$);
                % X
\draw [line width=0.25mm,   red] (notx.output)
            -- ([xshift=0cm]notx.output) |- (xandy2.input 1);
% Z
\draw ($(z) + (0, -1ex)$)|- (xandy2.input 3);
% W
\draw [line width=0.25mm,   red] (notw.output)
            -- ([xshift=0cm]notw.output) |- (xandy2.input 4);
\node[or gate US, draw, rotate=0, logic gate inputs=nnnn] at (6, 1.20) (xory0) {};


            \draw (xory0.output) -- node[above]{\scriptsize $F$} ($(xory0.east) + (+3ex, 0)$);


            \draw (xandy0.output) -- ([xshift=1.60cm]xandy0.output) |- (xory0.input 3);

\draw (xandy1.output) -- ([xshift=1.55cm]xandy1.output) |- (xory0.input 2);

\draw (xandy2.output) -- ([xshift=1.60cm]xandy2.output) |- (xory0.input 1);

 \end{tikzpicture}


\pagebreak

\paragraph{Q1}

Etudier la fonction suivante

f(a,b,c,d)= $ b.d + a.b.c + \bar b.\bar c  +  \bar a.\bar c + \bar c.\bar d + a.b.c.d $


\hrule width 1\linewidth
\pagebreak

\subsection{Correction}


\paragraph{Q1}

$$f(a,b,c,d)=$ b.d + a.b.c + b'.c'  +  a'.c' + c'.d' + a.b.c.d $
$$$$f(a,b,c,d)=$ \sum(b.d + a.b.c + b'.c'  +  a'.c' + c'.d' + a.b.c.d) $ 
$$
%%\begin{table}
        \begin{tabular}{|c|c|c|c|c||c|}
    \toprule
        N° & A & B & C & D & S0\\ \midrule0 & 0 & 0 & 0 & 0 & 1\\1 & 0 & 0 & 0 & 1 & 1\\2 & 0 & 0 & 1 & 0 & 0\\3 & 0 & 0 & 1 & 1 & 0\\\midrule4 & 0 & 1 & 0 & 0 & 1\\5 & 0 & 1 & 0 & 1 & 1\\6 & 0 & 1 & 1 & 0 & 0\\7 & 0 & 1 & 1 & 1 & 1\\\midrule8 & 1 & 0 & 0 & 0 & 1\\9 & 1 & 0 & 0 & 1 & 1\\10 & 1 & 0 & 1 & 0 & 0\\11 & 1 & 0 & 1 & 1 & 0\\\midrule12 & 1 & 1 & 0 & 0 & 1\\13 & 1 & 1 & 0 & 1 & 1\\14 & 1 & 1 & 1 & 0 & 1\\15 & 1 & 1 & 1 & 1 & 1\\\bottomrule
        \end{tabular}
        %%\end{table}
        $$
Sum of products 
 f(a,b,c,d) = \sum($\bar A.\bar B.\bar C.\bar D + \bar A.\bar B.\bar C.D + \bar A.B.\bar C.\bar D + \bar A.B.\bar C.D + \bar A.B.C.D + A.\bar B.\bar C.\bar D + A.\bar B.\bar C.D + A.B.\bar C.\bar D + A.B.\bar C.D + A.B.C.\bar D + A.B.C.D$)
$$$$
Product of sums 
 f(a,b,c,d) = \prod($(A+B+\bar C+D) . (A+B+\bar C+\bar D) . (A+\bar B+\bar C+D) . (\bar A+B+\bar C+D) . (\bar A+B+\bar C+\bar D)$)
$$
Karnough map
\begin{karnaugh-map}[4][4][1][CD][AB]
          \minterms{0, 1, 4, 5, 7, 8, 9, 12, 13, 14, 15}
          \maxterms{2, 3, 6, 10, 11}
        %\autoterms[0]
         \terms{}{X}
        % simplification
        \implicant{0}{9}
\implicant{12}{14}
\implicant{5}{15}
          %\implicant{5}{15}
          %\implicantedge{8}{8}{10}{10}
          %\implicantedge{8}{8}{10}{10}[8,10]
        \end{karnaugh-map}

$$Simplified Sum of products: $ c' + a.b + b.d $
$$$$
Simplified Product of sums: $(b+c').(a+c'+d)$
$$\paragraph{Logigramme} de la fonction\\
        %%\missingfigure[figwidth=6cm]{Logigramme}

 \label{logigram-F}
\begin{tikzpicture}

 %%Paramaters
%% var position, can be modified
\def\varPos{3.00}
\node (x) at (0, \varPos) {$A$};
\node (y) at (0.5, \varPos) {$B$};
\node (z) at (1, \varPos) {$C$};
\node (w) at (1.5, \varPos) {$D$};

            \node[not gate US, draw, rotate=270] at ($(x) + (0.25, -0.6)$) (notx) {};
            \draw ($(x)+(0,-1ex)$) -| (notx.input); 
            \node[not gate US, draw, rotate=270] at ($(y) + (0.25, -0.6)$) (noty) {};
            \draw ($(y)+(0,-1ex)$) -| (noty.input); 
            \node[not gate US, draw, rotate=270] at ($(z) + (0.25, -0.6)$) (notz) {};
            \draw ($(z)+(0,-1ex)$) -| (notz.input);
            \node[not gate US, draw, rotate=270] at ($(w) + (0.25, -0.6)$) (notw) {};
            \draw ($(w)+(0,-1ex)$) -| (notw.input);
         

      
                \node[and gate US, draw, rotate=0, logic gate inputs=nnnn] at (2.5, 0.00) (xandy0) {};\draw (xandy0.output) -- node[above]{\scriptsize $ \bar C $} ($(xandy0) + (1.8, 0)$);
                % Z
\draw [line width=0.25mm,   red] (notz.output)
            -- ([xshift=0cm]notz.output) |- (xandy0.input 3);
 

      
                \node[and gate US, draw, rotate=0, logic gate inputs=nnnn] at (2.5, 1.20) (xandy1) {};\draw (xandy1.output) -- node[above]{\scriptsize $ A.B $} ($(xandy1) + (1.8, 0)$);
                % X
\draw ($(x) + (0, -1ex)$)|- (xandy1.input 1);
% Y
\draw ($(y) + (0, -1ex)$)|- (xandy1.input 2);
 

      
                \node[and gate US, draw, rotate=0, logic gate inputs=nnnn] at (2.5, 2.40) (xandy2) {};\draw (xandy2.output) -- node[above]{\scriptsize $ B.D $} ($(xandy2) + (1.8, 0)$);
                % Y
\draw ($(y) + (0, -1ex)$)|- (xandy2.input 2);
% W
\draw ($(w) + (0, -1ex)$)|- (xandy2.input 4);
\node[or gate US, draw, rotate=0, logic gate inputs=nnnn] at (6, 1.20) (xory0) {};


            \draw (xory0.output) -- node[above]{\scriptsize $F$} ($(xory0.east) + (+3ex, 0)$);


            \draw (xandy0.output) -- ([xshift=1.60cm]xandy0.output) |- (xory0.input 3);

\draw (xandy1.output) -- ([xshift=1.55cm]xandy1.output) |- (xory0.input 2);

\draw (xandy2.output) -- ([xshift=1.60cm]xandy2.output) |- (xory0.input 1);

 \end{tikzpicture}


\pagebreak

\paragraph{Q1}

Faire les conversions suivantes: 
$(187)_{5} = (........)_{2}$

\hrule width 1\linewidth
\pagebreak

\subsection{Correction}


\paragraph{Q1}

$(187)_{5} = (10111011)_{2}$
\pagebreak

\paragraph{Q1}

Faire les conversions suivantes: 
$(224)_{8} = (........)_{2}$

\hrule width 1\linewidth
\pagebreak

\subsection{Correction}


\paragraph{Q1}

$(224)_{8} = (11100000)_{2}$
\pagebreak

\paragraph{Q1}

Faire les conversions suivantes: 
$(162186048)_{8} = (........)_{16}$

\hrule width 1\linewidth
\pagebreak

\subsection{Correction}


\paragraph{Q1}

$(162186048)_{8} = (9aac340)_{16}$
\pagebreak

\paragraph{Q1}

Faire les conversions suivantes: 
$(214)_{8} = (........)_{2}$

\hrule width 1\linewidth
\pagebreak

\subsection{Correction}


\paragraph{Q1}

$(214)_{8} = (11010110)_{2}$
\pagebreak

\paragraph{Q1}

Faire les conversions suivantes: 
$(243)_{8} = (........)_{2}$

\hrule width 1\linewidth
\pagebreak

\subsection{Correction}


\paragraph{Q1}

$(243)_{8} = (11110011)_{2}$
\pagebreak

\paragraph{Q1}

Faire les opérations suivantes: 
  41230334
-  2140344
----------



\hrule width 1\linewidth
\pagebreak

\subsection{Correction}


\paragraph{Q1}

  41230334
-  2140344
----------
  34034440

\pagebreak

\paragraph{Q1}

Faire les opérations suivantes: 
   3213032
-  1241103
----------



\hrule width 1\linewidth
\pagebreak

\subsection{Correction}


\paragraph{Q1}

   3213032
-  1241103
----------
   1421424

\pagebreak

\paragraph{Q1}

Faire les opérations suivantes: 
  33544353
+ 54302204
----------



\hrule width 1\linewidth
\pagebreak

\subsection{Correction}


\paragraph{Q1}

  33544353
+ 54302204
----------
 132251001

\pagebreak

\paragraph{Q1}

Faire les opérations suivantes: 
  215d4878
+ 8531091f
----------



\hrule width 1\linewidth
\pagebreak

\subsection{Correction}


\paragraph{Q1}

  215d4878
+ 8531091f
----------
  a68e5197

\pagebreak

\paragraph{Q1}

Faire les opérations suivantes: 
  35501437
- 27663367
----------



\hrule width 1\linewidth
\pagebreak

\subsection{Correction}


\paragraph{Q1}

  35501437
- 27663367
----------
   5616050

\pagebreak

\paragraph{Q1}

Faire les opérations suivantes: 
  42455547
- 13667761
----------



\hrule width 1\linewidth
\pagebreak

\subsection{Correction}


\paragraph{Q1}

  42455547
- 13667761
----------
  26565566

\pagebreak

\paragraph{Q1}

Faire les opérations suivantes: 
  17612432
- 10745041
----------



\hrule width 1\linewidth
\pagebreak

\subsection{Correction}


\paragraph{Q1}

  17612432
- 10745041
----------
   6645371

\pagebreak

\paragraph{Q1}

Faire les opérations suivantes: 
  50447342
- 30452277
----------



\hrule width 1\linewidth
\pagebreak

\subsection{Correction}


\paragraph{Q1}

  50447342
- 30452277
----------
  17775043

\pagebreak

\paragraph{Q1}

Faire les opérations suivantes: 
  bc013fbe
+ 5c40fce4
----------



\hrule width 1\linewidth
\pagebreak

\subsection{Correction}


\paragraph{Q1}

  bc013fbe
+ 5c40fce4
----------
 118423ca2

\pagebreak

\paragraph{Q1}

Faire les opérations suivantes: 
  10101011
-  1100111
----------



\hrule width 1\linewidth
\pagebreak

\subsection{Correction}


\paragraph{Q1}

  10101011
-  1100111
----------
   1000100

\pagebreak

\paragraph{Q1}


P10(A, B, C, D) = $[10, 11, 12, 13, 15]$


\hrule width 1\linewidth
\pagebreak

\subsection{Correction}


\paragraph{Q1}

P10(A, B, C, D) =$[10, 11, 12, 13, 15]$

P10(A, B, C, D) =$ \sum([10, 11, 12, 13, 15]) $ 




Sum of products 
 P10(A, B, C, D) = $A.\bar B.C.\bar D + A.\bar B.C.D + A.B.\bar C.\bar D + A.B.\bar C.D + A.B.C.D$


Product of sums 
 P10(A, B, C, D) = $(A+B+C+D) . (A+B+C+\bar D) . (A+B+\bar C+D) . (A+B+\bar C+\bar D) . (A+\bar B+C+D) . (A+\bar B+C+\bar D) . (A+\bar B+\bar C+D) . (A+\bar B+\bar C+\bar D) . (\bar A+B+C+D) . (\bar A+B+C+\bar D) . (\bar A+\bar B+\bar C+D)$


\paragraph{Karnough map}

\begin{karnaugh-map}[4][4][1][CD][AB]
          \minterms{10, 11, 12, 13, 15}
          \maxterms{0, 1, 2, 3, 5, 6, 7, 8}
        %\autoterms[0]
         \terms{4, 9, 14}{X}
        % simplification
        \implicant{12}{14}
\implicant{15}{10}
          %\implicant{5}{15}
          %\implicantedge{8}{8}{10}{10}
          %\implicantedge{8}{8}{10}{10}[8,10]
        \end{karnaugh-map}



Simplified Sum of products: $ a.b + a.c $


Simplified Product of sums: $(a).(b+c)$


NAND
 first simplified from: $$$$ a.b + a.c $$
$$\overline{\overline{ a.b + a.c }}$$
$$\overline{(\overline{ a.b }).(\overline{ a.c })}$$
$$( A\uparrow B )\big\uparrow ( A\uparrow C )$$$$


NAND
 second simplified from: $$$$(a).(b+c)$$
$$(\overline{\overline{a}}).(\overline{\overline{b+c}})$$
$$\overline{\overline{(\overline{\overline{a}}).(\overline{\overline{b+c}})}}$$
$$(((A \uparrow  A))\uparrow ((B \uparrow  B)\uparrow (C \uparrow  C)))\uparrow (((A \uparrow  A))\uparrow ((B \uparrow  B)\uparrow (C \uparrow  C)))$$$$


NAND
 first from: $$$$A.\bar B.C.\bar D + A.\bar B.C.D + A.B.\bar C.\bar D + A.B.\bar C.D + A.B.C.D$$
$$\overline{\overline{A.\bar B.C.\bar D + A.\bar B.C.D + A.B.\bar C.\bar D + A.B.\bar C.D + A.B.C.D}}$$
$$\overline{(\overline{A.\bar B.C.\bar D }).(\overline{ A.\bar B.C.D }).(\overline{ A.B.\bar C.\bar D }).(\overline{ A.B.\bar C.D }).(\overline{ A.B.C.D})}$$
$$(A\uparrow (B \uparrow  B)\uparrow C\uparrow (D \uparrow  D) )\big\uparrow ( A\uparrow (B \uparrow  B)\uparrow C\uparrow D )\big\uparrow ( A\uparrow B\uparrow (C \uparrow  C)\uparrow (D \uparrow  D) )\big\uparrow ( A\uparrow B\uparrow (C \uparrow  C)\uparrow D )\big\uparrow ( A\uparrow B\uparrow C\uparrow D)$$$$


NAND
 second from: $$$$(A+B+C+D) . (A+B+C+\bar D) . (A+B+\bar C+D) . (A+B+\bar C+\bar D) . (A+\bar B+C+D) . (A+\bar B+C+\bar D) . (A+\bar B+\bar C+D) . (A+\bar B+\bar C+\bar D) . (\bar A+B+C+D) . (\bar A+B+C+\bar D) . (\bar A+\bar B+\bar C+D)$$
$$(\overline{\overline{A+B+C+D}}) . (\overline{\overline{A+B+C+\bar D}}) . (\overline{\overline{A+B+\bar C+D}}) . (\overline{\overline{A+B+\bar C+\bar D}}) . (\overline{\overline{A+\bar B+C+D}}) . (\overline{\overline{A+\bar B+C+\bar D}}) . (\overline{\overline{A+\bar B+\bar C+D}}) . (\overline{\overline{A+\bar B+\bar C+\bar D}}) . (\overline{\overline{\bar A+B+C+D}}) . (\overline{\overline{\bar A+B+C+\bar D}}) . (\overline{\overline{\bar A+\bar B+\bar C+D}})$$
$$\overline{\overline{(\overline{\overline{A+B+C+D}}) . (\overline{\overline{A+B+C+\bar D}}) . (\overline{\overline{A+B+\bar C+D}}) . (\overline{\overline{A+B+\bar C+\bar D}}) . (\overline{\overline{A+\bar B+C+D}}) . (\overline{\overline{A+\bar B+C+\bar D}}) . (\overline{\overline{A+\bar B+\bar C+D}}) . (\overline{\overline{A+\bar B+\bar C+\bar D}}) . (\overline{\overline{\bar A+B+C+D}}) . (\overline{\overline{\bar A+B+C+\bar D}}) . (\overline{\overline{\bar A+\bar B+\bar C+D}})}}$$
$$(((A \uparrow  A)\uparrow (B \uparrow  B)\uparrow (C \uparrow  C)\uparrow (D \uparrow  D)) \uparrow  ((A \uparrow  A)\uparrow (B \uparrow  B)\uparrow (C \uparrow  C)\uparrow D) \uparrow  ((A \uparrow  A)\uparrow (B \uparrow  B)\uparrow C\uparrow (D \uparrow  D)) \uparrow  ((A \uparrow  A)\uparrow (B \uparrow  B)\uparrow C\uparrow D) \uparrow  ((A \uparrow  A)\uparrow B\uparrow (C \uparrow  C)\uparrow (D \uparrow  D)) \uparrow  ((A \uparrow  A)\uparrow B\uparrow (C \uparrow  C)\uparrow D) \uparrow  ((A \uparrow  A)\uparrow B\uparrow C\uparrow (D \uparrow  D)) \uparrow  ((A \uparrow  A)\uparrow B\uparrow C\uparrow D) \uparrow  (A\uparrow (B \uparrow  B)\uparrow (C \uparrow  C)\uparrow (D \uparrow  D)) \uparrow  (A\uparrow (B \uparrow  B)\uparrow (C \uparrow  C)\uparrow D) \uparrow  (A\uparrow B\uparrow C\uparrow (D \uparrow  D)))\uparrow (((A \uparrow  A)\uparrow (B \uparrow  B)\uparrow (C \uparrow  C)\uparrow (D \uparrow  D)) \uparrow  ((A \uparrow  A)\uparrow (B \uparrow  B)\uparrow (C \uparrow  C)\uparrow D) \uparrow  ((A \uparrow  A)\uparrow (B \uparrow  B)\uparrow C\uparrow (D \uparrow  D)) \uparrow  ((A \uparrow  A)\uparrow (B \uparrow  B)\uparrow C\uparrow D) \uparrow  ((A \uparrow  A)\uparrow B\uparrow (C \uparrow  C)\uparrow (D \uparrow  D)) \uparrow  ((A \uparrow  A)\uparrow B\uparrow (C \uparrow  C)\uparrow D) \uparrow  ((A \uparrow  A)\uparrow B\uparrow C\uparrow (D \uparrow  D)) \uparrow  ((A \uparrow  A)\uparrow B\uparrow C\uparrow D) \uparrow  (A\uparrow (B \uparrow  B)\uparrow (C \uparrow  C)\uparrow (D \uparrow  D)) \uparrow  (A\uparrow (B \uparrow  B)\uparrow (C \uparrow  C)\uparrow D) \uparrow  (A\uparrow B\uparrow C\uparrow (D \uparrow  D)))$$$$


\paragraph{Logigramme de la fonction}

 \label{logigram-P10}
\begin{tikzpicture}

 %%Paramaters
%% var position, can be modified
\def\varPos{2.00}
\node (x) at (0, \varPos) {$A$};
\node (y) at (0.5, \varPos) {$B$};
\node (z) at (1, \varPos) {$C$};
\node (w) at (1.5, \varPos) {$D$};

            \node[nand gate US, draw, rotate=270, scale=0.5, logic gate inputs=nn] at ($(x) + (0.25, -0.6)$) (notx) {};
            \draw ($(x)+(0,-1ex)$) -| (notx.input 1); 
            \node[nand gate US, draw, rotate=270, scale=0.5, logic gate inputs=nn] at ($(y) + (0.25, -0.6)$) (noty) {};
            \draw ($(y)+(0,-1ex)$) -| (noty.input 1); 
            \node[nand gate US, draw, rotate=270, scale=0.5, logic gate inputs=nn] at ($(z) + (0.25, -0.6)$) (notz) {};
            \draw ($(z)+(0,-1ex)$) -| (notz.input 1);
            \node[nand gate US, draw, rotate=270, scale=0.5, logic gate inputs=nn] at ($(w) + (0.25, -0.6)$) (notw) {};
            \draw ($(w)+(0,-1ex)$) -| (notw.input 1);
         

      
                \node[nand gate US, draw, rotate=0, logic gate inputs=nnnn] at (2.5, 0.00) (xandy0) {};\draw (xandy0.output) -- node[above]{\scriptsize $ A.B $} ($(xandy0) + (1.8, 0)$);
                % X
\draw ($(x) + (0, -1ex)$)|- (xandy0.input 1);
% Y
\draw ($(y) + (0, -1ex)$)|- (xandy0.input 2);
 

      
                \node[nand gate US, draw, rotate=0, logic gate inputs=nnnn] at (2.5, 1.20) (xandy1) {};\draw (xandy1.output) -- node[above]{\scriptsize $ A.C $} ($(xandy1) + (1.8, 0)$);
                % X
\draw ($(x) + (0, -1ex)$)|- (xandy1.input 1);
% Z
\draw ($(z) + (0, -1ex)$)|- (xandy1.input 3);
\node[nand gate US, draw, rotate=0, logic gate inputs=nnn] at (6, 0.60) (xory0) {};


            \draw (xory0.output) -- node[above]{\scriptsize $P10$} ($(xory0.east) + (+3ex, 0)$);


            \draw (xandy0.output) -- ([xshift=1.60cm]xandy0.output) |- (xory0.input 2);

\draw (xandy1.output) -- ([xshift=1.55cm]xandy1.output) |- (xory0.input 1);

 \end{tikzpicture}


\pagebreak

\paragraph{Q1}


P10(A, B, C, D) = $[10, 11, 12, 13, 15]$


\hrule width 1\linewidth
\pagebreak

\subsection{Correction}


\paragraph{Q1}

P10(A, B, C, D) =$[10, 11, 12, 13, 15]$

P10(A, B, C, D) =$ \sum([10, 11, 12, 13, 15]) $ 




Sum of products 
 P10(A, B, C, D) = $A.\bar B.C.\bar D + A.\bar B.C.D + A.B.\bar C.\bar D + A.B.\bar C.D + A.B.C.D$


Product of sums 
 P10(A, B, C, D) = $(A+B+C+D) . (A+B+C+\bar D) . (A+B+\bar C+D) . (A+B+\bar C+\bar D) . (A+\bar B+C+D) . (A+\bar B+C+\bar D) . (A+\bar B+\bar C+D) . (A+\bar B+\bar C+\bar D) . (\bar A+B+C+D) . (\bar A+B+C+\bar D) . (\bar A+\bar B+\bar C+D)$


\paragraph{Karnough map}

\begin{karnaugh-map}[4][4][1][CD][AB]
          \minterms{10, 11, 12, 13, 15}
          \maxterms{0, 1, 2, 3, 5, 6, 7, 8}
        %\autoterms[0]
         \terms{4, 9, 14}{X}
        % simplification
        \implicant{12}{14}
\implicant{15}{10}
          %\implicant{5}{15}
          %\implicantedge{8}{8}{10}{10}
          %\implicantedge{8}{8}{10}{10}[8,10]
        \end{karnaugh-map}



Simplified Sum of products: $ a.b + a.c $


Simplified Product of sums: $(a).(b+c)$


NAND
 first simplified from: $$$$ a.b + a.c $$
$$\overline{\overline{ a.b + a.c }}$$
$$\overline{(\overline{ a.b }).(\overline{ a.c })}$$
$$( A\uparrow B )\big\uparrow ( A\uparrow C )$$$$


NAND
 second simplified from: $$$$(a).(b+c)$$
$$(\overline{\overline{a}}).(\overline{\overline{b+c}})$$
$$\overline{\overline{(\overline{\overline{a}}).(\overline{\overline{b+c}})}}$$
$$(((A \uparrow  A))\uparrow ((B \uparrow  B)\uparrow (C \uparrow  C)))\uparrow (((A \uparrow  A))\uparrow ((B \uparrow  B)\uparrow (C \uparrow  C)))$$$$


NAND
 first from: $$$$A.\bar B.C.\bar D + A.\bar B.C.D + A.B.\bar C.\bar D + A.B.\bar C.D + A.B.C.D$$
$$\overline{\overline{A.\bar B.C.\bar D + A.\bar B.C.D + A.B.\bar C.\bar D + A.B.\bar C.D + A.B.C.D}}$$
$$\overline{(\overline{A.\bar B.C.\bar D }).(\overline{ A.\bar B.C.D }).(\overline{ A.B.\bar C.\bar D }).(\overline{ A.B.\bar C.D }).(\overline{ A.B.C.D})}$$
$$(A\uparrow (B \uparrow  B)\uparrow C\uparrow (D \uparrow  D) )\big\uparrow ( A\uparrow (B \uparrow  B)\uparrow C\uparrow D )\big\uparrow ( A\uparrow B\uparrow (C \uparrow  C)\uparrow (D \uparrow  D) )\big\uparrow ( A\uparrow B\uparrow (C \uparrow  C)\uparrow D )\big\uparrow ( A\uparrow B\uparrow C\uparrow D)$$$$


NAND
 second from: $$$$(A+B+C+D) . (A+B+C+\bar D) . (A+B+\bar C+D) . (A+B+\bar C+\bar D) . (A+\bar B+C+D) . (A+\bar B+C+\bar D) . (A+\bar B+\bar C+D) . (A+\bar B+\bar C+\bar D) . (\bar A+B+C+D) . (\bar A+B+C+\bar D) . (\bar A+\bar B+\bar C+D)$$
$$(\overline{\overline{A+B+C+D}}) . (\overline{\overline{A+B+C+\bar D}}) . (\overline{\overline{A+B+\bar C+D}}) . (\overline{\overline{A+B+\bar C+\bar D}}) . (\overline{\overline{A+\bar B+C+D}}) . (\overline{\overline{A+\bar B+C+\bar D}}) . (\overline{\overline{A+\bar B+\bar C+D}}) . (\overline{\overline{A+\bar B+\bar C+\bar D}}) . (\overline{\overline{\bar A+B+C+D}}) . (\overline{\overline{\bar A+B+C+\bar D}}) . (\overline{\overline{\bar A+\bar B+\bar C+D}})$$
$$\overline{\overline{(\overline{\overline{A+B+C+D}}) . (\overline{\overline{A+B+C+\bar D}}) . (\overline{\overline{A+B+\bar C+D}}) . (\overline{\overline{A+B+\bar C+\bar D}}) . (\overline{\overline{A+\bar B+C+D}}) . (\overline{\overline{A+\bar B+C+\bar D}}) . (\overline{\overline{A+\bar B+\bar C+D}}) . (\overline{\overline{A+\bar B+\bar C+\bar D}}) . (\overline{\overline{\bar A+B+C+D}}) . (\overline{\overline{\bar A+B+C+\bar D}}) . (\overline{\overline{\bar A+\bar B+\bar C+D}})}}$$
$$(((A \uparrow  A)\uparrow (B \uparrow  B)\uparrow (C \uparrow  C)\uparrow (D \uparrow  D)) \uparrow  ((A \uparrow  A)\uparrow (B \uparrow  B)\uparrow (C \uparrow  C)\uparrow D) \uparrow  ((A \uparrow  A)\uparrow (B \uparrow  B)\uparrow C\uparrow (D \uparrow  D)) \uparrow  ((A \uparrow  A)\uparrow (B \uparrow  B)\uparrow C\uparrow D) \uparrow  ((A \uparrow  A)\uparrow B\uparrow (C \uparrow  C)\uparrow (D \uparrow  D)) \uparrow  ((A \uparrow  A)\uparrow B\uparrow (C \uparrow  C)\uparrow D) \uparrow  ((A \uparrow  A)\uparrow B\uparrow C\uparrow (D \uparrow  D)) \uparrow  ((A \uparrow  A)\uparrow B\uparrow C\uparrow D) \uparrow  (A\uparrow (B \uparrow  B)\uparrow (C \uparrow  C)\uparrow (D \uparrow  D)) \uparrow  (A\uparrow (B \uparrow  B)\uparrow (C \uparrow  C)\uparrow D) \uparrow  (A\uparrow B\uparrow C\uparrow (D \uparrow  D)))\uparrow (((A \uparrow  A)\uparrow (B \uparrow  B)\uparrow (C \uparrow  C)\uparrow (D \uparrow  D)) \uparrow  ((A \uparrow  A)\uparrow (B \uparrow  B)\uparrow (C \uparrow  C)\uparrow D) \uparrow  ((A \uparrow  A)\uparrow (B \uparrow  B)\uparrow C\uparrow (D \uparrow  D)) \uparrow  ((A \uparrow  A)\uparrow (B \uparrow  B)\uparrow C\uparrow D) \uparrow  ((A \uparrow  A)\uparrow B\uparrow (C \uparrow  C)\uparrow (D \uparrow  D)) \uparrow  ((A \uparrow  A)\uparrow B\uparrow (C \uparrow  C)\uparrow D) \uparrow  ((A \uparrow  A)\uparrow B\uparrow C\uparrow (D \uparrow  D)) \uparrow  ((A \uparrow  A)\uparrow B\uparrow C\uparrow D) \uparrow  (A\uparrow (B \uparrow  B)\uparrow (C \uparrow  C)\uparrow (D \uparrow  D)) \uparrow  (A\uparrow (B \uparrow  B)\uparrow (C \uparrow  C)\uparrow D) \uparrow  (A\uparrow B\uparrow C\uparrow (D \uparrow  D)))$$$$


\paragraph{Logigramme de la fonction}

 \label{logigram-P10}
\begin{tikzpicture}

 %%Paramaters
%% var position, can be modified
\def\varPos{2.00}
\node (x) at (0, \varPos) {$A$};
\node (y) at (0.5, \varPos) {$B$};
\node (z) at (1, \varPos) {$C$};
\node (w) at (1.5, \varPos) {$D$};

            \node[nand gate US, draw, rotate=270, scale=0.5, logic gate inputs=nn] at ($(x) + (0.25, -0.6)$) (notx) {};
            \draw ($(x)+(0,-1ex)$) -| (notx.input 1); 
            \node[nand gate US, draw, rotate=270, scale=0.5, logic gate inputs=nn] at ($(y) + (0.25, -0.6)$) (noty) {};
            \draw ($(y)+(0,-1ex)$) -| (noty.input 1); 
            \node[nand gate US, draw, rotate=270, scale=0.5, logic gate inputs=nn] at ($(z) + (0.25, -0.6)$) (notz) {};
            \draw ($(z)+(0,-1ex)$) -| (notz.input 1);
            \node[nand gate US, draw, rotate=270, scale=0.5, logic gate inputs=nn] at ($(w) + (0.25, -0.6)$) (notw) {};
            \draw ($(w)+(0,-1ex)$) -| (notw.input 1);
         

      
                \node[nand gate US, draw, rotate=0, logic gate inputs=nnnn] at (2.5, 0.00) (xandy0) {};\draw (xandy0.output) -- node[above]{\scriptsize $ A.B $} ($(xandy0) + (1.8, 0)$);
                % X
\draw ($(x) + (0, -1ex)$)|- (xandy0.input 1);
% Y
\draw ($(y) + (0, -1ex)$)|- (xandy0.input 2);
 

      
                \node[nand gate US, draw, rotate=0, logic gate inputs=nnnn] at (2.5, 1.20) (xandy1) {};\draw (xandy1.output) -- node[above]{\scriptsize $ A.C $} ($(xandy1) + (1.8, 0)$);
                % X
\draw ($(x) + (0, -1ex)$)|- (xandy1.input 1);
% Z
\draw ($(z) + (0, -1ex)$)|- (xandy1.input 3);
\node[nand gate US, draw, rotate=0, logic gate inputs=nnn] at (6, 0.60) (xory0) {};


            \draw (xory0.output) -- node[above]{\scriptsize $P10$} ($(xory0.east) + (+3ex, 0)$);


            \draw (xandy0.output) -- ([xshift=1.60cm]xandy0.output) |- (xory0.input 2);

\draw (xandy1.output) -- ([xshift=1.55cm]xandy1.output) |- (xory0.input 1);

 \end{tikzpicture}


\pagebreak

\paragraph{Q1}


P10(A, B, C, D) = $[10, 11, 12, 13, 15]$


\hrule width 1\linewidth
\pagebreak

\subsection{Correction}


\paragraph{Q1}

P10(A, B, C, D) =$[10, 11, 12, 13, 15]$

P10(A, B, C, D) =$ \sum([10, 11, 12, 13, 15]) $ 




Sum of products 
 P10(A, B, C, D) = $A.\bar B.C.\bar D + A.\bar B.C.D + A.B.\bar C.\bar D + A.B.\bar C.D + A.B.C.D$


Product of sums 
 P10(A, B, C, D) = $(A+B+C+D) . (A+B+C+\bar D) . (A+B+\bar C+D) . (A+B+\bar C+\bar D) . (A+\bar B+C+D) . (A+\bar B+C+\bar D) . (A+\bar B+\bar C+D) . (A+\bar B+\bar C+\bar D) . (\bar A+B+C+D) . (\bar A+B+C+\bar D) . (\bar A+\bar B+\bar C+D)$


\paragraph{Karnough map}

\begin{karnaugh-map}[4][4][1][CD][AB]
          \minterms{10, 11, 12, 13, 15}
          \maxterms{0, 1, 2, 3, 5, 6, 7, 8}
        %\autoterms[0]
         \terms{4, 9, 14}{X}
        % simplification
        \implicant{12}{14}
\implicant{15}{10}
          %\implicant{5}{15}
          %\implicantedge{8}{8}{10}{10}
          %\implicantedge{8}{8}{10}{10}[8,10]
        \end{karnaugh-map}



Simplified Sum of products: $ a.b + a.c $


Simplified Product of sums: $(a).(b+c)$


NAND
 first simplified from: $$$$ a.b + a.c $$
$$\overline{\overline{ a.b + a.c }}$$
$$\overline{(\overline{ a.b }).(\overline{ a.c })}$$
$$( A\uparrow B )\big\uparrow ( A\uparrow C )$$$$


NAND
 second simplified from: $$$$(a).(b+c)$$
$$(\overline{\overline{a}}).(\overline{\overline{b+c}})$$
$$\overline{\overline{(\overline{\overline{a}}).(\overline{\overline{b+c}})}}$$
$$(((A \uparrow  A))\uparrow ((B \uparrow  B)\uparrow (C \uparrow  C)))\uparrow (((A \uparrow  A))\uparrow ((B \uparrow  B)\uparrow (C \uparrow  C)))$$$$


NAND
 first from: $$$$A.\bar B.C.\bar D + A.\bar B.C.D + A.B.\bar C.\bar D + A.B.\bar C.D + A.B.C.D$$
$$\overline{\overline{A.\bar B.C.\bar D + A.\bar B.C.D + A.B.\bar C.\bar D + A.B.\bar C.D + A.B.C.D}}$$
$$\overline{(\overline{A.\bar B.C.\bar D }).(\overline{ A.\bar B.C.D }).(\overline{ A.B.\bar C.\bar D }).(\overline{ A.B.\bar C.D }).(\overline{ A.B.C.D})}$$
$$(A\uparrow (B \uparrow  B)\uparrow C\uparrow (D \uparrow  D) )\big\uparrow ( A\uparrow (B \uparrow  B)\uparrow C\uparrow D )\big\uparrow ( A\uparrow B\uparrow (C \uparrow  C)\uparrow (D \uparrow  D) )\big\uparrow ( A\uparrow B\uparrow (C \uparrow  C)\uparrow D )\big\uparrow ( A\uparrow B\uparrow C\uparrow D)$$$$


NAND
 second from: $$$$(A+B+C+D) . (A+B+C+\bar D) . (A+B+\bar C+D) . (A+B+\bar C+\bar D) . (A+\bar B+C+D) . (A+\bar B+C+\bar D) . (A+\bar B+\bar C+D) . (A+\bar B+\bar C+\bar D) . (\bar A+B+C+D) . (\bar A+B+C+\bar D) . (\bar A+\bar B+\bar C+D)$$
$$(\overline{\overline{A+B+C+D}}) . (\overline{\overline{A+B+C+\bar D}}) . (\overline{\overline{A+B+\bar C+D}}) . (\overline{\overline{A+B+\bar C+\bar D}}) . (\overline{\overline{A+\bar B+C+D}}) . (\overline{\overline{A+\bar B+C+\bar D}}) . (\overline{\overline{A+\bar B+\bar C+D}}) . (\overline{\overline{A+\bar B+\bar C+\bar D}}) . (\overline{\overline{\bar A+B+C+D}}) . (\overline{\overline{\bar A+B+C+\bar D}}) . (\overline{\overline{\bar A+\bar B+\bar C+D}})$$
$$\overline{\overline{(\overline{\overline{A+B+C+D}}) . (\overline{\overline{A+B+C+\bar D}}) . (\overline{\overline{A+B+\bar C+D}}) . (\overline{\overline{A+B+\bar C+\bar D}}) . (\overline{\overline{A+\bar B+C+D}}) . (\overline{\overline{A+\bar B+C+\bar D}}) . (\overline{\overline{A+\bar B+\bar C+D}}) . (\overline{\overline{A+\bar B+\bar C+\bar D}}) . (\overline{\overline{\bar A+B+C+D}}) . (\overline{\overline{\bar A+B+C+\bar D}}) . (\overline{\overline{\bar A+\bar B+\bar C+D}})}}$$
$$(((A \uparrow  A)\uparrow (B \uparrow  B)\uparrow (C \uparrow  C)\uparrow (D \uparrow  D)) \uparrow  ((A \uparrow  A)\uparrow (B \uparrow  B)\uparrow (C \uparrow  C)\uparrow D) \uparrow  ((A \uparrow  A)\uparrow (B \uparrow  B)\uparrow C\uparrow (D \uparrow  D)) \uparrow  ((A \uparrow  A)\uparrow (B \uparrow  B)\uparrow C\uparrow D) \uparrow  ((A \uparrow  A)\uparrow B\uparrow (C \uparrow  C)\uparrow (D \uparrow  D)) \uparrow  ((A \uparrow  A)\uparrow B\uparrow (C \uparrow  C)\uparrow D) \uparrow  ((A \uparrow  A)\uparrow B\uparrow C\uparrow (D \uparrow  D)) \uparrow  ((A \uparrow  A)\uparrow B\uparrow C\uparrow D) \uparrow  (A\uparrow (B \uparrow  B)\uparrow (C \uparrow  C)\uparrow (D \uparrow  D)) \uparrow  (A\uparrow (B \uparrow  B)\uparrow (C \uparrow  C)\uparrow D) \uparrow  (A\uparrow B\uparrow C\uparrow (D \uparrow  D)))\uparrow (((A \uparrow  A)\uparrow (B \uparrow  B)\uparrow (C \uparrow  C)\uparrow (D \uparrow  D)) \uparrow  ((A \uparrow  A)\uparrow (B \uparrow  B)\uparrow (C \uparrow  C)\uparrow D) \uparrow  ((A \uparrow  A)\uparrow (B \uparrow  B)\uparrow C\uparrow (D \uparrow  D)) \uparrow  ((A \uparrow  A)\uparrow (B \uparrow  B)\uparrow C\uparrow D) \uparrow  ((A \uparrow  A)\uparrow B\uparrow (C \uparrow  C)\uparrow (D \uparrow  D)) \uparrow  ((A \uparrow  A)\uparrow B\uparrow (C \uparrow  C)\uparrow D) \uparrow  ((A \uparrow  A)\uparrow B\uparrow C\uparrow (D \uparrow  D)) \uparrow  ((A \uparrow  A)\uparrow B\uparrow C\uparrow D) \uparrow  (A\uparrow (B \uparrow  B)\uparrow (C \uparrow  C)\uparrow (D \uparrow  D)) \uparrow  (A\uparrow (B \uparrow  B)\uparrow (C \uparrow  C)\uparrow D) \uparrow  (A\uparrow B\uparrow C\uparrow (D \uparrow  D)))$$$$


\paragraph{Logigramme de la fonction}

 \label{logigram-P10}
\begin{tikzpicture}

 %%Paramaters
%% var position, can be modified
\def\varPos{2.00}
\node (x) at (0, \varPos) {$A$};
\node (y) at (0.5, \varPos) {$B$};
\node (z) at (1, \varPos) {$C$};
\node (w) at (1.5, \varPos) {$D$};

            \node[nand gate US, draw, rotate=270, scale=0.5, logic gate inputs=nn] at ($(x) + (0.25, -0.6)$) (notx) {};
            \draw ($(x)+(0,-1ex)$) -| (notx.input 1); 
            \node[nand gate US, draw, rotate=270, scale=0.5, logic gate inputs=nn] at ($(y) + (0.25, -0.6)$) (noty) {};
            \draw ($(y)+(0,-1ex)$) -| (noty.input 1); 
            \node[nand gate US, draw, rotate=270, scale=0.5, logic gate inputs=nn] at ($(z) + (0.25, -0.6)$) (notz) {};
            \draw ($(z)+(0,-1ex)$) -| (notz.input 1);
            \node[nand gate US, draw, rotate=270, scale=0.5, logic gate inputs=nn] at ($(w) + (0.25, -0.6)$) (notw) {};
            \draw ($(w)+(0,-1ex)$) -| (notw.input 1);
         

      
                \node[nand gate US, draw, rotate=0, logic gate inputs=nnnn] at (2.5, 0.00) (xandy0) {};\draw (xandy0.output) -- node[above]{\scriptsize $ A.B $} ($(xandy0) + (1.8, 0)$);
                % X
\draw ($(x) + (0, -1ex)$)|- (xandy0.input 1);
% Y
\draw ($(y) + (0, -1ex)$)|- (xandy0.input 2);
 

      
                \node[nand gate US, draw, rotate=0, logic gate inputs=nnnn] at (2.5, 1.20) (xandy1) {};\draw (xandy1.output) -- node[above]{\scriptsize $ A.C $} ($(xandy1) + (1.8, 0)$);
                % X
\draw ($(x) + (0, -1ex)$)|- (xandy1.input 1);
% Z
\draw ($(z) + (0, -1ex)$)|- (xandy1.input 3);
\node[nand gate US, draw, rotate=0, logic gate inputs=nnn] at (6, 0.60) (xory0) {};


            \draw (xory0.output) -- node[above]{\scriptsize $P10$} ($(xory0.east) + (+3ex, 0)$);


            \draw (xandy0.output) -- ([xshift=1.60cm]xandy0.output) |- (xory0.input 2);

\draw (xandy1.output) -- ([xshift=1.55cm]xandy1.output) |- (xory0.input 1);

 \end{tikzpicture}


\pagebreak

\paragraph{Q1}


P10(A, B, C, D) = $[10, 11, 12, 13, 15]$


\hrule width 1\linewidth
\pagebreak

\subsection{Correction}


\paragraph{Q1}

P10(A, B, C, D) =$[10, 11, 12, 13, 15]$

P10(A, B, C, D) =$ \sum([10, 11, 12, 13, 15]) $ 




Sum of products 
 P10(A, B, C, D) = $A.\bar B.C.\bar D + A.\bar B.C.D + A.B.\bar C.\bar D + A.B.\bar C.D + A.B.C.D$


Product of sums 
 P10(A, B, C, D) = $(A+B+C+D) . (A+B+C+\bar D) . (A+B+\bar C+D) . (A+B+\bar C+\bar D) . (A+\bar B+C+D) . (A+\bar B+C+\bar D) . (A+\bar B+\bar C+D) . (A+\bar B+\bar C+\bar D) . (\bar A+B+C+D) . (\bar A+B+C+\bar D) . (\bar A+\bar B+\bar C+D)$


\paragraph{Karnough map}

\begin{karnaugh-map}[4][4][1][CD][AB]
          \minterms{10, 11, 12, 13, 15}
          \maxterms{0, 1, 2, 3, 5, 6, 7, 8}
        %\autoterms[0]
         \terms{4, 9, 14}{X}
        % simplification
        \implicant{12}{14}
\implicant{15}{10}
          %\implicant{5}{15}
          %\implicantedge{8}{8}{10}{10}
          %\implicantedge{8}{8}{10}{10}[8,10]
        \end{karnaugh-map}



Simplified Sum of products: $ a.b + a.c $


Simplified Product of sums: $(a).(b+c)$


NAND
 first simplified from: $$$$ a.b + a.c $$
$$\overline{\overline{ a.b + a.c }}$$
$$\overline{(\overline{ a.b }).(\overline{ a.c })}$$
$$( A\uparrow B )\big\uparrow ( A\uparrow C )$$$$


NAND
 second simplified from: $$$$(a).(b+c)$$
$$(\overline{\overline{a}}).(\overline{\overline{b+c}})$$
$$\overline{\overline{(\overline{\overline{a}}).(\overline{\overline{b+c}})}}$$
$$(((A \uparrow  A))\uparrow ((B \uparrow  B)\uparrow (C \uparrow  C)))\uparrow (((A \uparrow  A))\uparrow ((B \uparrow  B)\uparrow (C \uparrow  C)))$$$$


NAND
 first from: $$$$A.\bar B.C.\bar D + A.\bar B.C.D + A.B.\bar C.\bar D + A.B.\bar C.D + A.B.C.D$$
$$\overline{\overline{A.\bar B.C.\bar D + A.\bar B.C.D + A.B.\bar C.\bar D + A.B.\bar C.D + A.B.C.D}}$$
$$\overline{(\overline{A.\bar B.C.\bar D }).(\overline{ A.\bar B.C.D }).(\overline{ A.B.\bar C.\bar D }).(\overline{ A.B.\bar C.D }).(\overline{ A.B.C.D})}$$
$$(A\uparrow (B \uparrow  B)\uparrow C\uparrow (D \uparrow  D) )\big\uparrow ( A\uparrow (B \uparrow  B)\uparrow C\uparrow D )\big\uparrow ( A\uparrow B\uparrow (C \uparrow  C)\uparrow (D \uparrow  D) )\big\uparrow ( A\uparrow B\uparrow (C \uparrow  C)\uparrow D )\big\uparrow ( A\uparrow B\uparrow C\uparrow D)$$$$


NAND
 second from: $$$$(A+B+C+D) . (A+B+C+\bar D) . (A+B+\bar C+D) . (A+B+\bar C+\bar D) . (A+\bar B+C+D) . (A+\bar B+C+\bar D) . (A+\bar B+\bar C+D) . (A+\bar B+\bar C+\bar D) . (\bar A+B+C+D) . (\bar A+B+C+\bar D) . (\bar A+\bar B+\bar C+D)$$
$$(\overline{\overline{A+B+C+D}}) . (\overline{\overline{A+B+C+\bar D}}) . (\overline{\overline{A+B+\bar C+D}}) . (\overline{\overline{A+B+\bar C+\bar D}}) . (\overline{\overline{A+\bar B+C+D}}) . (\overline{\overline{A+\bar B+C+\bar D}}) . (\overline{\overline{A+\bar B+\bar C+D}}) . (\overline{\overline{A+\bar B+\bar C+\bar D}}) . (\overline{\overline{\bar A+B+C+D}}) . (\overline{\overline{\bar A+B+C+\bar D}}) . (\overline{\overline{\bar A+\bar B+\bar C+D}})$$
$$\overline{\overline{(\overline{\overline{A+B+C+D}}) . (\overline{\overline{A+B+C+\bar D}}) . (\overline{\overline{A+B+\bar C+D}}) . (\overline{\overline{A+B+\bar C+\bar D}}) . (\overline{\overline{A+\bar B+C+D}}) . (\overline{\overline{A+\bar B+C+\bar D}}) . (\overline{\overline{A+\bar B+\bar C+D}}) . (\overline{\overline{A+\bar B+\bar C+\bar D}}) . (\overline{\overline{\bar A+B+C+D}}) . (\overline{\overline{\bar A+B+C+\bar D}}) . (\overline{\overline{\bar A+\bar B+\bar C+D}})}}$$
$$(((A \uparrow  A)\uparrow (B \uparrow  B)\uparrow (C \uparrow  C)\uparrow (D \uparrow  D)) \uparrow  ((A \uparrow  A)\uparrow (B \uparrow  B)\uparrow (C \uparrow  C)\uparrow D) \uparrow  ((A \uparrow  A)\uparrow (B \uparrow  B)\uparrow C\uparrow (D \uparrow  D)) \uparrow  ((A \uparrow  A)\uparrow (B \uparrow  B)\uparrow C\uparrow D) \uparrow  ((A \uparrow  A)\uparrow B\uparrow (C \uparrow  C)\uparrow (D \uparrow  D)) \uparrow  ((A \uparrow  A)\uparrow B\uparrow (C \uparrow  C)\uparrow D) \uparrow  ((A \uparrow  A)\uparrow B\uparrow C\uparrow (D \uparrow  D)) \uparrow  ((A \uparrow  A)\uparrow B\uparrow C\uparrow D) \uparrow  (A\uparrow (B \uparrow  B)\uparrow (C \uparrow  C)\uparrow (D \uparrow  D)) \uparrow  (A\uparrow (B \uparrow  B)\uparrow (C \uparrow  C)\uparrow D) \uparrow  (A\uparrow B\uparrow C\uparrow (D \uparrow  D)))\uparrow (((A \uparrow  A)\uparrow (B \uparrow  B)\uparrow (C \uparrow  C)\uparrow (D \uparrow  D)) \uparrow  ((A \uparrow  A)\uparrow (B \uparrow  B)\uparrow (C \uparrow  C)\uparrow D) \uparrow  ((A \uparrow  A)\uparrow (B \uparrow  B)\uparrow C\uparrow (D \uparrow  D)) \uparrow  ((A \uparrow  A)\uparrow (B \uparrow  B)\uparrow C\uparrow D) \uparrow  ((A \uparrow  A)\uparrow B\uparrow (C \uparrow  C)\uparrow (D \uparrow  D)) \uparrow  ((A \uparrow  A)\uparrow B\uparrow (C \uparrow  C)\uparrow D) \uparrow  ((A \uparrow  A)\uparrow B\uparrow C\uparrow (D \uparrow  D)) \uparrow  ((A \uparrow  A)\uparrow B\uparrow C\uparrow D) \uparrow  (A\uparrow (B \uparrow  B)\uparrow (C \uparrow  C)\uparrow (D \uparrow  D)) \uparrow  (A\uparrow (B \uparrow  B)\uparrow (C \uparrow  C)\uparrow D) \uparrow  (A\uparrow B\uparrow C\uparrow (D \uparrow  D)))$$$$


\paragraph{Logigramme de la fonction}

 \label{logigram-P10}
\begin{tikzpicture}

 %%Paramaters
%% var position, can be modified
\def\varPos{2.00}
\node (x) at (0, \varPos) {$A$};
\node (y) at (0.5, \varPos) {$B$};
\node (z) at (1, \varPos) {$C$};
\node (w) at (1.5, \varPos) {$D$};

            \node[nand gate US, draw, rotate=270, scale=0.5, logic gate inputs=nn] at ($(x) + (0.25, -0.6)$) (notx) {};
            \draw ($(x)+(0,-1ex)$) -| (notx.input 1); 
            \node[nand gate US, draw, rotate=270, scale=0.5, logic gate inputs=nn] at ($(y) + (0.25, -0.6)$) (noty) {};
            \draw ($(y)+(0,-1ex)$) -| (noty.input 1); 
            \node[nand gate US, draw, rotate=270, scale=0.5, logic gate inputs=nn] at ($(z) + (0.25, -0.6)$) (notz) {};
            \draw ($(z)+(0,-1ex)$) -| (notz.input 1);
            \node[nand gate US, draw, rotate=270, scale=0.5, logic gate inputs=nn] at ($(w) + (0.25, -0.6)$) (notw) {};
            \draw ($(w)+(0,-1ex)$) -| (notw.input 1);
         

      
                \node[nand gate US, draw, rotate=0, logic gate inputs=nnnn] at (2.5, 0.00) (xandy0) {};\draw (xandy0.output) -- node[above]{\scriptsize $ A.B $} ($(xandy0) + (1.8, 0)$);
                % X
\draw ($(x) + (0, -1ex)$)|- (xandy0.input 1);
% Y
\draw ($(y) + (0, -1ex)$)|- (xandy0.input 2);
 

      
                \node[nand gate US, draw, rotate=0, logic gate inputs=nnnn] at (2.5, 1.20) (xandy1) {};\draw (xandy1.output) -- node[above]{\scriptsize $ A.C $} ($(xandy1) + (1.8, 0)$);
                % X
\draw ($(x) + (0, -1ex)$)|- (xandy1.input 1);
% Z
\draw ($(z) + (0, -1ex)$)|- (xandy1.input 3);
\node[nand gate US, draw, rotate=0, logic gate inputs=nnn] at (6, 0.60) (xory0) {};


            \draw (xory0.output) -- node[above]{\scriptsize $P10$} ($(xory0.east) + (+3ex, 0)$);


            \draw (xandy0.output) -- ([xshift=1.60cm]xandy0.output) |- (xory0.input 2);

\draw (xandy1.output) -- ([xshift=1.55cm]xandy1.output) |- (xory0.input 1);

 \end{tikzpicture}


\pagebreak

\paragraph{Q1}


P10(A, B, C, D) = $[10, 11, 12, 13, 15]$


\hrule width 1\linewidth
\pagebreak

\subsection{Correction}


\paragraph{Q1}

P10(A, B, C, D) =$[10, 11, 12, 13, 15]$

P10(A, B, C, D) =$ \sum([10, 11, 12, 13, 15]) $ 




Sum of products 
 P10(A, B, C, D) = $A.\bar B.C.\bar D + A.\bar B.C.D + A.B.\bar C.\bar D + A.B.\bar C.D + A.B.C.D$


Product of sums 
 P10(A, B, C, D) = $(A+B+C+D) . (A+B+C+\bar D) . (A+B+\bar C+D) . (A+B+\bar C+\bar D) . (A+\bar B+C+D) . (A+\bar B+C+\bar D) . (A+\bar B+\bar C+D) . (A+\bar B+\bar C+\bar D) . (\bar A+B+C+D) . (\bar A+B+C+\bar D) . (\bar A+\bar B+\bar C+D)$


\paragraph{Karnough map}

\begin{karnaugh-map}[4][4][1][CD][AB]
          \minterms{10, 11, 12, 13, 15}
          \maxterms{0, 1, 2, 3, 5, 6, 7, 8}
        %\autoterms[0]
         \terms{4, 9, 14}{X}
        % simplification
        \implicant{12}{14}
\implicant{15}{10}
          %\implicant{5}{15}
          %\implicantedge{8}{8}{10}{10}
          %\implicantedge{8}{8}{10}{10}[8,10]
        \end{karnaugh-map}



Simplified Sum of products: $ a.b + a.c $


Simplified Product of sums: $(a).(b+c)$


NAND
 first simplified from: $$$$ a.b + a.c $$
$$\overline{\overline{ a.b + a.c }}$$
$$\overline{(\overline{ a.b }).(\overline{ a.c })}$$
$$( A\uparrow B )\big\uparrow ( A\uparrow C )$$$$


NAND
 second simplified from: $$$$(a).(b+c)$$
$$(\overline{\overline{a}}).(\overline{\overline{b+c}})$$
$$\overline{\overline{(\overline{\overline{a}}).(\overline{\overline{b+c}})}}$$
$$(((A \uparrow  A))\uparrow ((B \uparrow  B)\uparrow (C \uparrow  C)))\uparrow (((A \uparrow  A))\uparrow ((B \uparrow  B)\uparrow (C \uparrow  C)))$$$$


NAND
 first from: $$$$A.\bar B.C.\bar D + A.\bar B.C.D + A.B.\bar C.\bar D + A.B.\bar C.D + A.B.C.D$$
$$\overline{\overline{A.\bar B.C.\bar D + A.\bar B.C.D + A.B.\bar C.\bar D + A.B.\bar C.D + A.B.C.D}}$$
$$\overline{(\overline{A.\bar B.C.\bar D }).(\overline{ A.\bar B.C.D }).(\overline{ A.B.\bar C.\bar D }).(\overline{ A.B.\bar C.D }).(\overline{ A.B.C.D})}$$
$$(A\uparrow (B \uparrow  B)\uparrow C\uparrow (D \uparrow  D) )\big\uparrow ( A\uparrow (B \uparrow  B)\uparrow C\uparrow D )\big\uparrow ( A\uparrow B\uparrow (C \uparrow  C)\uparrow (D \uparrow  D) )\big\uparrow ( A\uparrow B\uparrow (C \uparrow  C)\uparrow D )\big\uparrow ( A\uparrow B\uparrow C\uparrow D)$$$$


NAND
 second from: $$$$(A+B+C+D) . (A+B+C+\bar D) . (A+B+\bar C+D) . (A+B+\bar C+\bar D) . (A+\bar B+C+D) . (A+\bar B+C+\bar D) . (A+\bar B+\bar C+D) . (A+\bar B+\bar C+\bar D) . (\bar A+B+C+D) . (\bar A+B+C+\bar D) . (\bar A+\bar B+\bar C+D)$$
$$(\overline{\overline{A+B+C+D}}) . (\overline{\overline{A+B+C+\bar D}}) . (\overline{\overline{A+B+\bar C+D}}) . (\overline{\overline{A+B+\bar C+\bar D}}) . (\overline{\overline{A+\bar B+C+D}}) . (\overline{\overline{A+\bar B+C+\bar D}}) . (\overline{\overline{A+\bar B+\bar C+D}}) . (\overline{\overline{A+\bar B+\bar C+\bar D}}) . (\overline{\overline{\bar A+B+C+D}}) . (\overline{\overline{\bar A+B+C+\bar D}}) . (\overline{\overline{\bar A+\bar B+\bar C+D}})$$
$$\overline{\overline{(\overline{\overline{A+B+C+D}}) . (\overline{\overline{A+B+C+\bar D}}) . (\overline{\overline{A+B+\bar C+D}}) . (\overline{\overline{A+B+\bar C+\bar D}}) . (\overline{\overline{A+\bar B+C+D}}) . (\overline{\overline{A+\bar B+C+\bar D}}) . (\overline{\overline{A+\bar B+\bar C+D}}) . (\overline{\overline{A+\bar B+\bar C+\bar D}}) . (\overline{\overline{\bar A+B+C+D}}) . (\overline{\overline{\bar A+B+C+\bar D}}) . (\overline{\overline{\bar A+\bar B+\bar C+D}})}}$$
$$(((A \uparrow  A)\uparrow (B \uparrow  B)\uparrow (C \uparrow  C)\uparrow (D \uparrow  D)) \uparrow  ((A \uparrow  A)\uparrow (B \uparrow  B)\uparrow (C \uparrow  C)\uparrow D) \uparrow  ((A \uparrow  A)\uparrow (B \uparrow  B)\uparrow C\uparrow (D \uparrow  D)) \uparrow  ((A \uparrow  A)\uparrow (B \uparrow  B)\uparrow C\uparrow D) \uparrow  ((A \uparrow  A)\uparrow B\uparrow (C \uparrow  C)\uparrow (D \uparrow  D)) \uparrow  ((A \uparrow  A)\uparrow B\uparrow (C \uparrow  C)\uparrow D) \uparrow  ((A \uparrow  A)\uparrow B\uparrow C\uparrow (D \uparrow  D)) \uparrow  ((A \uparrow  A)\uparrow B\uparrow C\uparrow D) \uparrow  (A\uparrow (B \uparrow  B)\uparrow (C \uparrow  C)\uparrow (D \uparrow  D)) \uparrow  (A\uparrow (B \uparrow  B)\uparrow (C \uparrow  C)\uparrow D) \uparrow  (A\uparrow B\uparrow C\uparrow (D \uparrow  D)))\uparrow (((A \uparrow  A)\uparrow (B \uparrow  B)\uparrow (C \uparrow  C)\uparrow (D \uparrow  D)) \uparrow  ((A \uparrow  A)\uparrow (B \uparrow  B)\uparrow (C \uparrow  C)\uparrow D) \uparrow  ((A \uparrow  A)\uparrow (B \uparrow  B)\uparrow C\uparrow (D \uparrow  D)) \uparrow  ((A \uparrow  A)\uparrow (B \uparrow  B)\uparrow C\uparrow D) \uparrow  ((A \uparrow  A)\uparrow B\uparrow (C \uparrow  C)\uparrow (D \uparrow  D)) \uparrow  ((A \uparrow  A)\uparrow B\uparrow (C \uparrow  C)\uparrow D) \uparrow  ((A \uparrow  A)\uparrow B\uparrow C\uparrow (D \uparrow  D)) \uparrow  ((A \uparrow  A)\uparrow B\uparrow C\uparrow D) \uparrow  (A\uparrow (B \uparrow  B)\uparrow (C \uparrow  C)\uparrow (D \uparrow  D)) \uparrow  (A\uparrow (B \uparrow  B)\uparrow (C \uparrow  C)\uparrow D) \uparrow  (A\uparrow B\uparrow C\uparrow (D \uparrow  D)))$$$$


\paragraph{Logigramme de la fonction}

 \label{logigram-P10}
\begin{tikzpicture}

 %%Paramaters
%% var position, can be modified
\def\varPos{2.00}
\node (x) at (0, \varPos) {$A$};
\node (y) at (0.5, \varPos) {$B$};
\node (z) at (1, \varPos) {$C$};
\node (w) at (1.5, \varPos) {$D$};

            \node[nand gate US, draw, rotate=270, scale=0.5, logic gate inputs=nn] at ($(x) + (0.25, -0.6)$) (notx) {};
            \draw ($(x)+(0,-1ex)$) -| (notx.input 1); 
            \node[nand gate US, draw, rotate=270, scale=0.5, logic gate inputs=nn] at ($(y) + (0.25, -0.6)$) (noty) {};
            \draw ($(y)+(0,-1ex)$) -| (noty.input 1); 
            \node[nand gate US, draw, rotate=270, scale=0.5, logic gate inputs=nn] at ($(z) + (0.25, -0.6)$) (notz) {};
            \draw ($(z)+(0,-1ex)$) -| (notz.input 1);
            \node[nand gate US, draw, rotate=270, scale=0.5, logic gate inputs=nn] at ($(w) + (0.25, -0.6)$) (notw) {};
            \draw ($(w)+(0,-1ex)$) -| (notw.input 1);
         

      
                \node[nand gate US, draw, rotate=0, logic gate inputs=nnnn] at (2.5, 0.00) (xandy0) {};\draw (xandy0.output) -- node[above]{\scriptsize $ A.B $} ($(xandy0) + (1.8, 0)$);
                % X
\draw ($(x) + (0, -1ex)$)|- (xandy0.input 1);
% Y
\draw ($(y) + (0, -1ex)$)|- (xandy0.input 2);
 

      
                \node[nand gate US, draw, rotate=0, logic gate inputs=nnnn] at (2.5, 1.20) (xandy1) {};\draw (xandy1.output) -- node[above]{\scriptsize $ A.C $} ($(xandy1) + (1.8, 0)$);
                % X
\draw ($(x) + (0, -1ex)$)|- (xandy1.input 1);
% Z
\draw ($(z) + (0, -1ex)$)|- (xandy1.input 3);
\node[nand gate US, draw, rotate=0, logic gate inputs=nnn] at (6, 0.60) (xory0) {};


            \draw (xory0.output) -- node[above]{\scriptsize $P10$} ($(xory0.east) + (+3ex, 0)$);


            \draw (xandy0.output) -- ([xshift=1.60cm]xandy0.output) |- (xory0.input 2);

\draw (xandy1.output) -- ([xshift=1.55cm]xandy1.output) |- (xory0.input 1);

 \end{tikzpicture}


\pagebreak

\paragraph{Q1}


P10(A, B, C, D) = $[10, 11, 12, 13, 15]$


\hrule width 1\linewidth
\pagebreak

\subsection{Correction}


\paragraph{Q1}

P10(A, B, C, D) =$[10, 11, 12, 13, 15]$

P10(A, B, C, D) =$ \sum([10, 11, 12, 13, 15]) $ 




Sum of products 
 P10(A, B, C, D) = $A.\bar B.C.\bar D + A.\bar B.C.D + A.B.\bar C.\bar D + A.B.\bar C.D + A.B.C.D$


Product of sums 
 P10(A, B, C, D) = $(A+B+C+D) . (A+B+C+\bar D) . (A+B+\bar C+D) . (A+B+\bar C+\bar D) . (A+\bar B+C+D) . (A+\bar B+C+\bar D) . (A+\bar B+\bar C+D) . (A+\bar B+\bar C+\bar D) . (\bar A+B+C+D) . (\bar A+B+C+\bar D) . (\bar A+\bar B+\bar C+D)$


\paragraph{Karnough map}

\begin{karnaugh-map}[4][4][1][CD][AB]
          \minterms{10, 11, 12, 13, 15}
          \maxterms{0, 1, 2, 3, 5, 6, 7, 8}
        %\autoterms[0]
         \terms{4, 9, 14}{X}
        % simplification
        \implicant{12}{14}
\implicant{15}{10}
          %\implicant{5}{15}
          %\implicantedge{8}{8}{10}{10}
          %\implicantedge{8}{8}{10}{10}[8,10]
        \end{karnaugh-map}



Simplified Sum of products: $ a.b + a.c $


Simplified Product of sums: $(a).(b+c)$


NOR
 first simplified from: $$$$ a.b + a.c $$
$$(\overline{\overline{ a.b }})+(\overline{\overline{ a.c }})$$
$$\overline{\overline{(\overline{\overline{ a.b }})+(\overline{\overline{ a.c }})}}$$
$$( (A \downarrow  A)\downarrow (B \downarrow  B) )\downarrow ( (A \downarrow  A)\downarrow (C \downarrow  C) )\downarrow ( (A \downarrow  A)\downarrow (B \downarrow  B) )\downarrow ( (A \downarrow  A)\downarrow (C \downarrow  C) )$$$$


NOR
 second simplified from: $$$$(a).(b+c)$$
$$\overline{\overline{(a).(b+c)}}$$
$$\overline{(\overline{(a)}+\overline{(b+c)})}$$
$$((A)\downarrow (B\downarrow C))$$$$


NOR
 first from: $$$$A.\bar B.C.\bar D + A.\bar B.C.D + A.B.\bar C.\bar D + A.B.\bar C.D + A.B.C.D$$
$$(\overline{\overline{A.\bar B.C.\bar D }})+(\overline{\overline{ A.\bar B.C.D }})+(\overline{\overline{ A.B.\bar C.\bar D }})+(\overline{\overline{ A.B.\bar C.D }})+(\overline{\overline{ A.B.C.D}})$$
$$\overline{\overline{(\overline{\overline{A.\bar B.C.\bar D }})+(\overline{\overline{ A.\bar B.C.D }})+(\overline{\overline{ A.B.\bar C.\bar D }})+(\overline{\overline{ A.B.\bar C.D }})+(\overline{\overline{ A.B.C.D}})}}$$
$$((A \downarrow  A)\downarrow B\downarrow (C \downarrow  C)\downarrow D )\downarrow ( (A \downarrow  A)\downarrow B\downarrow (C \downarrow  C)\downarrow (D \downarrow  D) )\downarrow ( (A \downarrow  A)\downarrow (B \downarrow  B)\downarrow C\downarrow D )\downarrow ( (A \downarrow  A)\downarrow (B \downarrow  B)\downarrow C\downarrow (D \downarrow  D) )\downarrow ( (A \downarrow  A)\downarrow (B \downarrow  B)\downarrow (C \downarrow  C)\downarrow (D \downarrow  D))\downarrow ((A \downarrow  A)\downarrow B\downarrow (C \downarrow  C)\downarrow D )\downarrow ( (A \downarrow  A)\downarrow B\downarrow (C \downarrow  C)\downarrow (D \downarrow  D) )\downarrow ( (A \downarrow  A)\downarrow (B \downarrow  B)\downarrow C\downarrow D )\downarrow ( (A \downarrow  A)\downarrow (B \downarrow  B)\downarrow C\downarrow (D \downarrow  D) )\downarrow ( (A \downarrow  A)\downarrow (B \downarrow  B)\downarrow (C \downarrow  C)\downarrow (D \downarrow  D))$$$$


NOR
 second from: $$$$(A+B+C+D) . (A+B+C+\bar D) . (A+B+\bar C+D) . (A+B+\bar C+\bar D) . (A+\bar B+C+D) . (A+\bar B+C+\bar D) . (A+\bar B+\bar C+D) . (A+\bar B+\bar C+\bar D) . (\bar A+B+C+D) . (\bar A+B+C+\bar D) . (\bar A+\bar B+\bar C+D)$$
$$\overline{\overline{(A+B+C+D) . (A+B+C+\bar D) . (A+B+\bar C+D) . (A+B+\bar C+\bar D) . (A+\bar B+C+D) . (A+\bar B+C+\bar D) . (A+\bar B+\bar C+D) . (A+\bar B+\bar C+\bar D) . (\bar A+B+C+D) . (\bar A+B+C+\bar D) . (\bar A+\bar B+\bar C+D)}}$$
$$\overline{(\overline{(A+B+C+D) }+\overline{ (A+B+C+\bar D) }+\overline{ (A+B+\bar C+D) }+\overline{ (A+B+\bar C+\bar D) }+\overline{ (A+\bar B+C+D) }+\overline{ (A+\bar B+C+\bar D) }+\overline{ (A+\bar B+\bar C+D) }+\overline{ (A+\bar B+\bar C+\bar D) }+\overline{ (\bar A+B+C+D) }+\overline{ (\bar A+B+C+\bar D) }+\overline{ (\bar A+\bar B+\bar C+D)})}$$
$$((A\downarrow B\downarrow C\downarrow D) \downarrow  (A\downarrow B\downarrow C\downarrow (D \downarrow  D)) \downarrow  (A\downarrow B\downarrow (C \downarrow  C)\downarrow D) \downarrow  (A\downarrow B\downarrow (C \downarrow  C)\downarrow (D \downarrow  D)) \downarrow  (A\downarrow (B \downarrow  B)\downarrow C\downarrow D) \downarrow  (A\downarrow (B \downarrow  B)\downarrow C\downarrow (D \downarrow  D)) \downarrow  (A\downarrow (B \downarrow  B)\downarrow (C \downarrow  C)\downarrow D) \downarrow  (A\downarrow (B \downarrow  B)\downarrow (C \downarrow  C)\downarrow (D \downarrow  D)) \downarrow  ((A \downarrow  A)\downarrow B\downarrow C\downarrow D) \downarrow  ((A \downarrow  A)\downarrow B\downarrow C\downarrow (D \downarrow  D)) \downarrow  ((A \downarrow  A)\downarrow (B \downarrow  B)\downarrow (C \downarrow  C)\downarrow D))$$$$


\paragraph{Logigramme de la fonction}

 \label{logigram-P10}
\begin{tikzpicture}

 %%Paramaters
%% var position, can be modified
\def\varPos{2.00}
\node (x) at (0, \varPos) {$A$};
\node (y) at (0.5, \varPos) {$B$};
\node (z) at (1, \varPos) {$C$};
\node (w) at (1.5, \varPos) {$D$};

            \node[nor gate US, draw, rotate=270, scale=0.5, logic gate inputs=nn] at ($(x) + (0.25, -0.6)$) (notx) {};
            \draw ($(x)+(0,-1ex)$) -| (notx.input 1); 
            \node[nor gate US, draw, rotate=270, scale=0.5, logic gate inputs=nn] at ($(y) + (0.25, -0.6)$) (noty) {};
            \draw ($(y)+(0,-1ex)$) -| (noty.input 1); 
            \node[nor gate US, draw, rotate=270, scale=0.5, logic gate inputs=nn] at ($(z) + (0.25, -0.6)$) (notz) {};
            \draw ($(z)+(0,-1ex)$) -| (notz.input 1);
            \node[nor gate US, draw, rotate=270, scale=0.5, logic gate inputs=nn] at ($(w) + (0.25, -0.6)$) (notw) {};
            \draw ($(w)+(0,-1ex)$) -| (notw.input 1);
         

      
                \node[nor gate US, draw, rotate=0, logic gate inputs=nnnn] at (2.5, 0.00) (xandy0) {};\draw (xandy0.output) -- node[above]{\scriptsize $ A.B $} ($(xandy0) + (1.8, 0)$);
                % X
\draw ($(x) + (0, -1ex)$)|- (xandy0.input 1);
% Y
\draw ($(y) + (0, -1ex)$)|- (xandy0.input 2);
 

      
                \node[nor gate US, draw, rotate=0, logic gate inputs=nnnn] at (2.5, 1.20) (xandy1) {};\draw (xandy1.output) -- node[above]{\scriptsize $ A.C $} ($(xandy1) + (1.8, 0)$);
                % X
\draw ($(x) + (0, -1ex)$)|- (xandy1.input 1);
% Z
\draw ($(z) + (0, -1ex)$)|- (xandy1.input 3);
\node[nor gate US, draw, rotate=0, logic gate inputs=nnn] at (6, 0.60) (xory0) {};


            \draw (xory0.output) -- node[above]{\scriptsize $P10$} ($(xory0.east) + (+3ex, 0)$);


            \draw (xandy0.output) -- ([xshift=1.60cm]xandy0.output) |- (xory0.input 2);

\draw (xandy1.output) -- ([xshift=1.55cm]xandy1.output) |- (xory0.input 1);

 \end{tikzpicture}


\pagebreak

\paragraph{Q1}


P10(A, B, C, D) = $[10, 11, 12, 13, 15]$


\hrule width 1\linewidth
\pagebreak

\subsection{Correction}


\paragraph{Q1}

P10(A, B, C, D) =$[10, 11, 12, 13, 15]$

P10(A, B, C, D) =$ \sum([10, 11, 12, 13, 15]) $ 




Sum of products 
 P10(A, B, C, D) = $A.\bar B.C.\bar D + A.\bar B.C.D + A.B.\bar C.\bar D + A.B.\bar C.D + A.B.C.D$


Product of sums 
 P10(A, B, C, D) = $(A+B+C+D) . (A+B+C+\bar D) . (A+B+\bar C+D) . (A+B+\bar C+\bar D) . (A+\bar B+C+D) . (A+\bar B+C+\bar D) . (A+\bar B+\bar C+D) . (A+\bar B+\bar C+\bar D) . (\bar A+B+C+D) . (\bar A+B+C+\bar D) . (\bar A+\bar B+\bar C+D)$


\paragraph{Karnough map}

\begin{karnaugh-map}[4][4][1][CD][AB]
          \minterms{10, 11, 12, 13, 15}
          \maxterms{0, 1, 2, 3, 5, 6, 7, 8}
        %\autoterms[0]
         \terms{4, 9, 14}{X}
        % simplification
        \implicant{12}{14}
\implicant{15}{10}
          %\implicant{5}{15}
          %\implicantedge{8}{8}{10}{10}
          %\implicantedge{8}{8}{10}{10}[8,10]
        \end{karnaugh-map}



Simplified Sum of products: $ a.b + a.c $


Simplified Product of sums: $(a).(b+c)$


NOR
 first simplified from: $$$$ a.b + a.c $$
$$(\overline{\overline{ a.b }})+(\overline{\overline{ a.c }})$$
$$\overline{\overline{(\overline{\overline{ a.b }})+(\overline{\overline{ a.c }})}}$$
$$( (A \downarrow  A)\downarrow (B \downarrow  B) )\downarrow ( (A \downarrow  A)\downarrow (C \downarrow  C) )\downarrow ( (A \downarrow  A)\downarrow (B \downarrow  B) )\downarrow ( (A \downarrow  A)\downarrow (C \downarrow  C) )$$$$


NOR
 second simplified from: $$$$(a).(b+c)$$
$$\overline{\overline{(a).(b+c)}}$$
$$\overline{(\overline{(a)}+\overline{(b+c)})}$$
$$((A)\downarrow (B\downarrow C))$$$$


NOR
 first from: $$$$A.\bar B.C.\bar D + A.\bar B.C.D + A.B.\bar C.\bar D + A.B.\bar C.D + A.B.C.D$$
$$(\overline{\overline{A.\bar B.C.\bar D }})+(\overline{\overline{ A.\bar B.C.D }})+(\overline{\overline{ A.B.\bar C.\bar D }})+(\overline{\overline{ A.B.\bar C.D }})+(\overline{\overline{ A.B.C.D}})$$
$$\overline{\overline{(\overline{\overline{A.\bar B.C.\bar D }})+(\overline{\overline{ A.\bar B.C.D }})+(\overline{\overline{ A.B.\bar C.\bar D }})+(\overline{\overline{ A.B.\bar C.D }})+(\overline{\overline{ A.B.C.D}})}}$$
$$((A \downarrow  A)\downarrow B\downarrow (C \downarrow  C)\downarrow D )\downarrow ( (A \downarrow  A)\downarrow B\downarrow (C \downarrow  C)\downarrow (D \downarrow  D) )\downarrow ( (A \downarrow  A)\downarrow (B \downarrow  B)\downarrow C\downarrow D )\downarrow ( (A \downarrow  A)\downarrow (B \downarrow  B)\downarrow C\downarrow (D \downarrow  D) )\downarrow ( (A \downarrow  A)\downarrow (B \downarrow  B)\downarrow (C \downarrow  C)\downarrow (D \downarrow  D))\downarrow ((A \downarrow  A)\downarrow B\downarrow (C \downarrow  C)\downarrow D )\downarrow ( (A \downarrow  A)\downarrow B\downarrow (C \downarrow  C)\downarrow (D \downarrow  D) )\downarrow ( (A \downarrow  A)\downarrow (B \downarrow  B)\downarrow C\downarrow D )\downarrow ( (A \downarrow  A)\downarrow (B \downarrow  B)\downarrow C\downarrow (D \downarrow  D) )\downarrow ( (A \downarrow  A)\downarrow (B \downarrow  B)\downarrow (C \downarrow  C)\downarrow (D \downarrow  D))$$$$


NOR
 second from: $$$$(A+B+C+D) . (A+B+C+\bar D) . (A+B+\bar C+D) . (A+B+\bar C+\bar D) . (A+\bar B+C+D) . (A+\bar B+C+\bar D) . (A+\bar B+\bar C+D) . (A+\bar B+\bar C+\bar D) . (\bar A+B+C+D) . (\bar A+B+C+\bar D) . (\bar A+\bar B+\bar C+D)$$
$$\overline{\overline{(A+B+C+D) . (A+B+C+\bar D) . (A+B+\bar C+D) . (A+B+\bar C+\bar D) . (A+\bar B+C+D) . (A+\bar B+C+\bar D) . (A+\bar B+\bar C+D) . (A+\bar B+\bar C+\bar D) . (\bar A+B+C+D) . (\bar A+B+C+\bar D) . (\bar A+\bar B+\bar C+D)}}$$
$$\overline{(\overline{(A+B+C+D) }+\overline{ (A+B+C+\bar D) }+\overline{ (A+B+\bar C+D) }+\overline{ (A+B+\bar C+\bar D) }+\overline{ (A+\bar B+C+D) }+\overline{ (A+\bar B+C+\bar D) }+\overline{ (A+\bar B+\bar C+D) }+\overline{ (A+\bar B+\bar C+\bar D) }+\overline{ (\bar A+B+C+D) }+\overline{ (\bar A+B+C+\bar D) }+\overline{ (\bar A+\bar B+\bar C+D)})}$$
$$((A\downarrow B\downarrow C\downarrow D) \downarrow  (A\downarrow B\downarrow C\downarrow (D \downarrow  D)) \downarrow  (A\downarrow B\downarrow (C \downarrow  C)\downarrow D) \downarrow  (A\downarrow B\downarrow (C \downarrow  C)\downarrow (D \downarrow  D)) \downarrow  (A\downarrow (B \downarrow  B)\downarrow C\downarrow D) \downarrow  (A\downarrow (B \downarrow  B)\downarrow C\downarrow (D \downarrow  D)) \downarrow  (A\downarrow (B \downarrow  B)\downarrow (C \downarrow  C)\downarrow D) \downarrow  (A\downarrow (B \downarrow  B)\downarrow (C \downarrow  C)\downarrow (D \downarrow  D)) \downarrow  ((A \downarrow  A)\downarrow B\downarrow C\downarrow D) \downarrow  ((A \downarrow  A)\downarrow B\downarrow C\downarrow (D \downarrow  D)) \downarrow  ((A \downarrow  A)\downarrow (B \downarrow  B)\downarrow (C \downarrow  C)\downarrow D))$$$$


\paragraph{Logigramme de la fonction}

 \label{logigram-P10}
\begin{tikzpicture}

 %%Paramaters
%% var position, can be modified
\def\varPos{2.00}
\node (x) at (0, \varPos) {$A$};
\node (y) at (0.5, \varPos) {$B$};
\node (z) at (1, \varPos) {$C$};
\node (w) at (1.5, \varPos) {$D$};

            \node[nor gate US, draw, rotate=270, scale=0.5, logic gate inputs=nn] at ($(x) + (0.25, -0.6)$) (notx) {};
            \draw ($(x)+(0,-1ex)$) -| (notx.input 1); 
            \node[nor gate US, draw, rotate=270, scale=0.5, logic gate inputs=nn] at ($(y) + (0.25, -0.6)$) (noty) {};
            \draw ($(y)+(0,-1ex)$) -| (noty.input 1); 
            \node[nor gate US, draw, rotate=270, scale=0.5, logic gate inputs=nn] at ($(z) + (0.25, -0.6)$) (notz) {};
            \draw ($(z)+(0,-1ex)$) -| (notz.input 1);
            \node[nor gate US, draw, rotate=270, scale=0.5, logic gate inputs=nn] at ($(w) + (0.25, -0.6)$) (notw) {};
            \draw ($(w)+(0,-1ex)$) -| (notw.input 1);
         

      
                \node[nor gate US, draw, rotate=0, logic gate inputs=nnnn] at (2.5, 0.00) (xandy0) {};\draw (xandy0.output) -- node[above]{\scriptsize $ A.B $} ($(xandy0) + (1.8, 0)$);
                % X
\draw ($(x) + (0, -1ex)$)|- (xandy0.input 1);
% Y
\draw ($(y) + (0, -1ex)$)|- (xandy0.input 2);
 

      
                \node[nor gate US, draw, rotate=0, logic gate inputs=nnnn] at (2.5, 1.20) (xandy1) {};\draw (xandy1.output) -- node[above]{\scriptsize $ A.C $} ($(xandy1) + (1.8, 0)$);
                % X
\draw ($(x) + (0, -1ex)$)|- (xandy1.input 1);
% Z
\draw ($(z) + (0, -1ex)$)|- (xandy1.input 3);
\node[nor gate US, draw, rotate=0, logic gate inputs=nnn] at (6, 0.60) (xory0) {};


            \draw (xory0.output) -- node[above]{\scriptsize $P10$} ($(xory0.east) + (+3ex, 0)$);


            \draw (xandy0.output) -- ([xshift=1.60cm]xandy0.output) |- (xory0.input 2);

\draw (xandy1.output) -- ([xshift=1.55cm]xandy1.output) |- (xory0.input 1);

 \end{tikzpicture}


\pagebreak

\paragraph{Q1}


P10(A, B, C, D) = $[10, 11, 12, 13, 15]$


\hrule width 1\linewidth
\pagebreak

\subsection{Correction}


\paragraph{Q1}

P10(A, B, C, D) =$[10, 11, 12, 13, 15]$

P10(A, B, C, D) =$ \sum([10, 11, 12, 13, 15]) $ 




Sum of products 
 P10(A, B, C, D) = $A.\bar B.C.\bar D + A.\bar B.C.D + A.B.\bar C.\bar D + A.B.\bar C.D + A.B.C.D$


Product of sums 
 P10(A, B, C, D) = $(A+B+C+D) . (A+B+C+\bar D) . (A+B+\bar C+D) . (A+B+\bar C+\bar D) . (A+\bar B+C+D) . (A+\bar B+C+\bar D) . (A+\bar B+\bar C+D) . (A+\bar B+\bar C+\bar D) . (\bar A+B+C+D) . (\bar A+B+C+\bar D) . (\bar A+\bar B+\bar C+D)$


\paragraph{Karnough map}

\begin{karnaugh-map}[4][4][1][CD][AB]
          \minterms{10, 11, 12, 13, 15}
          \maxterms{0, 1, 2, 3, 5, 6, 7, 8}
        %\autoterms[0]
         \terms{4, 9, 14}{X}
        % simplification
        \implicant{12}{14}
\implicant{15}{10}
          %\implicant{5}{15}
          %\implicantedge{8}{8}{10}{10}
          %\implicantedge{8}{8}{10}{10}[8,10]
        \end{karnaugh-map}



Simplified Sum of products: $ a.b + a.c $


Simplified Product of sums: $(a).(b+c)$


NOR
 first simplified from: $$$$ a.b + a.c $$
$$(\overline{\overline{ a.b }})+(\overline{\overline{ a.c }})$$
$$\overline{\overline{(\overline{\overline{ a.b }})+(\overline{\overline{ a.c }})}}$$
$$( (A \downarrow  A)\downarrow (B \downarrow  B) )\downarrow ( (A \downarrow  A)\downarrow (C \downarrow  C) )\downarrow ( (A \downarrow  A)\downarrow (B \downarrow  B) )\downarrow ( (A \downarrow  A)\downarrow (C \downarrow  C) )$$$$


NOR
 second simplified from: $$$$(a).(b+c)$$
$$\overline{\overline{(a).(b+c)}}$$
$$\overline{(\overline{(a)}+\overline{(b+c)})}$$
$$((A)\downarrow (B\downarrow C))$$$$


NOR
 first from: $$$$A.\bar B.C.\bar D + A.\bar B.C.D + A.B.\bar C.\bar D + A.B.\bar C.D + A.B.C.D$$
$$(\overline{\overline{A.\bar B.C.\bar D }})+(\overline{\overline{ A.\bar B.C.D }})+(\overline{\overline{ A.B.\bar C.\bar D }})+(\overline{\overline{ A.B.\bar C.D }})+(\overline{\overline{ A.B.C.D}})$$
$$\overline{\overline{(\overline{\overline{A.\bar B.C.\bar D }})+(\overline{\overline{ A.\bar B.C.D }})+(\overline{\overline{ A.B.\bar C.\bar D }})+(\overline{\overline{ A.B.\bar C.D }})+(\overline{\overline{ A.B.C.D}})}}$$
$$((A \downarrow  A)\downarrow B\downarrow (C \downarrow  C)\downarrow D )\downarrow ( (A \downarrow  A)\downarrow B\downarrow (C \downarrow  C)\downarrow (D \downarrow  D) )\downarrow ( (A \downarrow  A)\downarrow (B \downarrow  B)\downarrow C\downarrow D )\downarrow ( (A \downarrow  A)\downarrow (B \downarrow  B)\downarrow C\downarrow (D \downarrow  D) )\downarrow ( (A \downarrow  A)\downarrow (B \downarrow  B)\downarrow (C \downarrow  C)\downarrow (D \downarrow  D))\downarrow ((A \downarrow  A)\downarrow B\downarrow (C \downarrow  C)\downarrow D )\downarrow ( (A \downarrow  A)\downarrow B\downarrow (C \downarrow  C)\downarrow (D \downarrow  D) )\downarrow ( (A \downarrow  A)\downarrow (B \downarrow  B)\downarrow C\downarrow D )\downarrow ( (A \downarrow  A)\downarrow (B \downarrow  B)\downarrow C\downarrow (D \downarrow  D) )\downarrow ( (A \downarrow  A)\downarrow (B \downarrow  B)\downarrow (C \downarrow  C)\downarrow (D \downarrow  D))$$$$


NOR
 second from: $$$$(A+B+C+D) . (A+B+C+\bar D) . (A+B+\bar C+D) . (A+B+\bar C+\bar D) . (A+\bar B+C+D) . (A+\bar B+C+\bar D) . (A+\bar B+\bar C+D) . (A+\bar B+\bar C+\bar D) . (\bar A+B+C+D) . (\bar A+B+C+\bar D) . (\bar A+\bar B+\bar C+D)$$
$$\overline{\overline{(A+B+C+D) . (A+B+C+\bar D) . (A+B+\bar C+D) . (A+B+\bar C+\bar D) . (A+\bar B+C+D) . (A+\bar B+C+\bar D) . (A+\bar B+\bar C+D) . (A+\bar B+\bar C+\bar D) . (\bar A+B+C+D) . (\bar A+B+C+\bar D) . (\bar A+\bar B+\bar C+D)}}$$
$$\overline{(\overline{(A+B+C+D) }+\overline{ (A+B+C+\bar D) }+\overline{ (A+B+\bar C+D) }+\overline{ (A+B+\bar C+\bar D) }+\overline{ (A+\bar B+C+D) }+\overline{ (A+\bar B+C+\bar D) }+\overline{ (A+\bar B+\bar C+D) }+\overline{ (A+\bar B+\bar C+\bar D) }+\overline{ (\bar A+B+C+D) }+\overline{ (\bar A+B+C+\bar D) }+\overline{ (\bar A+\bar B+\bar C+D)})}$$
$$((A\downarrow B\downarrow C\downarrow D) \downarrow  (A\downarrow B\downarrow C\downarrow (D \downarrow  D)) \downarrow  (A\downarrow B\downarrow (C \downarrow  C)\downarrow D) \downarrow  (A\downarrow B\downarrow (C \downarrow  C)\downarrow (D \downarrow  D)) \downarrow  (A\downarrow (B \downarrow  B)\downarrow C\downarrow D) \downarrow  (A\downarrow (B \downarrow  B)\downarrow C\downarrow (D \downarrow  D)) \downarrow  (A\downarrow (B \downarrow  B)\downarrow (C \downarrow  C)\downarrow D) \downarrow  (A\downarrow (B \downarrow  B)\downarrow (C \downarrow  C)\downarrow (D \downarrow  D)) \downarrow  ((A \downarrow  A)\downarrow B\downarrow C\downarrow D) \downarrow  ((A \downarrow  A)\downarrow B\downarrow C\downarrow (D \downarrow  D)) \downarrow  ((A \downarrow  A)\downarrow (B \downarrow  B)\downarrow (C \downarrow  C)\downarrow D))$$$$


\paragraph{Logigramme de la fonction}

 \label{logigram-P10}
\begin{tikzpicture}

 %%Paramaters
%% var position, can be modified
\def\varPos{2.00}
\node (x) at (0, \varPos) {$A$};
\node (y) at (0.5, \varPos) {$B$};
\node (z) at (1, \varPos) {$C$};
\node (w) at (1.5, \varPos) {$D$};

            \node[nor gate US, draw, rotate=270, scale=0.5, logic gate inputs=nn] at ($(x) + (0.25, -0.6)$) (notx) {};
            \draw ($(x)+(0,-1ex)$) -| (notx.input 1); 
            \node[nor gate US, draw, rotate=270, scale=0.5, logic gate inputs=nn] at ($(y) + (0.25, -0.6)$) (noty) {};
            \draw ($(y)+(0,-1ex)$) -| (noty.input 1); 
            \node[nor gate US, draw, rotate=270, scale=0.5, logic gate inputs=nn] at ($(z) + (0.25, -0.6)$) (notz) {};
            \draw ($(z)+(0,-1ex)$) -| (notz.input 1);
            \node[nor gate US, draw, rotate=270, scale=0.5, logic gate inputs=nn] at ($(w) + (0.25, -0.6)$) (notw) {};
            \draw ($(w)+(0,-1ex)$) -| (notw.input 1);
         

      
                \node[nor gate US, draw, rotate=0, logic gate inputs=nnnn] at (2.5, 0.00) (xandy0) {};\draw (xandy0.output) -- node[above]{\scriptsize $ A.B $} ($(xandy0) + (1.8, 0)$);
                % X
\draw ($(x) + (0, -1ex)$)|- (xandy0.input 1);
% Y
\draw ($(y) + (0, -1ex)$)|- (xandy0.input 2);
 

      
                \node[nor gate US, draw, rotate=0, logic gate inputs=nnnn] at (2.5, 1.20) (xandy1) {};\draw (xandy1.output) -- node[above]{\scriptsize $ A.C $} ($(xandy1) + (1.8, 0)$);
                % X
\draw ($(x) + (0, -1ex)$)|- (xandy1.input 1);
% Z
\draw ($(z) + (0, -1ex)$)|- (xandy1.input 3);
\node[nor gate US, draw, rotate=0, logic gate inputs=nnn] at (6, 0.60) (xory0) {};


            \draw (xory0.output) -- node[above]{\scriptsize $P10$} ($(xory0.east) + (+3ex, 0)$);


            \draw (xandy0.output) -- ([xshift=1.60cm]xandy0.output) |- (xory0.input 2);

\draw (xandy1.output) -- ([xshift=1.55cm]xandy1.output) |- (xory0.input 1);

 \end{tikzpicture}


\pagebreak

\paragraph{Q1}


P10(A, B, C, D) = $[10, 11, 12, 13, 15]$


\hrule width 1\linewidth
\pagebreak

\subsection{Correction}


\paragraph{Q1}

P10(A, B, C, D) =$[10, 11, 12, 13, 15]$

P10(A, B, C, D) =$ \sum([10, 11, 12, 13, 15]) $ 




Sum of products 
 P10(A, B, C, D) = $A.\bar B.C.\bar D + A.\bar B.C.D + A.B.\bar C.\bar D + A.B.\bar C.D + A.B.C.D$


Product of sums 
 P10(A, B, C, D) = $(A+B+C+D) . (A+B+C+\bar D) . (A+B+\bar C+D) . (A+B+\bar C+\bar D) . (A+\bar B+C+D) . (A+\bar B+C+\bar D) . (A+\bar B+\bar C+D) . (A+\bar B+\bar C+\bar D) . (\bar A+B+C+D) . (\bar A+B+C+\bar D) . (\bar A+\bar B+\bar C+D)$


\paragraph{Karnough map}

\begin{karnaugh-map}[4][4][1][CD][AB]
          \minterms{10, 11, 12, 13, 15}
          \maxterms{0, 1, 2, 3, 5, 6, 7, 8}
        %\autoterms[0]
         \terms{4, 9, 14}{X}
        % simplification
        \implicant{12}{14}
\implicant{15}{10}
          %\implicant{5}{15}
          %\implicantedge{8}{8}{10}{10}
          %\implicantedge{8}{8}{10}{10}[8,10]
        \end{karnaugh-map}



Simplified Sum of products: $ a.b + a.c $


Simplified Product of sums: $(a).(b+c)$


NOR
 first simplified from: $$$$ a.b + a.c $$
$$(\overline{\overline{ a.b }})+(\overline{\overline{ a.c }})$$
$$\overline{\overline{(\overline{\overline{ a.b }})+(\overline{\overline{ a.c }})}}$$
$$( (A \downarrow  A)\downarrow (B \downarrow  B) )\downarrow ( (A \downarrow  A)\downarrow (C \downarrow  C) )\downarrow ( (A \downarrow  A)\downarrow (B \downarrow  B) )\downarrow ( (A \downarrow  A)\downarrow (C \downarrow  C) )$$$$


NOR
 second simplified from: $$$$(a).(b+c)$$
$$\overline{\overline{(a).(b+c)}}$$
$$\overline{(\overline{(a)}+\overline{(b+c)})}$$
$$((A)\downarrow (B\downarrow C))$$$$


NOR
 first from: $$$$A.\bar B.C.\bar D + A.\bar B.C.D + A.B.\bar C.\bar D + A.B.\bar C.D + A.B.C.D$$
$$(\overline{\overline{A.\bar B.C.\bar D }})+(\overline{\overline{ A.\bar B.C.D }})+(\overline{\overline{ A.B.\bar C.\bar D }})+(\overline{\overline{ A.B.\bar C.D }})+(\overline{\overline{ A.B.C.D}})$$
$$\overline{\overline{(\overline{\overline{A.\bar B.C.\bar D }})+(\overline{\overline{ A.\bar B.C.D }})+(\overline{\overline{ A.B.\bar C.\bar D }})+(\overline{\overline{ A.B.\bar C.D }})+(\overline{\overline{ A.B.C.D}})}}$$
$$((A \downarrow  A)\downarrow B\downarrow (C \downarrow  C)\downarrow D )\downarrow ( (A \downarrow  A)\downarrow B\downarrow (C \downarrow  C)\downarrow (D \downarrow  D) )\downarrow ( (A \downarrow  A)\downarrow (B \downarrow  B)\downarrow C\downarrow D )\downarrow ( (A \downarrow  A)\downarrow (B \downarrow  B)\downarrow C\downarrow (D \downarrow  D) )\downarrow ( (A \downarrow  A)\downarrow (B \downarrow  B)\downarrow (C \downarrow  C)\downarrow (D \downarrow  D))\downarrow ((A \downarrow  A)\downarrow B\downarrow (C \downarrow  C)\downarrow D )\downarrow ( (A \downarrow  A)\downarrow B\downarrow (C \downarrow  C)\downarrow (D \downarrow  D) )\downarrow ( (A \downarrow  A)\downarrow (B \downarrow  B)\downarrow C\downarrow D )\downarrow ( (A \downarrow  A)\downarrow (B \downarrow  B)\downarrow C\downarrow (D \downarrow  D) )\downarrow ( (A \downarrow  A)\downarrow (B \downarrow  B)\downarrow (C \downarrow  C)\downarrow (D \downarrow  D))$$$$


NOR
 second from: $$$$(A+B+C+D) . (A+B+C+\bar D) . (A+B+\bar C+D) . (A+B+\bar C+\bar D) . (A+\bar B+C+D) . (A+\bar B+C+\bar D) . (A+\bar B+\bar C+D) . (A+\bar B+\bar C+\bar D) . (\bar A+B+C+D) . (\bar A+B+C+\bar D) . (\bar A+\bar B+\bar C+D)$$
$$\overline{\overline{(A+B+C+D) . (A+B+C+\bar D) . (A+B+\bar C+D) . (A+B+\bar C+\bar D) . (A+\bar B+C+D) . (A+\bar B+C+\bar D) . (A+\bar B+\bar C+D) . (A+\bar B+\bar C+\bar D) . (\bar A+B+C+D) . (\bar A+B+C+\bar D) . (\bar A+\bar B+\bar C+D)}}$$
$$\overline{(\overline{(A+B+C+D) }+\overline{ (A+B+C+\bar D) }+\overline{ (A+B+\bar C+D) }+\overline{ (A+B+\bar C+\bar D) }+\overline{ (A+\bar B+C+D) }+\overline{ (A+\bar B+C+\bar D) }+\overline{ (A+\bar B+\bar C+D) }+\overline{ (A+\bar B+\bar C+\bar D) }+\overline{ (\bar A+B+C+D) }+\overline{ (\bar A+B+C+\bar D) }+\overline{ (\bar A+\bar B+\bar C+D)})}$$
$$((A\downarrow B\downarrow C\downarrow D) \downarrow  (A\downarrow B\downarrow C\downarrow (D \downarrow  D)) \downarrow  (A\downarrow B\downarrow (C \downarrow  C)\downarrow D) \downarrow  (A\downarrow B\downarrow (C \downarrow  C)\downarrow (D \downarrow  D)) \downarrow  (A\downarrow (B \downarrow  B)\downarrow C\downarrow D) \downarrow  (A\downarrow (B \downarrow  B)\downarrow C\downarrow (D \downarrow  D)) \downarrow  (A\downarrow (B \downarrow  B)\downarrow (C \downarrow  C)\downarrow D) \downarrow  (A\downarrow (B \downarrow  B)\downarrow (C \downarrow  C)\downarrow (D \downarrow  D)) \downarrow  ((A \downarrow  A)\downarrow B\downarrow C\downarrow D) \downarrow  ((A \downarrow  A)\downarrow B\downarrow C\downarrow (D \downarrow  D)) \downarrow  ((A \downarrow  A)\downarrow (B \downarrow  B)\downarrow (C \downarrow  C)\downarrow D))$$$$


\paragraph{Logigramme de la fonction}

 \label{logigram-P10}
\begin{tikzpicture}

 %%Paramaters
%% var position, can be modified
\def\varPos{2.00}
\node (x) at (0, \varPos) {$A$};
\node (y) at (0.5, \varPos) {$B$};
\node (z) at (1, \varPos) {$C$};
\node (w) at (1.5, \varPos) {$D$};

            \node[nor gate US, draw, rotate=270, scale=0.5, logic gate inputs=nn] at ($(x) + (0.25, -0.6)$) (notx) {};
            \draw ($(x)+(0,-1ex)$) -| (notx.input 1); 
            \node[nor gate US, draw, rotate=270, scale=0.5, logic gate inputs=nn] at ($(y) + (0.25, -0.6)$) (noty) {};
            \draw ($(y)+(0,-1ex)$) -| (noty.input 1); 
            \node[nor gate US, draw, rotate=270, scale=0.5, logic gate inputs=nn] at ($(z) + (0.25, -0.6)$) (notz) {};
            \draw ($(z)+(0,-1ex)$) -| (notz.input 1);
            \node[nor gate US, draw, rotate=270, scale=0.5, logic gate inputs=nn] at ($(w) + (0.25, -0.6)$) (notw) {};
            \draw ($(w)+(0,-1ex)$) -| (notw.input 1);
         

      
                \node[nor gate US, draw, rotate=0, logic gate inputs=nnnn] at (2.5, 0.00) (xandy0) {};\draw (xandy0.output) -- node[above]{\scriptsize $ A.B $} ($(xandy0) + (1.8, 0)$);
                % X
\draw ($(x) + (0, -1ex)$)|- (xandy0.input 1);
% Y
\draw ($(y) + (0, -1ex)$)|- (xandy0.input 2);
 

      
                \node[nor gate US, draw, rotate=0, logic gate inputs=nnnn] at (2.5, 1.20) (xandy1) {};\draw (xandy1.output) -- node[above]{\scriptsize $ A.C $} ($(xandy1) + (1.8, 0)$);
                % X
\draw ($(x) + (0, -1ex)$)|- (xandy1.input 1);
% Z
\draw ($(z) + (0, -1ex)$)|- (xandy1.input 3);
\node[nor gate US, draw, rotate=0, logic gate inputs=nnn] at (6, 0.60) (xory0) {};


            \draw (xory0.output) -- node[above]{\scriptsize $P10$} ($(xory0.east) + (+3ex, 0)$);


            \draw (xandy0.output) -- ([xshift=1.60cm]xandy0.output) |- (xory0.input 2);

\draw (xandy1.output) -- ([xshift=1.55cm]xandy1.output) |- (xory0.input 1);

 \end{tikzpicture}


\pagebreak

\paragraph{Q1}


P10(A, B, C, D) = $[10, 11, 12, 13, 15]$


\hrule width 1\linewidth
\pagebreak

\subsection{Correction}


\paragraph{Q1}

P10(A, B, C, D) =$[10, 11, 12, 13, 15]$

P10(A, B, C, D) =$ \sum([10, 11, 12, 13, 15]) $ 




Sum of products 
 P10(A, B, C, D) = $A.\bar B.C.\bar D + A.\bar B.C.D + A.B.\bar C.\bar D + A.B.\bar C.D + A.B.C.D$


Product of sums 
 P10(A, B, C, D) = $(A+B+C+D) . (A+B+C+\bar D) . (A+B+\bar C+D) . (A+B+\bar C+\bar D) . (A+\bar B+C+D) . (A+\bar B+C+\bar D) . (A+\bar B+\bar C+D) . (A+\bar B+\bar C+\bar D) . (\bar A+B+C+D) . (\bar A+B+C+\bar D) . (\bar A+\bar B+\bar C+D)$


\paragraph{Karnough map}

\begin{karnaugh-map}[4][4][1][CD][AB]
          \minterms{10, 11, 12, 13, 15}
          \maxterms{0, 1, 2, 3, 5, 6, 7, 8}
        %\autoterms[0]
         \terms{4, 9, 14}{X}
        % simplification
        \implicant{12}{14}
\implicant{15}{10}
          %\implicant{5}{15}
          %\implicantedge{8}{8}{10}{10}
          %\implicantedge{8}{8}{10}{10}[8,10]
        \end{karnaugh-map}



Simplified Sum of products: $ a.b + a.c $


Simplified Product of sums: $(a).(b+c)$


NOR
 first simplified from: $$$$ a.b + a.c $$
$$(\overline{\overline{ a.b }})+(\overline{\overline{ a.c }})$$
$$\overline{\overline{(\overline{\overline{ a.b }})+(\overline{\overline{ a.c }})}}$$
$$( (A \downarrow  A)\downarrow (B \downarrow  B) )\downarrow ( (A \downarrow  A)\downarrow (C \downarrow  C) )\downarrow ( (A \downarrow  A)\downarrow (B \downarrow  B) )\downarrow ( (A \downarrow  A)\downarrow (C \downarrow  C) )$$$$


NOR
 second simplified from: $$$$(a).(b+c)$$
$$\overline{\overline{(a).(b+c)}}$$
$$\overline{(\overline{(a)}+\overline{(b+c)})}$$
$$((A)\downarrow (B\downarrow C))$$$$


NOR
 first from: $$$$A.\bar B.C.\bar D + A.\bar B.C.D + A.B.\bar C.\bar D + A.B.\bar C.D + A.B.C.D$$
$$(\overline{\overline{A.\bar B.C.\bar D }})+(\overline{\overline{ A.\bar B.C.D }})+(\overline{\overline{ A.B.\bar C.\bar D }})+(\overline{\overline{ A.B.\bar C.D }})+(\overline{\overline{ A.B.C.D}})$$
$$\overline{\overline{(\overline{\overline{A.\bar B.C.\bar D }})+(\overline{\overline{ A.\bar B.C.D }})+(\overline{\overline{ A.B.\bar C.\bar D }})+(\overline{\overline{ A.B.\bar C.D }})+(\overline{\overline{ A.B.C.D}})}}$$
$$((A \downarrow  A)\downarrow B\downarrow (C \downarrow  C)\downarrow D )\downarrow ( (A \downarrow  A)\downarrow B\downarrow (C \downarrow  C)\downarrow (D \downarrow  D) )\downarrow ( (A \downarrow  A)\downarrow (B \downarrow  B)\downarrow C\downarrow D )\downarrow ( (A \downarrow  A)\downarrow (B \downarrow  B)\downarrow C\downarrow (D \downarrow  D) )\downarrow ( (A \downarrow  A)\downarrow (B \downarrow  B)\downarrow (C \downarrow  C)\downarrow (D \downarrow  D))\downarrow ((A \downarrow  A)\downarrow B\downarrow (C \downarrow  C)\downarrow D )\downarrow ( (A \downarrow  A)\downarrow B\downarrow (C \downarrow  C)\downarrow (D \downarrow  D) )\downarrow ( (A \downarrow  A)\downarrow (B \downarrow  B)\downarrow C\downarrow D )\downarrow ( (A \downarrow  A)\downarrow (B \downarrow  B)\downarrow C\downarrow (D \downarrow  D) )\downarrow ( (A \downarrow  A)\downarrow (B \downarrow  B)\downarrow (C \downarrow  C)\downarrow (D \downarrow  D))$$$$


NOR
 second from: $$$$(A+B+C+D) . (A+B+C+\bar D) . (A+B+\bar C+D) . (A+B+\bar C+\bar D) . (A+\bar B+C+D) . (A+\bar B+C+\bar D) . (A+\bar B+\bar C+D) . (A+\bar B+\bar C+\bar D) . (\bar A+B+C+D) . (\bar A+B+C+\bar D) . (\bar A+\bar B+\bar C+D)$$
$$\overline{\overline{(A+B+C+D) . (A+B+C+\bar D) . (A+B+\bar C+D) . (A+B+\bar C+\bar D) . (A+\bar B+C+D) . (A+\bar B+C+\bar D) . (A+\bar B+\bar C+D) . (A+\bar B+\bar C+\bar D) . (\bar A+B+C+D) . (\bar A+B+C+\bar D) . (\bar A+\bar B+\bar C+D)}}$$
$$\overline{(\overline{(A+B+C+D) }+\overline{ (A+B+C+\bar D) }+\overline{ (A+B+\bar C+D) }+\overline{ (A+B+\bar C+\bar D) }+\overline{ (A+\bar B+C+D) }+\overline{ (A+\bar B+C+\bar D) }+\overline{ (A+\bar B+\bar C+D) }+\overline{ (A+\bar B+\bar C+\bar D) }+\overline{ (\bar A+B+C+D) }+\overline{ (\bar A+B+C+\bar D) }+\overline{ (\bar A+\bar B+\bar C+D)})}$$
$$((A\downarrow B\downarrow C\downarrow D) \downarrow  (A\downarrow B\downarrow C\downarrow (D \downarrow  D)) \downarrow  (A\downarrow B\downarrow (C \downarrow  C)\downarrow D) \downarrow  (A\downarrow B\downarrow (C \downarrow  C)\downarrow (D \downarrow  D)) \downarrow  (A\downarrow (B \downarrow  B)\downarrow C\downarrow D) \downarrow  (A\downarrow (B \downarrow  B)\downarrow C\downarrow (D \downarrow  D)) \downarrow  (A\downarrow (B \downarrow  B)\downarrow (C \downarrow  C)\downarrow D) \downarrow  (A\downarrow (B \downarrow  B)\downarrow (C \downarrow  C)\downarrow (D \downarrow  D)) \downarrow  ((A \downarrow  A)\downarrow B\downarrow C\downarrow D) \downarrow  ((A \downarrow  A)\downarrow B\downarrow C\downarrow (D \downarrow  D)) \downarrow  ((A \downarrow  A)\downarrow (B \downarrow  B)\downarrow (C \downarrow  C)\downarrow D))$$$$


\paragraph{Logigramme de la fonction}

 \label{logigram-P10}
\begin{tikzpicture}

 %%Paramaters
%% var position, can be modified
\def\varPos{2.00}
\node (x) at (0, \varPos) {$A$};
\node (y) at (0.5, \varPos) {$B$};
\node (z) at (1, \varPos) {$C$};
\node (w) at (1.5, \varPos) {$D$};

            \node[nor gate US, draw, rotate=270, scale=0.5, logic gate inputs=nn] at ($(x) + (0.25, -0.6)$) (notx) {};
            \draw ($(x)+(0,-1ex)$) -| (notx.input 1); 
            \node[nor gate US, draw, rotate=270, scale=0.5, logic gate inputs=nn] at ($(y) + (0.25, -0.6)$) (noty) {};
            \draw ($(y)+(0,-1ex)$) -| (noty.input 1); 
            \node[nor gate US, draw, rotate=270, scale=0.5, logic gate inputs=nn] at ($(z) + (0.25, -0.6)$) (notz) {};
            \draw ($(z)+(0,-1ex)$) -| (notz.input 1);
            \node[nor gate US, draw, rotate=270, scale=0.5, logic gate inputs=nn] at ($(w) + (0.25, -0.6)$) (notw) {};
            \draw ($(w)+(0,-1ex)$) -| (notw.input 1);
         

      
                \node[nor gate US, draw, rotate=0, logic gate inputs=nnnn] at (2.5, 0.00) (xandy0) {};\draw (xandy0.output) -- node[above]{\scriptsize $ A.B $} ($(xandy0) + (1.8, 0)$);
                % X
\draw ($(x) + (0, -1ex)$)|- (xandy0.input 1);
% Y
\draw ($(y) + (0, -1ex)$)|- (xandy0.input 2);
 

      
                \node[nor gate US, draw, rotate=0, logic gate inputs=nnnn] at (2.5, 1.20) (xandy1) {};\draw (xandy1.output) -- node[above]{\scriptsize $ A.C $} ($(xandy1) + (1.8, 0)$);
                % X
\draw ($(x) + (0, -1ex)$)|- (xandy1.input 1);
% Z
\draw ($(z) + (0, -1ex)$)|- (xandy1.input 3);
\node[nor gate US, draw, rotate=0, logic gate inputs=nnn] at (6, 0.60) (xory0) {};


            \draw (xory0.output) -- node[above]{\scriptsize $P10$} ($(xory0.east) + (+3ex, 0)$);


            \draw (xandy0.output) -- ([xshift=1.60cm]xandy0.output) |- (xory0.input 2);

\draw (xandy1.output) -- ([xshift=1.55cm]xandy1.output) |- (xory0.input 1);

 \end{tikzpicture}


\pagebreak